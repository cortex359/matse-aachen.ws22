
\section{Aufgabe 7}

Bestimmen Sie ein Polynom der Form $\displaystyle p( x) =ax^{3} +bx^{2} +cx+d$, das durch die Punkte $\displaystyle ( -1;2) ,\ ( 0;1) ,\ ( 1;2) ,\ ( 12;12)$ geht. Nutzen Sie zur Berechnung ein lineares Gleichungssystem.



\subsection{Lösung 7}


Gesucht ist ein Polynom $\displaystyle p( x) =ax^{3} +bx^{2} +cx+d$ für das gilt:

\begin{align*}
	p( -1) & =2 & p( 0) & =1\\
	p( 1) & =2 & p\left(\frac{1}{2}\right) & =\frac{1}{2}
\end{align*}

Daraus ergibt sich ein System von vier linearen Gleichungen:
\begin{equation*}
	\begin{array}{ r l c c l }
		p( 1) & =2 & \Leftrightarrow  & 2 & =a+b+c+d\\
		p( -1) & =2 & \Leftrightarrow  & 2 & =-a+ b-c+d\\
		p\left(\frac{1}{2}\right) & =\frac{1}{2} & \Leftrightarrow  & \frac{1}{2} & =a\cdot \frac{1}{8} +b\cdot \frac{1}{4} +c\cdot \frac{1}{2} +d\\
		p( 0) & =1 & \Leftrightarrow  & 1 & =d
	\end{array}
\end{equation*}
Dieses LGS der Form $\displaystyle P\cdot \vec{x} =\vec{y}$, mit der Koeffizientenmatrix $\displaystyle P\in \mathbb{K}^{n\times n}$, den gesuchten Koeffizienten $\displaystyle \vec{x} =( a,b,c,d)$ und dem Ergebnisvektor $\displaystyle \vec{y}$ lässt sich als \textit{erweiterte Matrix} $\displaystyle [ P|\vec{y}]$ darstellen
\begin{gather*}
	P:=\begin{pmatrix}
		1 & 1 & 1 & 1\\
		-1 & 1 & -1 & 1\\
		\frac{1}{8} & \frac{1}{4} & \frac{1}{2} & 1\\
		0 & 0 & 0 & 1
	\end{pmatrix} ,\ \vec{x} \ =\begin{pmatrix}
		a\\
		b\\
		c\\
		d
	\end{pmatrix} ,\ \vec{y} :=\begin{pmatrix}
		2\\
		2\\
		\frac{1}{2}\\
		1
	\end{pmatrix}\\
	\\
	[ P|\vec{y}] \ :\Leftrightarrow \ P\cdot \vec{x} =\vec{y}\\
	\\
	[ P|\vec{y}] \ =\ \begin{pmatrix}
		1 & 1 & 1 & 1 & 2\\
		-1 & 1 & -1 & 1 & 2\\
		\frac{1}{8} & \frac{1}{4} & \frac{1}{2} & 1 & \frac{1}{2}\\
		0 & 0 & 0 & 1 & 1
	\end{pmatrix} \ 
\end{gather*}
Die erweiterte Matrix $\displaystyle [ P|\vec{y}]$ lässt nun mit dem Gaußschen Eliminationsverfahren durch elementare Zeilenoperationen in die Zeilenstufenform und danach in die normalisierte Zeilenstufenform bringen:
\begin{align*}
	\begin{pmatrix}
		1 & 1 & 1 & 1 & 2\\
		-1 & 1 & -1 & 1 & 2\\
		\frac{1}{8} & \frac{1}{4} & \frac{1}{2} & 1 & \frac{1}{2}\\
		0 & 0 & 0 & 1 & 1
	\end{pmatrix} & \begin{array}{ l c }
		& \\
		+\ I & \\
		\cdot 8 & -I\\
		& 
	\end{array}\\
	\begin{pmatrix}
		1 & 1 & 1 & 1 & 2\\
		0 & 2 & 0 & 2 & 4\\
		0 & 1 & 3 & 7 & 2\\
		0 & 0 & 0 & 1 & 1
	\end{pmatrix} & \begin{array}{ c c }
		-IV & \\
		:2 & -IV\\
		\cdot 2 & -II\\
		& 
	\end{array}\\
	\begin{pmatrix}
		1 & 1 & 1 & 0 & 1\\
		0 & 1 & 0 & 0 & 1\\
		0 & 0 & 6 & 12 & 0\\
		0 & 0 & 0 & 1 & 1
	\end{pmatrix} & \begin{array}{ c c }
		& \\
		& \\
		:6 & -2\cdot IV\\
		& 
	\end{array}\\
	\begin{pmatrix}
		1 & 1 & 1 & 0 & 1\\
		0 & 1 & 0 & 0 & 1\\
		0 & 0 & 1 & 0 & -2\\
		0 & 0 & 0 & 1 & 1
	\end{pmatrix} & \begin{array}{ c c }
		-III & -II\\
		& \\
		& \\
		& 
	\end{array}\\
	\begin{pmatrix}
		1 & 0 & 0 & 0 & 2\\
		0 & 1 & 0 & 0 & 1\\
		0 & 0 & 1 & 0 & -2\\
		0 & 0 & 0 & 1 & 1
	\end{pmatrix} & \begin{array}{ c c }
		& \\
		& \\
		& \\
		& 
	\end{array}
\end{align*}
Daraus lassen sich nun die Koeffizienten $\displaystyle \vec{x}$ des gesuchten Polynoms $\displaystyle p( x)$ ablesen
\begin{equation*}
	\begin{array}{ c r c }
		& \begin{pmatrix}
			1 & 0 & 0 & 0\\
			0 & 1 & 0 & 0\\
			0 & 0 & 1 & 0\\
			0 & 0 & 0 & 1
		\end{pmatrix} \ \cdot \begin{pmatrix}
			a\\
			b\\
			c\\
			d
		\end{pmatrix} & =\begin{pmatrix}
			2\\
			1\\
			-2\\
			1
		\end{pmatrix}\\
		\Leftrightarrow  & \begin{pmatrix}
			a\\
			b\\
			c\\
			d
		\end{pmatrix} & =\begin{pmatrix}
			2\\
			1\\
			-2\\
			1
		\end{pmatrix}
	\end{array}
\end{equation*}
$\displaystyle \Rightarrow $	$\displaystyle p( x) =2x^{3} +x^{2} -2x+1$