\section{Aufgabe 9}

Die erweiterte Matrix des linearen Gleichungssystems $\displaystyle Ax=b$ mit $\displaystyle \alpha ,\beta \in \mathbb{R}$ sei:


$\displaystyle ( A|b) =\begin{pmatrix}
	\alpha  & 0 & \beta  & 2\\
	\alpha  & \alpha  & 4 & 4\\
	0 & \alpha  & 2 & \beta 
\end{pmatrix}$



Für welche $\displaystyle \alpha $ und $\displaystyle \beta $ hat das System
\begin{itemize}
	\item eine eindeutige Lösung?
	\item eine einparametrige Lösung?
	\item eine zweiparametrige Lösung?
	\item keine Lösung?
\end{itemize}



\subsection{Lösung 9}

Zunächst wird die erweiterte Matrix $\displaystyle ( A|b)$ in die Zeilenstufenfrom gebracht:
\begin{align*}
	\begin{pmatrix}
		\alpha  & 0 & \beta  & 2\\
		\alpha  & \alpha  & 4 & 4\\
		0 & \alpha  & 2 & \beta 
	\end{pmatrix} & \begin{array}{ l c }
		& \\
		-\ I & \\
		& 
	\end{array}\\
	\begin{pmatrix}
		\alpha  & 0 & \beta  & 2\\
		0 & \alpha  & 4-\beta  & 2\\
		0 & \alpha  & 2 & \beta 
	\end{pmatrix} & \begin{array}{ l c }
		& \\
		-III & \\
		& 
	\end{array}\\
	\begin{pmatrix}
		\alpha  & 0 & \beta  & 2\\
		0 & 0 & 2-\beta  & 2-\beta \\
		0 & \alpha  & 2 & \beta 
	\end{pmatrix} & \begin{array}{ l c }
		& \\
		\rotatebox[origin=c]{270}{$\Rsh$}  & \\
		\rotatebox[origin=c]{270}{$\Lsh$}  & 
	\end{array}\\
	\begin{pmatrix}
		\alpha  & 0 & \beta  & 2\\
		0 & \alpha  & 2 & \beta \\
		0 & 0 & 2-\beta  & 2-\beta 
	\end{pmatrix} & \begin{array}{ l c }
		& \\
		& \\
		:( 2-\beta ) & 
	\end{array}\\
	\begin{pmatrix}
		\alpha  & 0 & \beta  & 2\\
		0 & \alpha  & 2 & \beta \\
		0 & 0 & 1 & 1
	\end{pmatrix} & \begin{array}{ l c }
		& \\
		-2\cdot III & \\
		& 
	\end{array}\\
	\begin{pmatrix}
		\alpha  & 0 & \beta  & 2\\
		0 & \alpha  & 0 & \beta -2\\
		0 & 0 & 1 & 1
	\end{pmatrix} & \begin{array}{ l c }
		& \\
		-2\cdot III & \\
		& 
	\end{array}\\
	\begin{pmatrix}
		\alpha  & 0 & 0 & 2-\beta \\
		0 & \alpha  & 0 & \beta -2\\
		0 & 0 & 1 & 1
	\end{pmatrix} & \begin{array}{ l c }
		& \\
		& \\
		& 
	\end{array}
\end{align*}
Daraus ergeben sich die folgenden Gleichungen:
\begin{gather*}
	x_{1} =\frac{2-\beta }{\alpha }\\
	\\
	x_{2} =\frac{\beta -2}{\alpha }\\
	\\
	x_{3} =1
\end{gather*}
Es lassen sich drei relevant zu unterscheidende Fälle erkennen.



\textbf{Fall 1} für $\displaystyle \alpha =0\land \beta\neq 2$ lässt sich zu einem Widerspruch führen
\begin{equation*}
	x_{1} \cdot 0=2-\beta \ \lightning 
\end{equation*}
und damit zeigen, dass das Gleichungssystem \textit{überbestimmt} ist, also keine Lösung existiert.

$\displaystyle \Rightarrow \mathbb{L}_{1} =\emptyset $



\textbf{Fall 2} für $\displaystyle \alpha =0\land \beta=2$
\begin{align*}
	\begin{pmatrix}
		0 & 0 & 0 & 0\\
		0 & 0 & 0 & 0\\
		0 & 0 & 1 & 1
	\end{pmatrix} & 
\end{align*}
ist das Gleichungssystem \textit{unterbestimmt}, also existieren unendlich viele Lösungen, da $\displaystyle x_{1}$ und $\displaystyle x_{2}$ einen beliebigen Wert annehmen können. Man könnte auch von einer zweiparametrigen Lösung sprechen:

$\displaystyle \Rightarrow \mathbb{L}_{2} =\left\{\begin{pmatrix}
	0\\
	0\\
	1
\end{pmatrix} +x_{1} \cdot \begin{pmatrix}
	1\\
	0\\
	0
\end{pmatrix} +x_{2} \cdot \begin{pmatrix}
	0\\
	1\\
	0
\end{pmatrix}\Bigl| x_{1} ,x_{2} \in \mathbb{R}\right\}$



\textbf{Fall 3} für $\displaystyle \alpha \neq 0\land \beta \neq 2$ 
\begin{equation*}
	\begin{array}{ c c c c c }
		& x_{2} & = & \frac{\beta -2}{\alpha } & \\
		\Leftrightarrow  & \alpha  & = & \frac{\beta -2}{x_{2}} & \\
		& x_{1} & = & \frac{2-\beta }{\alpha } & \Bigl|\text{ Einsetzen}\\
		\Leftrightarrow  & x_{1} & = & \frac{( 2-\beta ) \cdot x_{2}}{\beta -2} & \\
		\Leftrightarrow  & x_{1} & = & x_{2} \cdot \frac{2-\beta }{\beta -2} & \\
		\Leftrightarrow  & x_{1} \cdot \frac{\beta -2}{2-\beta } & = & x_{2} &
	\end{array}
\end{equation*}
existieren zwei äquivalente einparametrige Lösungen. Sowohl in der Schreibweise mit dem Parameter $\displaystyle x_{2}$, als auch mit dem Parameter $\displaystyle x_{1}$:

$\displaystyle \Rightarrow \mathbb{L}_{3} =\left\{\begin{pmatrix}
	0\\
	0\\
	1
\end{pmatrix} +x_{2} \cdot \begin{pmatrix}
	\frac{2-\beta }{\beta -2}\\
	1\\
	0
\end{pmatrix}\Bigl| x_{2} \in \mathbb{R}\right\} =\left\{\begin{pmatrix}
	0\\
	0\\
	1
\end{pmatrix} +x_{1} \cdot \begin{pmatrix}
	\frac{\beta -2}{2-\beta }\\
	1\\
	0
\end{pmatrix}\Bigl| x_{1} \in \mathbb{R}\right\}$



\textbf{Fall 4} für $\displaystyle \alpha \neq 0\land \beta =2$ existiert eine eindeutige Lösung



$\displaystyle \Rightarrow \mathbb{L}_{4} =\left\{\begin{pmatrix}
	0\\
	0\\
	1
\end{pmatrix}\right\}$