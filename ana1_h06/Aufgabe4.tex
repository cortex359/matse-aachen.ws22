\documentclass[main.tex]{subfiles}

\begin{document}

\section{Aufgabe 4}
Gegeben sei die Funktion
\begin{equation*}
    f(x) =x^{4} -5x^{2} + 4
\end{equation*}

\begin{enumerate}
    \item[a)] Beweisen Sie, dass $f(x)$ mindestens eine Nullstelle im Intervall $\left[ -\frac{3}{2} ;\frac{1}{2}\right]$ besitzt.
    \item[b)] Welche Auswirkung hat die Vergrößerung des zu untersuchenden Intervall auf $\left[ -\frac{5}{2} ;\frac{1}{2}\right]$? Was bedeutet dies für die Nullstellensuche?
    \item[c)] Wie viele Nullstellen kann ein Polynom $n$-ten Gerades maximal haben?
\end{enumerate}

\subsection{Lösung 4}
\subsubsection{Lösung 4a)}
\subsubsection{Lösung 4b)}
\subsubsection{Lösung 4c)}
Ein Polynom $p$ vom Grad $n$ kann keine oder endlich viele, aber maximal $n$ verschiedene Nullstellen haben $x_1,x_2,\ldots, x_r\ (r \leq n)$.


\end{document}
