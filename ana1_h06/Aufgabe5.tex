\documentclass[main.tex]{subfiles}

\begin{document}

\section{Aufgabe 5}
Gegeben ist die Funktion
\begin{equation*}
    f(x) =\frac{1}{x} + 2
\end{equation*}

\begin{enumerate}
    \item[a)] Zeigen Sie, dass die Funktion auf dem Intervall $[2;5]$ die Voraussetzungen des Fixpunktsatzes erfüllt.
    \item[b)] Mit dem Startpunkt $x_{0} = 3$ berechnen Sie mit der a-priori-Abschätzung die notwendige Anzahl Iterationen, um den Fixpunkt mit der Genauigkeit $\epsilon =\frac{1}{1.000}$ zu berechnen.
    \item[c)] Mit demselben Startpunkt und derselben verlangten Genauigkeit, berechnen Sie die Iterationen, bis mit der a-posteriori Abschätzung die Genauigkeit erreicht ist.
\end{enumerate}

Sie dürfen die Monotonie der Funktion ausnutzen.

\subsection{Lösung 5}

\end{document}
