\documentclass[main.tex]{subfiles}

\begin{document}

\section{Aufgabe 2}
Entscheiden Sie, ob die folgende Funktion auf dem Intervall $[-1;1]$ gleichmäßig stetig ist

\begin{equation*}
    f(x) =\frac{x}{4-x^{2}}
\end{equation*}

\subsection{Lösung 2}

\textcolor[rgb]{0.4,0.72,0.06}{Die Funktion }\textcolor[rgb]{0.4,0.72,0.06}{$\displaystyle f( x)$}\textcolor[rgb]{0.4,0.72,0.06}{ ist an der Stelle }\textcolor[rgb]{0.4,0.72,0.06}{$\displaystyle x_{0}$}\textcolor[rgb]{0.4,0.72,0.06}{ stetig, falls}
\textcolor[rgb]{0.4,0.72,0.06}{\begin{equation*}
    \forall \epsilon  >0\exists \delta ( x_{0} ,\epsilon )  >0\ :\ \forall ( x-x_{0}) < \delta \ :\ | f( x) -f( x_{0})| < \epsilon \text{.}
\end{equation*}}

Die Funktion $f( x)$ ist gleichmäßig stetig in einem Intervall $D$, wenn in jedem Punkt $x_{0} \in D$ gilt:
\begin{equation*}
    \forall \epsilon  >0\exists \delta ( \epsilon )  >0\forall x,x_{0} \in D:(| x-x_{0}| < \delta ( \epsilon ) \Longrightarrow | f( x) -f( x_{0})| < \epsilon )\text{.}
\end{equation*}
Hinweis: Dabei muss $\delta $ unabhängig von $x_{0}$ sein.

\tikzset{every picture/.style={line width=0.75pt}} %set default line width to 0.75pt

% \begin{tikzpicture}
% %\draw (10,15.4) node [anchor=north west][inner sep=0.75pt]  [color={rgb, 255:red, 74; green, 144; blue, 226 }  ,opacity=1 ]  {${\displaystyle f( x) =\frac{x}{4-x}}$};
% \begin{axis}[
%     width=\linewidth, % Scale the plot to \linewidth
%     axis lines = left,
%     xlabel = \(x\),
%     ylabel = {\(f(x)=\frac{x}{4-x^2}\)},
%     xmin = -3, xmax = 3,
%     ymin = -2, ymax = 2,
%     grid=major,
%     grid style={dashed,gray!30},
% ]
% \addplot [
%     domain=-3:0:3,
%     samples=100,
%     color=td-blue
% ]{x/(4-(x*x))};
% %\addlegendentry{\(f(x) =\frac{x}{4-x^2}\)}
% \end{axis}
% \end{tikzpicture}

\begin{figure}
    \includesvg{fig1.svg}
    \caption{Graph 1}
\end{figure}

Stetigkeit mit $\epsilon $-$\delta $-Beweis zu zeigen. Sei $\epsilon  >0$ und $x_{0} \in [ -1;1]$


\end{document}
