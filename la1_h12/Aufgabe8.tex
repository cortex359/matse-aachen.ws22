\documentclass[main.tex]{subfiles}

\begin{document}

\section{Aufgabe 8}
Folgende Vektoren spannen einen Unterraum $U$ des $\mathbb{R}^{4}$ auf:
\begin{equation*}
V=\left\{\begin{pmatrix}[1]
1\\
0\\
2\\
-2
\end{pmatrix} ,\begin{pmatrix}[1]
2\\
1\\
4\\
-4
\end{pmatrix} ,\begin{pmatrix}[1]
0\\
1\\
-1\\
2
\end{pmatrix}\right\}
\end{equation*}

\begin{enumerate}
    \item Bestimmen Sie aus $V$ eine Orthonormalbasis von $U$, falls dies möglich ist.
    \item Welche Dimension hat das orthogonale Komplement $U^{\perp }$?
\end{enumerate}

\subsection{Lösung 8a}
Orthogonalisierungsverfahren nach Gram-Schmidt.
\begin{equation*}
\begin{array}{ c l }
w_{1} & :=\frac{v_{1}}{\| v_{1} \| }\\
r_{k+1} & :=v_{k+1} -\sum\limits _{i=1}^{k}< v_{k+1} ,w_{i}> w_{i}\\
w_{k+1} & =\frac{r_{k+1}}{\| r_{k+1} \| }
\end{array}
\end{equation*}
Sei $V$ ein unitärer Vektorraum und die Vektoren $v_{1} ,\dotsc ,v_{m}$ linear unabhängig, dann bilden $w_{1} ,\dotsc ,w_{m}$ eine Orthonormalbasis von $\mathcal{L}( v_{1} ,\dotsc ,v_{m})$.
\begin{gather*}
\begin{array}{ c l }
w_{1} & :=\frac{v_{1}}{\| v_{1} \| }\\
 & =\frac{v_{1}}{3}\\
 & =\begin{pmatrix}[1]
1/3\\
0\\
2/3\\
-2/3
\end{pmatrix}\\
r_{2} & :=v_{2} -< v_{2} ,w_{1}> w_{1}\\
 & =\begin{pmatrix}[1]
2\\
1\\
4\\
-4
\end{pmatrix} -6\cdotp \begin{pmatrix}[1]
1/3\\
0\\
2/3\\
-2/3
\end{pmatrix}\\
 & =\begin{pmatrix}[1]
0\\
1\\
0\\
0
\end{pmatrix}\\
w_{2} & =\frac{r_{2}}{\| r_{2} \| }\\
 & =\begin{pmatrix}[1]
0\\
1\\
0\\
0
\end{pmatrix}\\
r_{3} & :=v_{3} -(< v_{3} ,w_{1}> w_{1} +< v_{3} ,w_{2}> w_{2})\\
 & =\begin{pmatrix}[1]
0\\
1\\
-1\\
2
\end{pmatrix} -\left(\begin{pmatrix}[1]
-2/3\\
0\\
-4/3\\
4/3
\end{pmatrix} +\begin{pmatrix}[1]
0\\
1\\
0\\
0
\end{pmatrix}\right)\\
 & =\begin{pmatrix}[1]
2/3\\
0\\
1/3\\
2/3
\end{pmatrix}\\
w_{3} & =\frac{r_{3}}{\| r_{3} \| }\\
 & =\begin{pmatrix}[1]
2/3\\
0\\
1/3\\
2/3
\end{pmatrix}
\end{array}\\
\end{gather*}
\begin{gather*}
    W=\left\{\begin{pmatrix}[1]
    1/3\\
    0\\
    2/3\\
    -2/3
    \end{pmatrix} ,\begin{pmatrix}[1]
    0\\
    1\\
    0\\
    0
    \end{pmatrix} ,\begin{pmatrix}[1]
    2/3\\
    0\\
    1/3\\
    2/3
    \end{pmatrix}\right\}
\end{gather*}
\textbf{Probe:}

Lineare Unabhängigkeit:

Die Vektoren $w_{1} ,w_{2}$ und $w_{3}$ sind linear unabhängig.

Normiertheit:

 $\| w_{1} \| =\sqrt{\frac{1}{9} +\frac{4}{9} +\frac{4}{9}} =1\land \| w_{2} \| =1\land \| w_{3} \| =\sqrt{\frac{4}{9} +\frac{1}{9} +\frac{4}{9} =1} \ \checked $

Orthogonalität:

$< w_{1} ,w_{2}> =0\land < w_{2} ,w_{3}> =0\ \land < w_{1} ,w_{3}> =0\ \checked $



$\Rightarrow $ $W$ ist eine Orthonormalbasis von $U$.

\subsection{Lösung 8b}
Nach Folgerung 3.145 gilt für einen endlich erzeugten unitären Vektorraum $V$ und einem beliebigen Untervektorraum $U$
\begin{equation*}
    \dim( V) =\dim( U) +\dim\left( U^{\perp }\right) .
\end{equation*}
Damit gilt auch $\dim\left( U^{\perp }\right) =\ \dim( V) -\dim( U)$, was bedeutet, dass $\dim\left( U^{\perp }\right) =1$

\end{document}
