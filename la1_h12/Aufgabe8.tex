\documentclass[main.tex]{subfiles}

\begin{document}

\section{Aufgabe 8}
Folgende Vektoren spannen einen Unterraum $U$ des $\mathbb{R}^{4}$ auf:
\begin{equation*}
	V = \left\{
		\vektor{ 1 \\ 0 \\ 2 \\ -2 },
		\vektor{2 \\ 1 \\ 4 \\ -4},
		\vektor{0 \\1 \\ -1 \\ 2 }
	\right\}
\end{equation*}

\begin{enumerate}
    \item Bestimmen Sie aus $V$ eine Orthonormalbasis von $U$, falls dies möglich ist.
    \item Welche Dimension hat das orthogonale Komplement $U^{\perp }$?
\end{enumerate}

\subsection{Lösung 8a}
Orthogonalisierungsverfahren nach Gram-Schmidt.
\begin{equation*}
	\begin{array}{ c l }
		w_{1}   & := \frac{v_{1}}{\norm{ v_{1} } }\\
		r_{k+1} & := v_{k+1} -\sum\limits _{i=1}^{k}\scalarprod{ v_{k+1} ,w_{i}} w_{i}\\
		w_{k+1} &  = \frac{r_{k+1}}{\norm{ r_{k+1} } }
	\end{array}
\end{equation*}

Sei $V$ ein unitärer Vektorraum und die Vektoren $v_{1} ,\dotsc ,v_{m}$ linear unabhängig, dann bilden $w_{1} ,\dotsc ,w_{m}$ eine Orthonormalbasis von $\mathcal{L}( v_{1} ,\dotsc ,v_{m})$.

\begin{gather*}
	\begin{array}{ c l }
		w_{1} & :=\frac{v_{1}}{\norm{ v_{1} } }\\
		 & =\frac{v_{1}}{3}\\
		 & =\vektor{ 1/3 \\ 0 \\ 2/3 \\ -2/3 }\\
		r_{2} & :=v_{2} -\scalarprod{ v_{2} ,w_{1}} w_{1}\\
		 & =\vektor{ 2 \\ 1 \\ 4 \\ -4 } -6\cdotp \vektor{ 1/3 \\ 0 \\ 2/3 \\ -2/3 }\\
		 & =\vektor{ 0 \\ 1 \\ 0 \\ 0 }\\
		w_{2} & =\frac{r_{2}}{\norm{ r_{2} } }\\
		 & =\vektor{ 0 \\ 1 \\ 0 \\ 0 }\\
		r_{3} & :=v_{3} -(\scalarprod{ v_{3} ,w_{1}} w_{1} +\scalarprod{ v_{3} ,w_{2}} w_{2})\\
		 & =\vektor{ 0 \\ 1 \\ -1 \\ 2 } -\left(\vektor{ -2/3 \\ 0 \\ -4/3 \\ 4/3 } + \vektor{ 0 \\ 1 \\ 0 \\ 0 }\right)\\
		 & =\vektor{ 2/3 \\ 0 \\ 1/3 \\ 2/3 }\\
		w_{3} & =\frac{r_{3}}{\norm{ r_{3} } }\\
		 & =\vektor{ 2/3 \\ 0 \\ 1/3 \\ 2/3 }
	\end{array}\\
\end{gather*}
\begin{gather*}
    W=\left\{\vektor{ 1/3 \\ 0 \\ 2/3 \\ -2/3 } ,\vektor{ 0 \\ 1 \\ 0 \\ 0 } ,\vektor{ 2/3 \\ 0 \\ 1/3 \\ 2/3 }\right\}
\end{gather*}
\textbf{Probe:}

Lineare Unabhängigkeit:

Die Vektoren $w_{1} ,w_{2}$ und $w_{3}$ sind linear unabhängig.

Normiertheit:

 $\norm{ w_{1} } =\sqrt{\frac{1}{9} +\frac{4}{9} +\frac{4}{9}} =1\land \norm{ w_{2} } =1\land \norm{ w_{3} } =\sqrt{\frac{4}{9} +\frac{1}{9} +\frac{4}{9} =1} \ \checked $

Orthogonalität:

$\scalarprod{ w_{1} ,w_{2}} =0\land \scalarprod{ w_{2} ,w_{3}} =0\ \land \scalarprod{ w_{1} ,w_{3}} =0\ \checked $



$\Rightarrow $ $W$ ist eine Orthonormalbasis von $U$.

\subsection{Lösung 8b}
Nach Folgerung 3.145 gilt für einen endlich erzeugten unitären Vektorraum $V$ und einem beliebigen Untervektorraum $U$
\begin{equation*}
    \dim( V) =\dim( U) +\dim\left( U^{\perp }\right).
\end{equation*}
Damit gilt auch $\dim\left( U^{\perp }\right) =\ \dim( V) -\dim( U)$, was bedeutet, dass $\dim\left( U^{\perp }\right) =1$

\end{document}
