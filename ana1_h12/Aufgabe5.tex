\documentclass[main.tex]{subfiles}

\begin{document}

\section{Aufgabe 5}
Zeigen Sie:

\begin{enumerate}
    \item Ist die Funktion $f( x) =\sqrt{x}$ im Intervall $[ 1;3]$ selbstkontrahierend?
    \item Ist die Funktion $f( x) =x^{2} -2x$ im Intervall $[ 0;2]$ selbstkontrahierend?
\end{enumerate}

\subsection{Lösung 5}
Fixpunktsatz:

Sei $f:[ a;b]\rightarrow [ c;d]$ stetig mit $[ c;d] \subset [ a;b]$ (selbstkontrahierend), dann exisitert ein Fixpunkt $u=f( x)$.

\subsection{Lösung 5a}
Die Wurzelfunktion ist bekannterweise stetig und monoton wachsen für $x >0$, also ist sie auch im Intervall $[ 1;3]$ stetig und monoton wachsend.

Setze die Grenzen ein:
\begin{gather*}
f( 1) =1\in [ 1;3]\\
f( 3) =\sqrt{3} \in [ 1;3]
\end{gather*}
Somit ist $\left[ 1;\sqrt{3}\right] \subset [ 1;3]$ und die Funktion selbstkontrahierend.

\subsection{Lösung 5b}

\textit{Selbstkontraktion:}
\begin{equation*}
f( 1) =-1\notin [ 0;2]
\end{equation*}
Somit ist die Funktion nicht selbstkontrahierend.

\end{document}
