\documentclass[main.tex]{subfiles}

\begin{document}

\section{Aufgabe 3}
Untersuchen Sie das folgende Integral auf Konvergenz
\begin{equation*}
    \int\limits _{3}^{\infty }\frac{x-2}{x^{4} -x^{3} +13x^{2} -4x+3} \ dx
\end{equation*}

\subsection{Lösung 3}
Nach dem Integralkriterium können wir das Konvergenzverhalten dieses Integrals untersuchen, in dem wir die Reihe auf konvergenz untersuchen.

\begin{gather*}
    \lim _{a\rightarrow \infty }\int\limits _{3}^{a}\frac{x-2}{x^{4} -x^{3} +13x^{2} -4x+3} \ dx\\
    \\
    \sum _{x=3}^{\infty }\frac{x-2}{x^{4} -x^{3} +13x^{2} -4x+3}
\end{gather*}

Aus dem Majorantenkriterium folgt, dass wenn
\begin{equation*}
    \forall x\in \mathbb{N} ,x\geq 3:\frac{x-2}{x^{4} -x^{3} +13x^{2} -4x+3} \leq \frac{1}{x^{2}} \ 
\end{equation*}auch

\begin{equation*}
\sum _{n=3}^{\infty }\frac{x-2}{x^{4} -x^{3} +13x^{2} -4x+3} \leq \sum _{n=3}^{\infty }\frac{1}{x^{2}}
\end{equation*}
gelten muss.

Weil die Reihe $\sum _{n=1}^{\infty }\frac{1}{n^{2}}$ nach $\frac{\pi ^{2}}{6}$ konvergiert und für die Reihe ab Laufindex $n=3$ gilt
\begin{equation*}
    \sum _{n=3}^{\infty }\frac{1}{n^{2}} =\sum _{n=1}^{\infty }\frac{1}{n^{2}} -\frac{5}{4} =\frac{\pi ^{2}}{6} -\frac{5}{4}
\end{equation*}
muss auch die zu untersuchende Reihe konvergieren und damit auch das Integral.

\end{document}
