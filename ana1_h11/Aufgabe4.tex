\documentclass[main.tex]{subfiles}

\begin{document}

\section{Aufgabe 4}
Untersuchen Sie mit dem Integralkriterium die Konvergenz der folgenden Reihen:

\begin{enumerate}
    \item $\sum\limits _{n=1}^{\infty }\frac{1}{\sqrt[3]{n}}$
    \item $\sum\limits _{n=1}^{\infty }\frac{\ln( n)}{n}$
\end{enumerate}

\subsection{Lösung 4}
Integralkriterium, Satz 210:

Sei $f$ eine auf dem Intervall $[ m-1;\infty )$ monoton fallende Funktion mit $f( x) \geq 0\ \forall x\in [ m;\infty )$, dann ist die Reihe $\sum _{n=m}^{\infty } f( n)$ genau dann konvergent, wenn $\int _{m}^{\infty } f( x) \ dx$ existiert.

\subsection{Lösung 4a}
Wir betrachten die Funktion $f( n) =n^{-3/2}$ im Intervall $[ 0;\infty )$ und bilden ihre erste Ableitung $f'( n) =-\frac{3}{2} n^{-5/2}$ um zu zeigen, dass für $n\geq 0:f'( n) < 0$ gilt. Die Funktion ist somit monoton fallend. Außerdem ist $f( n) \geq 0$ im Intervall $[ 1;\infty )$, sodass das Integralkriterium angewandt werden kann.

Zur Untersuchung der Konvergenz der gegebenen Reihe betrachten wir also das folgende Integral:
\begin{equation*}
\begin{array}{ c l }
\int\limits _{1}^{\infty }\frac{1}{\sqrt[3]{n}} \ dn & =\lim\limits _{a\rightarrow \infty }\int\limits _{1}^{a} n^{-1/3} \ dn\\
 & =\lim\limits _{a\rightarrow \infty }\left[\frac{3}{2} n^{2/3}\right]_{1}^{a}\\
 & =\lim\limits _{a\rightarrow \infty }\frac{3}{2} a^{2/3} -\frac{3}{2}\\
 & =-\frac{3}{2} +\frac{3}{2} \cdotp \lim\limits _{a\rightarrow \infty }\underbrace{a^{2/3}}_{\rightarrow \infty }\\
 & =\infty 
\end{array}
\end{equation*}
Das Integral divergiert und somit divergiert die Reihe nach dem Integralkriterium ebenfalls.


\subsection{Lösung 4b}
Die Funktion $f( n) =\frac{\ln( n)}{n}$ ist für $n\geq 1$ positiv, jedoch erkennen wir an $f'( n) =\frac{1-\ln( n)}{n^{2}}$, dass $f'( n) \leq 0\ \forall n\geq e$ gilt, die Funktion also nur im Intervall $[ e;\infty )$ monoton steigend ist.

Da für die Konvergenz einer Reihe endlich viele Summanden weggelassen oder verändert werden können, ohne dass sich dabei das Konvergenzverhalten ändert, untersuchen wir die Reihe mit einem Anfangswert des Laufindizes von $3$. 

Wir untersuchen also die Reihe $\sum _{n=3}^{\infty }\frac{\ln( n)}{n}$ mit dem Integralkriterium.

Wir betrachten also das Integral $\int \frac{\ln( n)}{n} \ dn$ und substituieren mit $u=\ln( n)$ um die Stammfunktion zu bestimmen. Aus der Substitution $\frac{du}{dn} =\frac{1}{n} \Leftrightarrow dn=n\cdotp du$ und der Rücksubstitution ergibt sich:
\begin{equation*}
    \int \frac{\ln( n)}{n} \ dn=\int \frac{u}{n} \cdotp n\ \mathrm{d} u\Bigl|_{u=\ln( n)} =\left[\frac{1}{2} u^{2}\right] =\left[\frac{1}{2}\ln( n)^{2}\right]
\end{equation*}

Mit der Stammfunktion bestimmen wir nun das uneigentliche Integral im Intervall $[ 3;\infty )$ um eine Aussage über die Konvergenz der Reihe treffen zu können.
\begin{equation*}
    \begin{array}{ c l }
    \int\limits _{3}^{\infty }\frac{\ln( n)}{n} \ dn & =\lim\limits _{a\rightarrow \infty }\int\limits _{3}^{a}\frac{\ln( n)}{n} \ dn\\
    & =\lim\limits _{a\rightarrow \infty }\left[\frac{1}{2}\ln( n)^{2}\right]_{3}^{a}\\
    & =\lim\limits _{a\rightarrow \infty }\frac{1}{2}\left(\ln( a)^{2} -\ln( 3)^{2}\right)\\
    & =\frac{1}{2}\lim\limits _{a\rightarrow \infty }\left(\underbrace{\ln( a)^{2}}_{\rightarrow \infty }\right) -\frac{\ln( 3)^{2}}{2}\\
    & =\infty 
    \end{array}
\end{equation*}

Da das Integral der Funktion $\int _{3}^{\infty } f( n)$ im Intervall $[ 3;\infty )$ divergiert, divergiert auch die Reihe $\sum _{n=3}^{\infty } f( n)$.

Da $\sum _{n=1}^{\infty } f( n) =\frac{\ln( 2)}{2} +\sum _{n=3}^{\infty } f( n)$ ist auch das Konvergenzverhalten der Reihe $\sum _{n=3}^{\infty } f( n)$ dasselbe wie das der Reihe $\sum _{n=1}^{\infty } f( n)$.

Daraus folgt, dass die Reihe über das Intervall $[ 1;\infty )$ divergiert.

\end{document}
