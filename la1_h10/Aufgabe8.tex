\documentclass[main.tex]{subfiles}

\begin{document}

\section{Aufgabe 8}
Gegeben sei die Basis $B=\left\{1,\ x^{2} ,\ x^{4}\right\}$ des Vektorraums V der achsensymmetrischen Polynome maximal 4. Grades. 

\begin{enumerate}
    \item Zeigen Sie, dass die Lineare Hülle von $B$ einen Untervektorraum vom Vektorraum $P_{4}$ der Polynome vom Grad $4$ bildet.
    \item Untersuchen Sie die folgenden Polynome aus $V$ auf lineare Unabhängigkeit:
    \begin{gather*}
        3x^{4} -7x^{2} +2\\
        -x^{4} +2x^{2} -1\\
        4x^{4} +3x^{2} +2
    \end{gather*}    
\end{enumerate}

% Korrektur 2022-12-20 umgesetzt
\subsection{Lösung 8a}

Für einen Untervektorraum $U \subseteq V$ muss nach Definition 3.27 gelten:
\begin{gather*}
	U \neq \emptyset \\
	\forall x,y \in U : x \oplus y \in U \\
	\forall x \in U, \forall \lambda \in K : \lambda \odot x \in U
\end{gather*}

Die Elemente des Untervektorraums sei die Lineare Hülle von $B$, also 
	$$
		\mathcal{L}(B) = \lambda_1 + \lambda_2 \cdotp x^{2} + \lambda_3 \cdotp x^{4} \text{ mit } \lambda_{1,2,3} \in K \text{.}
	$$
	
Außerdem betrachten wir den Vektorraum der Polynome vom Grad 4 allgemein als
	$$
		P_4 = \left\{ p \in P \middle| \forall a_k \in K : p = \sum_{k=0}^{4} a_k \cdotp x^k \right\}
	$$

\textit{Teilmengenbeziehung:}

\begin{equation*}
	\mathcal{L}(B) \subseteq P_4 \Leftrightarrow x \in \mathcal{L}(B) \Rightarrow x \in P_4
\end{equation*}

Setze für $a_0 = \lambda_1, a_1 = 0, a_2 = \lambda_2, a_3 = 0, a_4 = \lambda_3$, dann gilt $\mathcal{L}(B) = P_4$. Da $0 \in K$ und $\lambda_{1,2,3} \in K$ folgt daraus die bezeichnete Teilmengenrelation.\\

\textit{Abgeschlossenheit bzgl. der Addition:}

Seien $p_1, p_2 \in \mathcal{L}(B)$ beliebig, so muss auch $(p_1 + p_2) \in \mathcal{L}(B)$ sein.
\begin{equation*}
	\begin{array}{rl}
		p_1 + p_2 & = \lambda_1 + \lambda_2 \cdotp x^{2} + \lambda_3 \cdotp x^{4} + \mu_1 + \mu_2 \cdotp x^{2} + \mu_3 \cdotp x^{4} \\
				  & = \underbrace{(\lambda_1 +  \mu_1)}_{\in K} + \underbrace{(\lambda_2 + \mu_2)}_{\in K} \cdotp x^{2} + \underbrace{(\lambda_3 + \mu_3)}_{\in K} \cdotp x^{4} \\
		\Rightarrow & (p_1 + p_2) \in \mathcal{L}(B) \ \checkmark
	\end{array}
\end{equation*}

\textit{Abgeschlossenheit bzgl. der Multiplikation:}

Sei $p_1 \in \mathcal{L}(B)$ beliebig und $r \in K$, so muss auch $(r \cdotp p_1) \in \mathcal{L}(B)$ sein.
\begin{equation*}
	\begin{array}{rl}
		r \cdotp p_1 & = r \cdotp (\lambda_1 + \lambda_2 \cdotp x^{2} + \lambda_3 \cdotp x^{4}) \\
						& = \underbrace{r \cdotp \lambda_1}_{\in K} + 
							\underbrace{r \cdotp \lambda_2}_{\in K} \cdotp x^{2} + 
							\underbrace{r \cdotp \lambda_3}_{\in K} \cdotp x^{4} \\
		\Rightarrow & (r \cdotp p_1) \in \mathcal{L}(B) \ \checkmark
	\end{array}
\end{equation*}

\subsection{Lösung 8b}

Die Polynome sind linear \textbf{unabhängig}, wenn die Determinante der Koeffizientenmatrix \textbf{ungleich} $0$ ist.
\begin{equation*}
    \det\begin{pmatrix}[1]
    3 & -7 & 2\\
    -1 & 2 & -1\\
    4 & 3 & 2
    \end{pmatrix} = 12+28-6-14+9-16 = 13 \neq 0
\end{equation*}
Daraus folgt, dass die Polynome linear unabhängig sind.

\end{document}
