\documentclass[main.tex]{subfiles}

\begin{document}

\section{Aufgabe 8}
Gegeben sei die Basis $B=\left\{1,\ x^{2} ,\ x^{4}\right\}$ des Vektorraums V der achsensymmetrischen Polynome maximal 4. Grades. 

\begin{enumerate}
    \item Zeigen Sie, dass die Lineare Hülle von $B$ einen Untervektorraum vom Vektorraum $P_{4}$ der Polynome vom Grad $4$ bildet.
    \item Untersuchen Sie die folgenden Polynome aus $V$ auf lineare Unabhängigkeit:
    \begin{gather*}
        3x^{4} -7x^{2} +2\\
        -x^{4} +2x^{2} -1\\
        4x^{4} +3x^{2} +2
    \end{gather*}    
\end{enumerate}

\subsection{Lösung 8a}

\begin{gather*}
    L( B) =\lambda +\mu \cdotp x^{2} +\rho \cdotp x^{4} \ \text{mit} \ \lambda ,\mu ,\rho \in K\\
    \\
    L( P_{4}) =\lambda _{1} +\lambda _{2} x+\lambda _{3} x^{2} +\lambda _{4} x^{3} +\lambda _{5} x^{4} \ \text{mit} \ \lambda _{i} \in K,\ i\in [ 1;5]\\
    \\
    L( B) \ =\{L( P_{4}) | \lambda _{2} =0\land \lambda _{4} =0\} \ \Rightarrow \ L( B) \subseteq L( P_{4}) \;\; \checked 
\end{gather*}    
        
\subsection{Lösung 8b}

Die Polynome sind linear unabhängig, wenn die Determinante der Koeffizientenmatrix \textbf{ungleich} $0$ ist.
\begin{equation*}
    \det\begin{pmatrix}[1]
    3 & -7 & 2\\
    -1 & 2 & -1\\
    4 & 3 & 2
    \end{pmatrix} = 12+28-6-14+9-16 = 49-36 = 13
\end{equation*}
Daraus folgt, dass die Polynome linear unabhängig sind.

\end{document}
