\documentclass[main.tex]{subfiles}

\begin{document}

\section{Aufgabe 8}
Gegeben sind die zwei Punkte $ P=( 1|2|3)$ und $ Q=( -1|1|2)$ und die Vektoren
\begin{equation*}
    a=\begin{pmatrix}
    2\\
    1\\
    3
    \end{pmatrix} \ \text{und} \ b=\begin{pmatrix}
    2\\
    0\\
    4
    \end{pmatrix}\text{.}
\end{equation*}

\begin{enumerate}
    \item[(a)] Bestimmen Sie die Gleichungen der beiden Geraden $g_{1}$ bzw. $g_{2}$ durch den Punkt $P$ in Richtung von $a$ bzw. durch $Q$ in Richtung von $b$.
    \item[(b)] Sind die Geraden windschief (d.h. sind sie weder parallel noch haben sie einen Schnittpunkt)?
    \item[(c)] Falls das der Fall ist, bestimmen Sie einen Vektor, der senkrecht auf beiden Geraden steht.
\end{enumerate}

\subsection{Lösung 8}

\end{document}
