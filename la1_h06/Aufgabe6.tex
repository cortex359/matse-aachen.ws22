\documentclass[main.tex]{subfiles}

\begin{document}

\section{Aufgabe 6}
Untersuchen Sie, ob die folgenden Ebenen einen eindeutigen Schnittpunkt im $\displaystyle \mathbb{R}^{3}$ besitzen:

a)


\begin{equation*}
\begin{array}{ l l }
E_{1} :x= & \begin{pmatrix}
1\\
0\\
0
\end{pmatrix} +\lambda \begin{pmatrix}
2\\
-2\\
0
\end{pmatrix} +\mu \begin{pmatrix}
-2\\
0\\
2
\end{pmatrix} ,\\
E_{2} :x= & \begin{pmatrix}
0\\
1\\
0
\end{pmatrix} +\lambda \begin{pmatrix}
3\\
0\\
1
\end{pmatrix} +\mu \begin{pmatrix}
0\\
3\\
5
\end{pmatrix} ,\\
E_{3} :x= & \begin{pmatrix}
0\\
0\\
2
\end{pmatrix} +\lambda \begin{pmatrix}
1\\
0\\
1
\end{pmatrix} +\mu \begin{pmatrix}
0\\
1\\
3
\end{pmatrix}
\end{array}
\end{equation*}
b)
\begin{equation*}
\begin{array}{ l l }
E_{1} :x= & \begin{pmatrix}
1\\
1\\
0
\end{pmatrix} +\lambda \begin{pmatrix}
2\\
-1\\
0
\end{pmatrix} +\mu \begin{pmatrix}
-3\\
0\\
1
\end{pmatrix} ,\\
E_{2} :x= & \begin{pmatrix}
0\\
1\\
0
\end{pmatrix} +\lambda \begin{pmatrix}
0\\
1\\
0
\end{pmatrix} +\mu \begin{pmatrix}
-1\\
0\\
3
\end{pmatrix} ,\\
E_{3} :x= & \begin{pmatrix}
2\\
0\\
1
\end{pmatrix} +\lambda \begin{pmatrix}
1\\
-4\\
0
\end{pmatrix} +\mu \begin{pmatrix}
2\\
0\\
-4
\end{pmatrix}
\end{array}
\end{equation*}

\subsection{Lösung 6}
Zunächst werden die Normalenvektoren der gegebenen Ebenen berechnet.



Ist das LGS der drei Ebenen eindeutig lösbar, so existiert ein eindeutiger Schnittpunkt der Ebenen.

Dies ist der Fall, wenn die Ebenen $E_{1}$ und $E_{2}$ nicht parallel liegen ($\overrightarrow{n_{E_{1}}} \times \overrightarrow{n_{E2}} \neq 0)$ und deren Schnittgerade nicht parallel zu $E_{3}$ ist.
\begin{equation*}
\left< \left(\overrightarrow{n_{E_{1}}} \times \overrightarrow{n_{E_{2}}}\right) ,\overrightarrow{n_{E_{3}}}\right> \neq 0
\end{equation*}
Da dies auch die Definition der Determinante ist, gilt ebenso:
\begin{equation*}
\det\left(\overrightarrow{n_{E_{1}}} ,\overrightarrow{n_{E_{2}}} ,\overrightarrow{n_{E_{3}}}\right) \neq 0
\end{equation*}

\subsubsection{Lösung 6a}
\begin{gather*}
\begin{array}{ r l }
\overrightarrow{n_{E_{1}}} = & \begin{pmatrix}
2\\
-2\\
0
\end{pmatrix} \times \begin{pmatrix}
-2\\
0\\
2
\end{pmatrix} =\begin{pmatrix}
-4\\
-4\\
-4
\end{pmatrix}\\
    & \\
\overrightarrow{n_{E_{2}}} = & \begin{pmatrix}
3\\
0\\
1
\end{pmatrix} \times \begin{pmatrix}
0\\
3\\
5
\end{pmatrix} =\begin{pmatrix}
-3\\
-15\\
9
\end{pmatrix}\\
    & \\
\overrightarrow{n_{E_{3}}} = & \begin{pmatrix}
1\\
0\\
1
\end{pmatrix} \times \begin{pmatrix}
0\\
1\\
3
\end{pmatrix} =\begin{pmatrix}
-1\\
-3\\
1
\end{pmatrix}
\end{array}\\
\\
\begin{array}{ r l }
\det\left(\overrightarrow{n_{E_{1}}} ,\overrightarrow{n_{E_{2}}} ,\overrightarrow{n_{E_{3}}}\right) = & \det\begin{pmatrix}
-4 & -3 & -1\\
-4 & -15 & -3\\
-4 & 9 & 1
\end{pmatrix}\\
= & ( -4)( -15) +( -3)( -3)( -4) +( -1)( -4) \cdotp 9\\
 & -( -4)( -3) \cdotp 9-( -3)( -4) -( -1)( -15)( -4)\\
= & 60-36+36-108-12+60\\
= & 0
\end{array}
\end{gather*}
$\implies$ Es existiert \textbf{kein} Schnittpunkt in $\mathbb{R}^{3}$.


\subsubsection{Lösung 6b}

\begin{gather*}
\begin{array}{ r l }
\overrightarrow{n_{E_{1}}} = & \begin{pmatrix}
2\\
-1\\
0
\end{pmatrix} \times \begin{pmatrix}
-3\\
0\\
1
\end{pmatrix} =\begin{pmatrix}
-1\\
-2\\
-3
\end{pmatrix}\\
    & \\
\overrightarrow{n_{E_{2}}} = & \begin{pmatrix}
0\\
1\\
0
\end{pmatrix} \times \begin{pmatrix}
-1\\
0\\
3
\end{pmatrix} =\begin{pmatrix}
3\\
0\\
1
\end{pmatrix}\\
    & \\
\overrightarrow{n_{E_{3}}} = & \begin{pmatrix}
1\\
-4\\
0
\end{pmatrix} \times \begin{pmatrix}
2\\
0\\
-4
\end{pmatrix} =\begin{pmatrix}
16\\
4\\
8
\end{pmatrix}
\end{array}\\
\\
\begin{array}{ r l }
\det\left(\overrightarrow{n_{E_{1}}} ,\overrightarrow{n_{E_{2}}} ,\overrightarrow{n_{E_{3}}}\right) = & \det\begin{pmatrix}
-1 & 3 & 16\\
-2 & 0 & 4\\
-3 & 1 & 8
\end{pmatrix}\\
= & 3\cdotp 4\cdotp ( -3) +( 16\cdotp ( -2) -( -1) \cdotp 4-3\cdotp ( -2) \cdotp 8\\
= & -36-32+4+48\\
= & -16
\end{array}
\end{gather*}
$\implies$ Es existiert \textbf{ein} eindeutiger Schnittpunkt in $\mathbb{R}^{3}$.

\end{document}
