\documentclass[main.tex]{subfiles}

\begin{document}

\section{Aufgabe 5}
Gegeben sind die folgenden Geraden in der Parmaeterdarstellung:
\begin{equation*}
    \begin{array}{ c l c l }
        g_{1} : & x=1+t & , & y=3-2\cdot t\\
        g_{2} : & x=\nicefrac{1}{2} -\nicefrac{3}{2} \cdotp t & , & y=1-4\cdotp t
    \end{array}
\end{equation*}
Geben Sie die jeweiligen Normalform und Hesse-Normalform an. Gibt es einen Schnittpunkt?

\subsection{Lösung 5}
Punkt-Richtungsform:
\begin{equation*}
    \begin{array}{ c l }
        g_{1} : & \vec{X} =\begin{pmatrix}
        1\\
        3
        \end{pmatrix} +t\cdotp \begin{pmatrix}
        1\\
        -2
        \end{pmatrix}\\
        g_{2} : & \vec{X} =\begin{pmatrix}
        \nicefrac{1}{2}\\
        1
        \end{pmatrix} +t\cdotp \begin{pmatrix}
        \nicefrac{3}{2}\\
        -4
        \end{pmatrix}
    \end{array}
\end{equation*}
Normalform mit dem Normalenvektor $\overrightarrow{n_{g1}} =\begin{pmatrix}
2\\
1
\end{pmatrix}$ da $\left< \begin{pmatrix}
1\\
-2
\end{pmatrix} ,\begin{pmatrix}
2\\
1
\end{pmatrix}\right> =0$:

\begin{equation*}
    g_{1} :0=\begin{pmatrix}
    2\\
    1
    \end{pmatrix} \cdotp \left(\vec{X} -\begin{pmatrix}
    1\\
    3
    \end{pmatrix}\right)
\end{equation*}

Hesse-Normalform mit $\displaystyle |\vec{n} |=\sqrt{5}$:
\begin{equation*}
    g_{1} :0=\begin{pmatrix}
    \nicefrac{2}{\sqrt{5}}\\
    \nicefrac{1}{\sqrt{5}}
    \end{pmatrix} \cdotp \left(\vec{X} -\begin{pmatrix}
    1\\
    3
    \end{pmatrix}\right)
\end{equation*}


Normalform mit dem Normalenvektor $\overrightarrow{n_{g2}} =\begin{pmatrix}
4\\
\nicefrac{3}{2}
\end{pmatrix}$ da $\left< \begin{pmatrix}
\nicefrac{3}{2}\\
-4
\end{pmatrix} ,\begin{pmatrix}
4\\
\nicefrac{3}{2}
\end{pmatrix}\right> =0$:

\begin{equation*}
    g_{2} :0=\begin{pmatrix}
    4\\
    \nicefrac{3}{2}
    \end{pmatrix} \cdotp \left(\vec{X} -\begin{pmatrix}
    \nicefrac{1}{2}\\
    1
    \end{pmatrix}\right)
\end{equation*}

Hesse-Normalform mit $|\vec{n} |=\sqrt{\nicefrac{73}{4}}$:
\begin{equation*}
    g_{2} :0=\frac{\begin{pmatrix}
    4\\
    \nicefrac{3}{2}
    \end{pmatrix}}{\sqrt{\nicefrac{73}{4}}} \cdotp \left(\vec{X} -\begin{pmatrix}
    \nicefrac{1}{2}\\
    1
    \end{pmatrix}\right)
\end{equation*}

Die beiden Geraden schneiden sich in $\displaystyle \mathbb{R}^{2}$, wenn sie nicht parallel zueinander sind.

Beweis duch Widerspruch: Wir nehmen an $\displaystyle g_{1} \parallel g_{2}$, dann $\displaystyle \exists \ t\in \mathbb{R}$ für das gilt

\begin{gather*}
    \begin{array}{ c c c c }
        & t\cdotp \begin{pmatrix}
        1\\
        -2
        \end{pmatrix} & = & \begin{pmatrix}
        \nicefrac{3}{2}\\
        -4
        \end{pmatrix}\\
        \text{mit} \ t=2: & 2\cdotp \begin{pmatrix}
        1\\
        -2
        \end{pmatrix} & = & \begin{pmatrix}
        \nicefrac{3}{2}\\
        -4
        \end{pmatrix}\\
        \Leftrightarrow  & \begin{pmatrix}
        1\\
        -4
        \end{pmatrix} & = & \begin{pmatrix}
        \nicefrac{3}{2}\\
        -4
        \end{pmatrix}
    \end{array}\\
    1=\nicefrac{3}{2} \ \lightning
\end{gather*}

Da die Geraden nicht parallel sind, muss es einen Schnittpunkt geben.

\end{document}
