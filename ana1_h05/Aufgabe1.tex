\section{Aufgabe 1}

Bestimmen Sie die Potenzreihe und den Konvergenzradius der Funktion $\displaystyle g( x) =\frac{4}{2x-5}$ mit $\displaystyle x\neq \frac{5}{2}$ um die Entwicklungspunkte
\begin{align*}
	\begin{array}{ l c }
		\text{a)} & x_{0} =0
	\end{array} & & \begin{array}{ l c }
		\text{b)} & x_{0} =1
	\end{array} & & \begin{array}{ l c }
		\text{c)} & x_{0} =-1
	\end{array}
\end{align*}

\subsection{Lösung 1a}
\begin{equation*}
	g( x) =\frac{4}{2x-5}
\end{equation*}
Es soll die Funktion $\displaystyle g( x)$ in eine Potenzreihe der Form 

\begin{equation*}
	\frac{1}{1-a( x-x_{0})} =\sum _{n=0}^{\infty }( a( x-x_{0}))^{n}
\end{equation*}entwickelt werden.

Dazu wird $\displaystyle g( x)$ wie folgt umgeformt
\begin{equation*}
	\begin{array}{ r l }
		g( x) = & \frac{4}{2x-5}\\
		= & \frac{-4}{5-2x}\\
		= & ( -4) \cdotp \frac{1}{5-2x}\\
		= & ( -4) \cdotp \frac{1}{5\cdotp \left( 1-\frac{2}{5} x\right)}\\
		= & \frac{-4}{5} \cdotp \frac{1}{1-\frac{2}{5} x}\\
		= & \frac{-4}{5} \cdotp \frac{1}{1-\frac{2}{5}( x-0)}
	\end{array}
\end{equation*}
Die Funktion wird nun als Potenzreihe in der zuvor genannten Form mit $\displaystyle a=\frac{2}{5}$ für den Bereich $\displaystyle \left| \frac{2}{5} \cdotp x\right| < 1$ geschrieben. 
\begin{equation*}
	\left( -\frac{4}{5}\right) \cdotp \sum _{n=0}^{\infty }\left(\frac{2}{5} x\right)^{n} =\left( -\frac{4}{5}\right) \cdotp \sum _{n=0}^{\infty }\left(\left(\frac{2}{5}\right)^{n} x^{n}\right)
\end{equation*}


Für den Konvergenzradius gilt 
\begin{equation*}
	\begin{array}{ l r c l }
		& 1 &  > & | a( x-x_{0})| \\
		\Leftrightarrow  & \ 1 &  > & \left| \frac{2}{5} \cdotp x\right| \\
		\Leftrightarrow  & 1 &  > & \frac{2}{5} \cdotp |x|\\
		\Leftrightarrow  & \frac{5}{2} &  > & |x|
	\end{array}
\end{equation*}


$\displaystyle \Rightarrow \ \left( -\frac{5}{2}\right) < x< \frac{5}{2}$ 



Die Funktion $\displaystyle g( x)$ lässt sich für den Konvergenzradius $\displaystyle a$ um den Entwicklungspunkt $\displaystyle x_{0} =0$ als Potenzreihe darstellen.



Von der Potenzreihe ist bekannt, dass sie konvergent in dem Bereich ist. Nun bleibt zu untersuchen, ob die Ränder selbst konvergent oder diverget sind.



Die Reihe
\begin{equation*}
	\left( -\frac{4}{5}\right) \cdotp \sum _{n=0}^{\infty }\left(\frac{2}{5} x\right)^{n}
\end{equation*}
ist für $\displaystyle x= -\frac{5}{2}$:
\begin{equation*}
	\left( -\frac{4}{5}\right) \cdotp \sum _{n=0}^{\infty }( -1)^{n}
\end{equation*}
divergent, da die Folge $\displaystyle \left(( -1)^{n}\right)_{n\in \mathbb{N}}$ divergent ist.



Für $\displaystyle x=\frac{5}{2}$ muss die Divergenz nicht untersucht werden, da bereits gegeben war, dass $\displaystyle \frac{5}{2} \notin X$.

Daher ist die Potenzreihe konvergent in dem Intervall $\displaystyle x\in ] -\frac{5}{2} ;\frac{5}{2}[$. 



\subsection{Lösung 1b}

Für die Betrachtung um den Entwicklungspunkt $\displaystyle x_{0} =1$
\begin{equation*}
	\begin{array}{ r l }
		g( x) = & \frac{4}{2x-5}\\
		= & 4\cdotp \frac{1}{2\cdotp ( x-1) -5+2}\\
		= & 4\cdotp \frac{1}{2\cdotp ( x-1) -3}\\
		= & 4\cdotp \frac{1}{-3+2\cdotp ( x-1)}\\
		= & 4\cdotp \frac{1}{-3\left( 1+\frac{2}{-3} \cdotp ( x-1)\right)}\\
		= & \frac{4}{-3} \cdotp \frac{1}{1-\frac{2}{3} \cdotp ( x-1)}
	\end{array}
\end{equation*}
Die Funktion wird nun als Potenzreihe in der zuvor genannten Form mit $\displaystyle a=\frac{2}{3}$ für den Bereich $\displaystyle \left| \frac{2}{3} \cdotp ( x-x_{0})\right| < 1$ geschrieben.
\begin{equation*}
	\left( -\frac{4}{3}\right) \cdotp \sum _{n=0}^{\infty }\left(\frac{2}{3}( x-1)\right)^{n}
\end{equation*}
Für den Konvergenzradius gilt 
\begin{equation*}
	\begin{array}{ l r c l }
		& 1 &  > & | a( x-x_{0})| \\
		\Leftrightarrow  & \ 1 &  > & \left| \frac{2}{3} \cdotp ( x-1)\right| \\
		\Leftrightarrow  & 1 &  > & \left| \frac{2}{3} x-\frac{2}{3}\right| \\
		\Leftrightarrow  & 1 &  > & \left| \frac{2x-2}{3}\right| 
	\end{array}
\end{equation*}
$\displaystyle \Rightarrow $ Für $\displaystyle x >1$:
\begin{equation*}
	\begin{array}{ r c l }
		1 &  > & \frac{2x-2}{3}\\
		3 &  > & 2x-2\\
		5 &  > & 2x\\
		\frac{5}{2} &  > & x
	\end{array}
\end{equation*}
$\displaystyle \Rightarrow $ Für $\displaystyle x< 1$:
\begin{equation*}
	\begin{array}{ r c l }
		1 &  > & -\frac{2x-2}{3}\\
		3 &  > & -2x+2\\
		1 &  > & -2x\\
		-\frac{1}{2} & <  & x
	\end{array}
\end{equation*}


Die Funktion $\displaystyle g( x)$ lässt sich für den Konvergenzradius von
\begin{equation*}
	-\frac{1}{2} < x< \frac{5}{2}
\end{equation*}
um den Entwicklungspunkt $\displaystyle x_{0} =1$ als Potenzreihe darstellen.

Da der obere Rand nicht Teil der Definitionsmenge ist, bleibt die Konvergenz am unteren Rand zu untersuchen. 

Die Reihe
\begin{equation*}
	\left( -\frac{4}{3}\right) \cdotp \sum _{n=0}^{\infty }\left(\frac{2}{3}( x-1)\right)^{n}
\end{equation*}
ist für $\displaystyle x=-\frac{1}{2}$:
\begin{equation*}
	\left( -\frac{4}{3}\right) \cdotp \sum _{n=0}^{\infty }\left(\frac{2}{3} \cdotp \left( -\frac{1}{2} -1\right)\right)^{n} =\left( -\frac{4}{3}\right) \cdotp \sum _{n=0}^{\infty }\left(\frac{2}{3} \cdotp \left( -\frac{3}{2}\right)\right)^{n} =\left( -\frac{4}{3}\right) \cdotp \sum _{n=0}^{\infty }( -1)^{n}
\end{equation*}
divergent, da die Folge $\displaystyle \left(( -1)^{n}\right)_{n\in \mathbb{N}}$ divergent ist.\\

\subsection{Lösung 1c}
[Fehlt]