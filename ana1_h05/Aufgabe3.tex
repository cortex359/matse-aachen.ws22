\section{Aufgabe 3}

Vereinfachen Sie die folgenden Terme mithilfe der Additionstheoreme.
\begin{align*}
	\begin{array}{ l c }
		\text{a)} & \frac{\sin( 2x)}{\sin( x)}
	\end{array} & & \begin{array}{ l c }
		\text{b)} & \sin^{4}( x) -\cos^{4}( x)
	\end{array}
\end{align*}


\subsection{Lösung 3a}

Mit dem Additionstheorem (\textit{Lemma 108, Seite 131)}
\begin{equation*}
	\sin( u+v) =\sin( u) \cdotp \cos( v) +\sin( v) \cdotp \cos( u)
\end{equation*}
gilt mit $\displaystyle \sin( 2x) =\sin( x+x)$ für
\begin{equation*}
	\begin{array}{ r l }
		\sin( x+x) = & \sin( x) \cdotp \cos( x) +\sin( x) \cdotp \cos( x)\\
		= & \sin( x) \cdotp (\cos( x) +\cos( x))
	\end{array}
\end{equation*}
sodass
\begin{equation*}
	\begin{array}{ r l }
		\frac{\sin( 2x)}{\sin( x)} = & \frac{\sin( x) \cdotp (\cos( x) +\cos( x))}{\sin( x)}\\
		= & \cos( x) +\cos( x)\\
		= & 2\cdotp \cos( x)
	\end{array}
\end{equation*}
vereinfacht werden kann. 



\subsection{Lösung 3b}
Aus dem Additionstheorem (\textit{Lemma 108, Seite 131)}\begin{equation*}
	1=\cos^{2}( x) +\sin^{2}( x)
\end{equation*}

folgt mit der 3. Binomischen Formel:
\begin{equation*}
	\begin{array}{ c l }
		\sin^{4}( x) -\cos^{4}( x) & =\left(\sin^{2}( x) +\cos^{2}( x)\right) \cdotp \left(\sin^{2}( x) -\cos^{2}( x)\right)\\
		& =\sin^{2}( x) -\cos^{2}( x)\\
		& =1-\cos^{2}( x) -\cos^{2}( x)\\
		& =1-2\cos^{2}( x)\\
		& =-\cos( 2x)
	\end{array}
\end{equation*}
