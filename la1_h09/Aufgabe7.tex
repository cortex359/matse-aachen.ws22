\documentclass[main.tex]{subfiles}

\begin{document}

\section{Aufgabe 7}

$x_{1} ,\ \dotsc ,\ x_{n}$ seien linear unabhängige Vektoren aus einem $K$-Vektorraum $V$. Weiter sei $x=\sum _{i=1}^{n} \mu _{i} x_{i}$ und $\mu _{i} \in K$ für alle $i=1,\ \dotsc ,\ n$. Es ist zu zeigen, dass unter der Voraussetzung $\sum _{i=1}^{n} \mu _{i} \ \neq 1$ die Vektoren $x-x_{1} ,\ \dotsc x-x_{n}$ linear unabhängig sind.

\subsection{Lösung 7a}

\begin{equation*}
    \begin{array}{ c l }
    0 & =\sum\limits _{i=1}^{n} \lambda _{i}( x-x_{i})\\
    & =\sum\limits _{i=1}^{n}( \lambda _{i} x) -\sum\limits _{i=1}^{n}( \lambda _{i} x_{i})\\
    & =x\cdotp \sum\limits _{i=1}^{n}( \lambda _{i}) -\sum\limits _{i=1}^{n}( \lambda _{i} x_{i})\\
    & =\sum\limits _{i=1}^{n}( \mu _{i} x_{i}) \cdotp \sum\limits _{i=1}^{n}( \lambda _{i}) -\sum\limits _{i=1}^{n}( \lambda _{i} x_{i})\\
    & =\sum\limits _{i=1}^{n}\left(\sum\limits _{j=1}^{n}( \lambda _{j}) \cdotp \mu _{i} x_{i}\right) -\sum\limits _{i=1}^{n}( \lambda _{i} x_{i})\\
    & =\sum\limits _{i=1}^{n}\left(\left(\left(\sum\limits _{j=1}^{n}( \lambda _{j}) \cdotp \mu _{i}\right) -\sum\limits _{i=1}^{n}( \lambda _{i})\right) x_{i}\right)
    \end{array}
\end{equation*}
Aus der linearen Unabhängigkeit folgt:
\begin{equation*}
    \begin{array}{ c r l }
    & 0 & =\sum\limits _{i=1}^{n}\left(\sum\limits _{j=1}^{n}( \lambda _{j}) \cdotp \mu _{i}\right) -\sum\limits _{i=1}^{n}( \lambda _{i})\\
    \Leftrightarrow  & \sum\limits _{i=1}^{n} \lambda _{i} & =\sum\limits _{i=1}^{n}\left(\sum\limits _{j=1}^{n}( \lambda _{j}) \cdotp \mu _{i}\right)\\
    \Rightarrow  & \forall i\in [ 1;n] : & \lambda _{i} =\sum\limits _{j=1}^{n}( \lambda _{j}) \cdotp \mu _{i}
    \end{array}
\end{equation*}
Außerdem gilt
\begin{equation*}
    \begin{array}{ c c l }
    & \sum\limits _{i=1}^{n} \lambda _{i} & =\sum\limits _{i=1}^{n}\left( \mu _{i} \cdotp \sum\limits _{j=1}^{n} \lambda _{j}\right)\\
    \Leftrightarrow  & 0 & =\sum\limits _{i=1}^{n}\left( \mu _{i} \cdotp \sum\limits _{j=1}^{n} \lambda _{j}\right) -\sum\limits _{i=1}^{n} \lambda _{i}\\
    \Leftrightarrow  & 0 & =\sum\limits _{i=1}^{n} \lambda _{i} \cdotp \left(\sum\limits _{i=1}^{n}( \mu _{i}) -1\right)
    \end{array}
\end{equation*}

Da $\sum _{i=1}^{n} \mu _{i} \ \neq 1$ ist, muss $\sum\nolimits _{i=1}^{n} \lambda _{i} =0$ gelten.

\subsection{Lösung 7b}

Aus $\sum\nolimits _{i=1}^{n} \lambda _{i} =0$ und $\sum\nolimits _{i=1}^{n} \lambda _{i}( x-x_{i}) =0$ folgt
\begin{equation*}
    \begin{array}{ c l }
    0 & =\sum\limits _{i=1}^{n} \lambda _{i}( x-x_{i})\\
    & =\sum\limits _{i=1}^{n} \lambda _{i} x-\sum\limits _{i=1}^{n} \lambda _{i} x_{i}\\
    & =x\cdotp \underbrace{\sum\limits _{i=1}^{n} \lambda _{i}}_{=0} -\sum\limits _{i=1}^{n} \lambda _{i} x_{i}\\
    & =-\sum\limits _{i=1}^{n} \lambda _{i} x_{i}\\
    & =\sum\limits _{i=1}^{n} \lambda _{i} x_{i} \ \checked 
    \end{array}
\end{equation*}

\end{document}
