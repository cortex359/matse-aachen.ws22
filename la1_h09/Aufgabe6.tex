\documentclass[main.tex]{subfiles}

\begin{document}

\arraycolsep=1pt\def\arraystretch{1} % streach array

\section{Aufgabe 6}

Untersuchen Sie, ob die folgenden Systeme von Vektoren des $\mathbb{R}^{n}$ linear abhängig oder unabhängig sind.

\begin{enumerate}
    \item $\begin{pmatrix}[1]
        4\\
        -1\\
        2
        \end{pmatrix} ,\begin{pmatrix}[1]
        -4\\
        10\\
        2
        \end{pmatrix}$
    \item  $\begin{pmatrix}[1]
        -2\\
        0\\
        1
        \end{pmatrix} ,\ \begin{pmatrix}[1]
        3\\
        2\\
        5
        \end{pmatrix} ,\begin{pmatrix}[1]
        6\\
        -1\\
        1
        \end{pmatrix} ,\begin{pmatrix}[1]
        4\\
        0\\
        2
        \end{pmatrix}$
    \item $\begin{pmatrix}[1]
        0\\
        0\\
        2\\
        2
        \end{pmatrix} ,\begin{pmatrix}[1]
        3\\
        3\\
        0\\
        0
        \end{pmatrix} ,\begin{pmatrix}[1]
        1\\
        1\\
        0\\
        -1
        \end{pmatrix}$
\end{enumerate}

\subsection{Lösung 6a}

Wären die beiden Vektoren linear abhängig, dann würde gelten: $\exists \ \lambda $, sodass gilt:

\begin{equation*}
    \lambda \cdotp \begin{pmatrix}[1]
    4\\
    -1\\
    2
    \end{pmatrix} =\begin{pmatrix}[1]
    -4\\
    10\\
    2
    \end{pmatrix} \ \begin{array}{ r r l c }
    & \lambda  & =-1 & \\
    & \lambda  & =-10 & \lightning \\
    & \lambda  & =1 & \lightning 
    \end{array}
\end{equation*}
Da $\lambda \cdotp 4=-4\ \land \lambda \cdotp ( -1) =10\land \lambda \cdotp 2=2\ \lightning $, also kein $\lambda $ existiert, welches die Gleichung erfüllen könnte, müssen die Vektoren linear unabhängig sein.


\subsection{Lösung 6b}

Die vier gegebenen Vektoren aus dem $\mathbb{R}^{3}$ müssen zwangsläufig linear abhängig voneinander sein, da maximal $\dim\left(\mathbb{R}^{3}\right) =3$ Vektoren linear unabhängig aus dem Vektorraum sein können. 

\subsection{Lösung 6c}

Wären die drei Vektoren linear abhängig, dann würde gelten: $\exists \ \lambda ,\mu $, sodas gilt:
\begin{equation*}
    \lambda \cdotp \begin{pmatrix}[1]
    0\\
    0\\
    2\\
    2
    \end{pmatrix} +\mu \begin{pmatrix}[1]
    3\\
    3\\
    0\\
    0
    \end{pmatrix} =\begin{pmatrix}[1]
    1\\
    1\\
    0\\
    -1
    \end{pmatrix} \ \begin{array}[1]{ r r l c }
    & \mu  & =1/3 & \\
    & \mu  & =1/3 & \\
    & \lambda  & =0 & \\
    & \lambda  & =-1/2 & \lightning 
    \end{array}
\end{equation*}

Da $\lambda \cdotp 2=0\land \lambda \cdotp 2=-1\ \lightning$ sind die Vektoren linear unabhängig. 

\end{document}
