\section{Aufgabe 7}

Gegeben sei die Gerade $g$ mit

\begin{equation*}
  g:x=\begin{pmatrix}
    1\\
    0
  \end{pmatrix} +\alpha \begin{pmatrix}
    -3\\
    2
  \end{pmatrix} .
\end{equation*}Finden Sie eine Gerade, die senkrecht zu der Geraden g ist und zusätzlich durch den Punkt $\displaystyle ( 4|3)$ geht.

\subsection{Lösung 7}

Durch Bemerkung 2.21 wissen wir, dass der Vektor $\displaystyle \begin{pmatrix}
  -3\\
  2
\end{pmatrix} \perp \begin{pmatrix}
  -2\\
  -3
\end{pmatrix}$ liegt. Damit ist der Richtungsvektor der gesuchten Geraden gegeben.\\

Verwendet man nun den Punkt $\displaystyle (4|3)$ als Aufpunkt, so erhält man die Gerade $h$, welche notwendigerweise senkrecht zu $g$ ist und durch den Punkt $(4|3)$ geht:
\begin{equation*}
  h:x=\begin{pmatrix}
    4\\
    3
  \end{pmatrix} +\beta \begin{pmatrix}
    2\\
    3
  \end{pmatrix}
\end{equation*}