\renewcommand{\arraystretch}{1.1}

\section{Aufgabe 8}

Bestimmen Sie die Lösungen $\displaystyle x$ der Gleichung $\displaystyle a\times x=b$ für $\displaystyle a=( 2,\ -1,\ 3)^{T}$ und

(a)
\begin{equation*}
  b=\begin{pmatrix}
    1\\
    -4\\
    -2
  \end{pmatrix}
\end{equation*}
(b)
\begin{equation*}
  b=\begin{pmatrix}
    1\\
    -4\\
    -1
  \end{pmatrix}
\end{equation*}
falls es überhaupt Lösungen gibt.

\subsection{Lösung 8a}

\begin{gather*}
  a\times x=b\\
  \\
  \begin{pmatrix}
    2\\
    -1\\
    3
  \end{pmatrix} \times \begin{pmatrix}
    x_{1}\\
    x_{2}\\
    x_{3}
  \end{pmatrix} =\begin{pmatrix}
    1\\
    -4\\
    -2
  \end{pmatrix}\\
  \\
  \Leftrightarrow \ \begin{pmatrix}
    -x_{3} -3x_{2}\\
    3x_{1} -2x_{3}\\
    2x_{2} +x_{1}
  \end{pmatrix} =\begin{pmatrix}
    1\\
    -4\\
    -2
  \end{pmatrix}
\end{gather*}

Dies kann man als Lineares Gleichungssystem schreiben:

\begin{equation*}
  \begin{array}{ c r l }
    & \left(\begin{array}{rrr|r}
      0 & -3 & -1 & 1\\
      3 & 0 & -2 & -4\\
      1 & 2 & 0 & -2
    \end{array}\right) & \begin{array}{ l }
      \rightarrow III\\
      \rightarrow II\\
      \rightarrow I
    \end{array}\vspace{2mm}\\
    \Leftrightarrow  & \left(\begin{array}{rrr|r}
      1 & 2 & 0 & -2\\
      0 & -3 & -1 & 1\\
      3 & 0 & -2 & -4
    \end{array}\right) & \begin{array}{ l }
      \\
      \cdotp ( -\nicefrac{1}{3})\\
      +( -3\cdotp I)
    \end{array}\vspace{2mm}\\
    \Leftrightarrow  & \left(\begin{array}{rrr|r}
      1 & 2 & 0 & -2\\
      0 & 1 & \nicefrac{1}{3} & \nicefrac{-1}{3}\\
      0 & -6 & -2 & 2
    \end{array}\right) & \begin{array}{ l }
      \\
      \\
      +( 6\cdotp II)
    \end{array}\vspace{2mm}\\
    \Leftrightarrow  & \left(\begin{array}{rrr|r}
      1 & 2 & 0 & -2\\
      0 & 1 & \nicefrac{1}{3} & \nicefrac{-1}{3}\\
      0 & 0 & 0 & 0
    \end{array}\right) & \begin{array}{ l }
      \\
      \\
      
    \end{array}
  \end{array}
\end{equation*}
Das LGS ist unterbestimmt, also existiert keine eindeutige Lösung.

\subsection{Lösung 8b}
\begin{gather*}
  a\times x=b\\
  \\
  \begin{pmatrix}
    2\\
    -1\\
    3
  \end{pmatrix} \times \begin{pmatrix}
    x_{1}\\
    x_{2}\\
    x_{3}
  \end{pmatrix} =\begin{pmatrix}
    1\\
    -4\\
    -1
  \end{pmatrix}\\
  \Leftrightarrow \ \begin{pmatrix}
    -x_{3} -3x_{2}\\
    3x_{1} -2x_{3}\\
    2x_{2} +x_{1}
  \end{pmatrix} =\begin{pmatrix}
    1\\
    -4\\
    -1
  \end{pmatrix}
\end{gather*}

Als LGS:

\begin{equation*}
  \begin{array}{ c r l }
    & \left(\begin{array}{rrr|r}
      0 & -3 & -1 & 1\\
      3 & 0 & -2 & -4\\
      1 & 2 & 0 & -1
    \end{array}\right) & \begin{array}{ l }
      \rightarrow III\\
      \rightarrow II\\
      \rightarrow I
    \end{array}\vspace{2mm}\\
    \Leftrightarrow  & \left(\begin{array}{rrr|r}
      1 & 2 & 0 & -1\\
      0 & -3 & -1 & 1\\
      3 & 0 & -2 & -4
    \end{array}\right) & \begin{array}{ l }
      \\
      \cdotp ( -\nicefrac{1}{3})\\
      +( -3\cdotp I)
    \end{array}\vspace{2mm}\\
    \Leftrightarrow  & \left(\begin{array}{rrr|r}
      1 & 2 & 0 & -1\\
      0 & 1 & \nicefrac{1}{3} & \nicefrac{-1}{3}\\
      0 & -6 & -2 & -1
    \end{array}\right) & \begin{array}{ l }
      \\
      \\
      +( 6\cdotp II)
    \end{array}\vspace{2mm}\\
    \Leftrightarrow  & \left(\begin{array}{rrr|r}
      1 & 2 & 0 & -2\\
      0 & 1 & \nicefrac{1}{3} & \nicefrac{-1}{3}\\
      0 & 0 & 0 & -3
    \end{array}\right) & \begin{array}{ l }
      \\
      \\
      \lightning 
    \end{array}
  \end{array}
\end{equation*}
\\
Durch den Widerspruch ist gezeigt, dass keine Lösung existiert. $\mathbb{L}=\emptyset$