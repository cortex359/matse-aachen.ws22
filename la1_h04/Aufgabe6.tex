\section{Aufgabe 6}

Bestimmen Sie eine Gleichung der Geraden durch die Punkte A und B. Untersuchen Sie jeweils, ob der Punkt C auf dieser Geraden liegt.

(a)
\begin{equation*}
  A=( -2|1) ,\ B=( 2|2) ,\ C=( -10|5)
\end{equation*}
(b)
\begin{equation*}
  A=( 1|2|3) ,\ B=( 3|1|2) ,\ C=( -9|7|8)
\end{equation*}

\subsection{Lösung 6a}

Eine Gerade $\displaystyle G$ kann durch eine Geradengleichung der Form $\displaystyle \forall x\in G:x=\vec{p} +\alpha \cdotp \vec{v}$ beschrieben werden, welche den Ortsvektor $\displaystyle \vec{p}$ zum Punkt $\displaystyle A$, sowie den Richtungsvektor $\displaystyle \vec{v} =\overrightarrow{OB} -\vec{p}$ zwischen dem Punkt B und A verwendet.
\begin{gather*}
  \begin{array}{ c r l }
    & x= & \begin{pmatrix}
      -2\\
      1
    \end{pmatrix} +\alpha \cdotp \left(\begin{pmatrix}
      2\\
      2
    \end{pmatrix} -\begin{pmatrix}
      -2\\
      1
    \end{pmatrix}\right)\\
    \Leftrightarrow  & x= & \begin{pmatrix}
      -2\\
      1
    \end{pmatrix} +\alpha \cdotp \begin{pmatrix}
      4\\
      1
    \end{pmatrix}\\
    \Leftrightarrow  & x= & \begin{pmatrix}
      4\alpha -2\\
      \alpha +1
    \end{pmatrix}
  \end{array}\\
  \\
  x_{1} =4\alpha -2\\
  x_{2} =\alpha +1\ 
\end{gather*}
Der Punkt C liegt auf der Gerade, wenn $\displaystyle \exists \alpha :-10=4\alpha -2\land 5=\alpha +1\ \lightning $
\\
$\Rightarrow$ Es existiert keine Gerade, auf der alle Punkte liegen.

\subsection{Lösung 6b}
Nach der gleichen Argumentation gilt in $\mathbb{R}^3$:
\begin{gather*}
  \begin{array}{ c r l }
    & x= & \begin{pmatrix}
      1\\
      2\\
      3
    \end{pmatrix} +\alpha \cdotp \left(\begin{pmatrix}
      3\\
      1\\
      2
    \end{pmatrix} -\begin{pmatrix}
      1\\
      2\\
      3
    \end{pmatrix}\right)\\
    \\
    \Leftrightarrow  & x= & \begin{pmatrix}
      1\\
      2\\
      3
    \end{pmatrix} +\alpha \cdotp \begin{pmatrix}
      2\\
      -1\\
      -1
    \end{pmatrix}\\
    \\
    \Leftrightarrow  & x= & \begin{pmatrix}
      1+2\alpha \\
      2-\alpha \\
      3-\alpha 
    \end{pmatrix}
  \end{array}\\
\end{gather*}

Wenn der Punkt $C$ auf der Geraden liegt, muss ein $\alpha$ existieren, für dass $x=c$:

\begin{gather*}
  x_{1} =1+2\alpha \\
  x_{2} =2-\alpha \\
  x_{3} =3-\alpha \\
  \\
  \begin{array}{ l r r l c }
    x_{1} : &  & -9= & 1+2\alpha  & \\
    & \Leftrightarrow  & \alpha = & -5 & \\
    x_{2} : &  & 7= & 2-( -5) & \checkmark \\
    x_{3} : &  & 8= & 3-( -5) & \checkmark 
  \end{array}
\end{gather*}

$\Rightarrow$ Der Punkt C liegt auf der Geraden.