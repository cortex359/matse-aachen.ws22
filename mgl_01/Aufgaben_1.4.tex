\documentclass[main.tex]{subfiles}

\begin{document}

\section{1.4 Aufgaben}

\subsection{Aufgabe 1}
Überprüfen Sie mithilfe von Wahrheitstafeln, ob die folgenden Aussageformen allgemeingültig sind.

\subsubsection{a)}
$$
    (A \lor B) \implies (( \lnot A \lor B) \land (A \implies \lnot A))
$$
\newcolumntype{d}{>{\columncolor{yellow!15}}c}
\newcolumntype{e}{>{\columncolor{yellow!30}}c}
\newcolumntype{f}{>{\columncolor{yellow!45}}c}
\newcolumntype{g}{>{\columncolor{yellow!60}}c}
\newcolumntype{h}{>{\columncolor{yellow!75}}c}
\begin{equation*}
\begin{array}{d|e|d|h|e|d|f|e|g|d|f|e|d}
    (A  & \lor &  B) & \implies & (( \lnot &  A & \lor & B) & \land & (A & \implies & \lnot & A))\\
    \hline
     w  & w  & w  & f  & f  & w  & w  & w  & f  & w  & f  & f  & w\\
     w  & w  & f  & f  & f  & w  & f  & f  & f  & w  & f  & f  & w\\
     f  & w  & w  & w  & w  & f  & w  & w  & w  & f  & w  & w  & f\\
     f  & f  & f  & w  & w  & f  & w  & f  & w  & f  & w  & w  & f\\
    \hline
     1. & 2. & 1. & 5. & 2. & 1. & 3. & 1. & 4. & 1. & 3. & 2. & 1.\\
\end{array}
\end{equation*}

Die Aussage ist nicht allgemeingültig.

\subsubsection{b)}
$$
    (A \implies (B \implies C)) \equiv ((A \land B) \implies C)
$$

\begin{equation*}
    \begin{array}{d|f|d|e|d|g|d|e|d|f|d}
        (A  & \implies  & (B    & \implies  & C)   & ) \equiv ( & (A   & \land & B)    & \implies  & C) \\
        \hline
         w  & w  & w  & w  & w  & w  & w  & w  & w  & w  &  w \\
         w  & f  & w  & f  & f  & w  & w  & w  & w  & f  &  f \\
         w  & w  & f  & w  & w  & w  & w  & f  & f  & w  &  w \\
         w  & w  & f  & w  & f  & w  & w  & f  & f  & w  &  f \\
         f  & w  & w  & w  & w  & w  & f  & f  & w  & w  &  w \\
         f  & w  & w  & f  & f  & w  & f  & f  & w  & w  &  f \\
         f  & w  & f  & w  & w  & w  & f  & f  & f  & w  &  w \\
         f  & w  & f  & w  & f  & w  & f  & f  & f  & w  &  f \\ \hline
         1. & 3. & 1. & 2. & 1. & 4. & 1. & 2. & 1. & 3. &  1.\\
    \end{array}
\end{equation*}

Die Aussage ist allgemeingültig.

\subsection{Aufgabe 2}
Vereinfachen Sie die folgenden Ausdrücke:

\subsubsection{a)}
\begin{equiveqs}[rrcl]
        & (A \implies B) & \lor & ((A \land B) \Leftarrow B) \\
\equiv  & \lnot A \lor B & \lor & \lnot B \lor (A \land B) \\
\equiv  & B & \lor & \lnot B \\
\equiv  & & T &  \\
\end{equiveqs}

\subsubsection{b)}
\begin{equiveqs}[rrcl]
        & (A \land B) & \Leftarrow & ((A \implies B) \land B) \\
\equiv  & ((A \implies B) \land B)         & \implies & (A \land B)  \\
\equiv  & ((\lnot A \lor B) \land B)       & \implies & (A \land B)  \\
\equiv  & \lnot ((\lnot A \lor B) \land B) & \lor     & (A \land B)  \\
\equiv  & \lnot ((\lnot A \land B) \lor (B \land B)) & \lor     & (A \land B)  \\
\equiv  & \lnot ((\lnot A \land B) \lor B) & \lor     & (A \land B)  \\
\equiv  & \lnot B & \lor     & (A \land B)  \\
\equiv  & \lnot B & \lor & A  \\
\end{equiveqs}

\subsection{Aufgabe 3}
Erstellen Sie für die folgende logische Aussageform $\alpha$ eine Wahrheitstafel:
$$
    \alpha = \lnot ( A \equiv  (B  \implies  C) ) \implies (\lnot  A  \lor  C)  \land  (B \land  \lnot  C)
$$

\begin{equation*}
\begin{array}{g|d|f|d|e|d|h|e|d|f|d|g|d|f|e|d}
    \lnot & ( A & \equiv & (B & \implies & C) &) \implies (&\lnot & A & \lor & C) & \land & (B &\land & \lnot & C) \\
    \hline
    f & w  & w  & w  & w  &  w  & w  & f  & w  & w  & w  & f  & w  & f  &  f  & w \\
    w & w  & f  & w  & f  &  f  & f  & f  & w  & f  & f  & f  & w  & w  &  w  & f \\
    f & w  & w  & f  & w  &  w  & w  & f  & w  & w  & w  & f  & f  & f  &  f  & w \\
    f & w  & w  & f  & w  &  f  & w  & f  & w  & f  & f  & f  & f  & f  &  w  & f \\
    w & f  & f  & w  & w  &  w  & f  & w  & f  & w  & w  & f  & w  & f  &  f  & w \\
    f & f  & w  & w  & f  &  f  & w  & w  & f  & w  & f  & w  & w  & w  &  w  & f \\
    w & f  & f  & f  & w  &  w  & f  & w  & f  & w  & w  & f  & f  & f  &  f  & w \\
    w & f  & f  & f  & w  &  f  & f  & w  & f  & w  & f  & f  & f  & f  &  w  & f \\ \hline
    4.& 1. & 3. & 1. & 2. &  1. & 5. & 2. & 1. & 3. & 1. & 4. & 1. & 3. &  2. & 1.\\
\end{array}
\end{equation*}
\\[4mm]
Wann ist die Aussage wahr?

\begin{equation*}
\begin{array}{lcl}
    \text{wahr} & & \text{falsch} \\
    \begin{array}{d|d|d|h}
        A & B & C & \alpha \\
        \hline
        w & w & w  & w \\
        w & f & w  & w \\
        w & f & f  & w \\
        f & w & f  & w \\
    \end{array}
    & \quad &
    \begin{array}{d|d|d|h}
        A & B & C & \alpha \\
        \hline
        w & w & f  & f \\
        f & w & w  & f \\
        f & f & w  & f \\
        f & f & f  & f \\
    \end{array}
\end{array}
\end{equation*}
\\[4mm]

Geben Sie eine disjunktive und eine konjunktive Normalform an.

\begin{equation*}
    \begin{aligned}
        \text{DNF von } \alpha: & (A \land B \land C) \lor (A \land \lnot B \land C) \lor (A \land \lnot B \land \lnot C) \lor (\lnot A \land B \land \lnot C)\\
    \end{aligned}
\end{equation*}

Die konjunktive Normalform ergibt sich über die DNF von $\lnot \alpha$, welche man zunächst aufstellen muss. 

\begin{equation*}
    \begin{aligned}
        \text{DNF von } \lnot \alpha: & (A \land B \land \lnot C) \lor (\lnot A \land B \land C) \lor (\lnot A \land \lnot B \land C) \lor (\lnot A \land \lnot B \land \lnot C)\\
    \end{aligned}
\end{equation*}

Danach lässt sich durch Invertierung und Anwendung von deMorgan die KNF konstruieren:

\begin{equiveqs}[lrcl]
    & \lnot \alpha & \equiv & (A \land B \land \lnot C) \lor (\lnot A \land B \land C) \lor (\lnot A \land \lnot B \land C) \lor (\lnot A \land \lnot B \land \lnot C)\\
\congruent & \lnot [\lnot \alpha] & \equiv & \lnot \left[ (A \land B \land \lnot C) \lor (\lnot A \land B \land C) \lor (\lnot A \land \lnot B \land C) \lor (\lnot A \land \lnot B \land \lnot C) \right]\\
\congruent & \alpha & \equiv & \lnot(A \land B \land \lnot C) \land \lnot(\lnot A \land B \land C) \land \lnot (\lnot A \land \lnot B \land C) \land \lnot (\lnot A \land \lnot B \land \lnot C)\\
\congruent & \alpha & \equiv & (\lnot A \lor \lnot B \lor C) \land (A \lor \lnot B \lor \lnot C) \land (A \lor B \lor \lnot C) \land (A \lor B \lor C)\\
\end{equiveqs}

\subsection*{Aufgabe 4}

Gegeben seien die folgenden Aussagen:
\begin{itemize}
\item[A:] Die Sonne scheint.
\item[B:] Ein Auftrag liegt vor.
\item[C:] Miss Peel übt Karate.
\item[D:] Miss Peel besucht Mr. Steed.
\item[E:] Mr. Steed spielt Golf.
\item[F:] Mr. Steed luncht mit Miss Peel.
\end{itemize}

Bestimmen Sie für folgende Aussagen die zugehörigen aussagenlogischen Formeln:
\begin{enumerate}
    \item Wenn die Sonne scheint, dann spielt Mr. Steed Golf.
    \item Wenn die Sonne nicht scheint und kein Auftrag vorliegt, dann luncht Mr. Steed mit Miss Peel.
    \item Entweder übt Miss Peel Karate oder sie besucht Mr. Steed.
    \item Miss Peel übt Karate genau dann, wenn Mr. Steed Golf spielt oder ein Auftrag vorliegt.
    \item Entweder scheint die Sonne und Mr. Steed spielt Golf oder Miss Peel besucht Mr. Steed und dieser luncht mit ihr.
    \item Es trifft nicht zu, dass Miss Peel Mr. Steed dann besucht, wenn ein Auftrag vorliegt.
    \item Genau dann, wenn kein Auftrag vorliegt, luncht Mr. Steed mit Miss Peel.
\end{enumerate}

\end{document}
