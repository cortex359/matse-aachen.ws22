\documentclass[main.tex]{subfiles}

\begin{document}

\section{Aufgabe 8}
Durch $3$ Punkte $( x_{0} ,y_{0}) ,\ ( x_{1} ,y_{1}) ,\ ( x_{2} ,y_{2})$ soll eine Kurve der Form
\begin{equation*}
    y=a_{0} +a_{1} \cdotp x+\textcolor[rgb]{0.82,0.01,0.11}{a_{2}} \cdotp x^{3}
\end{equation*}
gelegt werden. Stellen Sie das lineare Gleichungssystem auf und untersuchen Sie, in welchem der beiden folgenden Fälle die Lösung eindeutig ist.

\begin{enumerate}
    \item $( x_{0} ,y_{0}) =( -1,0) ,\ ( x_{1} ,y_{1}) =( 0,0) ,\ ( x_{2} ,y_{2}) =( 2,6)$
    \item $( x_{0} ,y_{0}) =( -1,-1) ,\ ( x_{1} ,y_{1}) =( 0,0) ,\ ( x_{2} ,y_{2}) =( 1,1)$
\end{enumerate}

\subsection{Lösung 8}
Das lineare Gleichungssystem sei allgemein:
\begin{equation*}
    \begin{array}{ r c l }
    y_{0} & = & a_{0} +a_{1} x_{0} +a_{2} +x_{0}^{3}\\
    y_{1} & = & a_{0} +a_{1} x_{1} +a_{2} +x_{1}^{3}\\
    y_{2} & = & a_{0} +a_{1} x_{2} +a_{2} +x_{2}^{3}
    \end{array}
\end{equation*}

\subsection{Lösung 8a}
Die erweiterte Koeffizientenmatrix:
\begin{gather*}
    \begin{pmatrix}[1]
    1 & -1 & -1 & 0\\
    1 & 0 & 0 & 0\\
    1 & 2 & 8 & 6
    \end{pmatrix}
\end{gather*}
Wobei die Ergebnisspalte nicht relevant für die Untersuchung der Lösbarkeit ist.
\begin{gather*}
    \det\begin{pmatrix}[1]
    1 & -1 & -1\\
    1 & 0 & 0\\
    1 & 2 & 8
    \end{pmatrix} =6
\end{gather*}
$\Rightarrow$ Das Gleichungssystem ist eindeutig lösbar, da die Determinante ungleich Null ist.

\subsection{Lösung 8b}
Nach dem gleichen Vorgehen gilt:
\begin{gather*}
    \begin{pmatrix}[1]
    1 & -1 & -1 & 0\\
    1 & 0 & 0 & 0\\
    1 & 1 & 1 & 1
    \end{pmatrix}\\
    \\
    \det\begin{pmatrix}[1]
    1 & -1 & -1\\
    1 & 0 & 0\\
    1 & 1 & 1
    \end{pmatrix} =0
\end{gather*}
$\Rightarrow$ Das Gleichungssystem ist \textbf{nicht} oder \textbf{nicht eindeutig} lösbar, da die Determinante ungleich Null ist.

\end{document}
