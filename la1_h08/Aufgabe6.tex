\documentclass[main.tex]{subfiles}

\begin{document}

\section{Aufgabe 6}
Zeigen Sie, dass die folgenden Mengen keine Vektorräume über $\mathbb{R}$ bilden.

\begin{enumerate}
    \item \begin{gather*}
        V=\mathbb{R}^{3}\text{ mit} \notag\\
        \notag\\
        x+y=\begin{pmatrix}[1]
        x_{1}\\
        x_{2}\\
        x_{3}
        \end{pmatrix} +\begin{pmatrix}[1]
        y_{1}\\
        y_{2}\\
        y_{3}
        \end{pmatrix} =\begin{pmatrix}[1]
        x_{1} +y_{1}\\
        x_{2} +y_{2}\\
        x_{3} +y_{3}
        \end{pmatrix}\text{ und} \notag\\
        \notag\\
        \textcolor[rgb]{0.96,0.65,0.14}{\lambda } x=\textcolor[rgb]{0.96,0.65,0.14}{\lambda }\begin{pmatrix}[1]
        x_{1}\\
        x_{2}\\
        x_{3}
        \end{pmatrix} =\begin{pmatrix}[1]
        \textcolor[rgb]{0.96,0.65,0.14}{\lambda } x_{1}\\
        x_{2}\\
        x_{3}
        \end{pmatrix}
    \end{gather*}
    \item \begin{gather*}
        V=\mathbb{R}^{3}\text{ mit} \notag\\
        \notag\\
        x+y=\begin{pmatrix}[1]
        x_{1}\\
        x_{2}\\
        x_{3}
        \end{pmatrix} +\begin{pmatrix}[1]
        y_{1}\\
        y_{2}\\
        y_{3}
        \end{pmatrix} =\begin{pmatrix}[1]
        x_{1} +y_{1}\\
        x_{2} +y_{2}\\
        x_{3} +y_{3}
        \end{pmatrix}\text{ und} \notag\\
        \notag\\
        \lambda x=\lambda \begin{pmatrix}[1]
            x_{1}\\
            x_{2}\\
            x_{3}
            \end{pmatrix} =\begin{pmatrix}[1]
            0\\
            0\\
            0
        \end{pmatrix}
    \end{gather*}
    \item $V=\{x\in \mathbb{R}^{n} |x_{1} \geq 0\}$ mit der Addition und skalaren Multiplikation des $\mathbb{R}^{n}$.
\end{enumerate}

\subsection{Lösung 6a}
Axiom 5
\begin{equation*}
    \forall \lambda ,\mu \in K,x\in V:( \lambda +\mu ) \odot x=( \lambda \odot x) \oplus ( \mu \odot x)
\end{equation*}
verletzt, da
\begin{equation*}
    \begin{array}{ c l }
    \begin{pmatrix}[1]
    ( \lambda +\mu ) x_{1}\\
    x_{2}\\
    x_{3}
    \end{pmatrix} & =\begin{pmatrix}[1]
    \lambda x_{1}\\
    x_{2}\\
    x_{3}
    \end{pmatrix} +\begin{pmatrix}[1]
    \mu x_{1}\\
    x_{2}\\
    x_{3}
    \end{pmatrix}\\
    & =\begin{pmatrix}[1]
    ( \lambda +\mu ) x_{1}\\
    2x_{2}\\
    2x_{3}
    \end{pmatrix} \ \lightning 
    \end{array}
\end{equation*}
Da $x_{2} \neq 2x_{2}$ und $x_{3} \neq 2x_{3}$.


\subsection{Lösung 6b}
Axiom 3
\begin{equation*}
    \forall x\in V:\mathbb{1} \odot x=x
\end{equation*}
(mit $\mathbb{1}$ als dem neutralen Element der Multiplikation) verletzt, da für jedes $x \neq {(0, 0, 0)}^{T}$ 
\begin{equation*}
    \mathbb{1} \odot x=\begin{pmatrix}[1]
    0\\
    0\\
    0
    \end{pmatrix} \neq x
\end{equation*}


\subsection{Lösung 6c}
Abgeschlossenheit verletzt.

Sei $n=2,\ \lambda =-1,\ x_{1} \neq 0$
\begin{equation*}
    -1\cdotp \begin{pmatrix}[1]
    x_{1}\\
    x_{2}
    \end{pmatrix} =\underbrace{\begin{pmatrix}[1]
    -x_{1}\\
    -x_{2}
    \end{pmatrix}}_{\notin V}
\end{equation*}

\end{document}
