\documentclass[12pt,a4paper]{article}

%%%%%%%%%%%%%%% Deutsch %%%%%%%%%%%%%%%
\usepackage[ngerman]{babel}
\usepackage[utf8]{inputenc}

%\usepackage[ansinew]{inputenc}
\usepackage[T1]{fontenc}

% Verfeinerungen des Schriftbildes durch Grauwertausgleich etc.
\usepackage[babel=true]{microtype}

%%%%%%%%%%%%%%% AMS %%%%%%%%%%%%%%%
\usepackage{amsmath}

\usepackage{makeidx}
\usepackage{graphicx}

\usepackage{xcolor}
%\usepackage{etex}  % wird für das Symbol beim beweis-Theorem benötigt

\usepackage{framed} % braucht man für die gerahmten Boxen

%%%% erweiterte Umgebung für Theoreme
\usepackage[amsmath, standard, framed, thmmarks]{ntheorem}

%%% für schräge Brüche in Arrays im Math-mode
%\usepackage{xfrac}
\usepackage{nicefrac}

\definecolor{gruen}{rgb}{0,0.7,0}

% Skript: dunkelblau
\definecolor{idxfarbe}{rgb}{0.0,0.0,0.4}

\definecolor{fhcyan}{rgb}{0,0.69411,0.6745}
\definecolor{fhdarkcyan}{rgb}{0,0.49411, 0.4745}
\definecolor{fhlightcyan}{rgb}{0.4,0.89411, 0.8745}
\definecolor{lightgrey}{rgb}{0.9,0.9,0.9}
\definecolor{defcolor}{RGB}{195,62,62}
\definecolor{red}{rgb}{0.9,0,0}

\usepackage[
naturalnames=true,  % mre 090528 beseitigt \ref Problem in Mik2.7
colorlinks=true,
hyperindex=true,
bookmarks=true,
bookmarksnumbered=true,
filecolor=fhcyan,
breaklinks,
urlcolor=fhcyan,
linkcolor=blue
]{hyperref}

\parindent0cm

\definecolor{grau}{gray}{0.4}

\setcounter{secnumdepth}{0}
\setcounter{tocdepth}{0}

% Index erzeugen lassen
\makeindex

% Serifenschrift: Palatino anstelle der Standard-Schrift
\usepackage[sc]{mathpazo}

% serifenlose Schrift: Helvetica anstelle der Standardschrift
% Die Formeln auf den Folien werden dadurch besser lesbar.
\usepackage[scaled=.95]{helvet}

% Schriftart Courier anstelle der Standardschrift für Maschinenschrift
\usepackage{courier}

\usepackage{listings}
% nötig, um in Formeln Kreise um Zahlen zu ermöglichen
\usepackage{tikz}

\usepackage{relsize}

%%%------------------------------------------------------------------------------------------------------%%%%

\usepackage[a4paper, 
includehead,% See below for an explanation
nomarginpar,% We don't want any margin paragraphs
headheight=28pt, % Set \headheight to 16pt to avoid fancyhr warnings
top=25mm,
bottom=25mm,
left=25mm,
right=25mm
]{geometry}


\usepackage{rotating}
\usepackage{pstricks}

\usepackage{siunitx} % für SI Einheiten
\sisetup{locale = DE}

%\usepackage{physics}

\usepackage{mathdots}
\usepackage{yhmath}
\usepackage{cancel}
\usepackage{color}
\usepackage{array}
\usepackage{multirow}
\usepackage{amssymb}
%\usepackage{gensymb}
%\usepackage{tabularx}
\usepackage{extarrows}
\usepackage{booktabs}

\usetikzlibrary{fadings}
\usetikzlibrary{patterns}
\usetikzlibrary{shadows.blur}
\usetikzlibrary{shapes}

\usepackage{stmaryrd}

\usepackage{lastpage}

\usepackage{fancyhdr}
\pagestyle{fancy}

%---- HEader ---%
\fancyhf{}
\fancyhead[L]{\textbf{Lineare Algebra 1}\\\textbf{WiSe 2022/2023}}
\fancyhead[C]{\textbf{Hausaufgaben}\\\textbf{Blatt 05}}
\fancyhead[R]{Ausgabe: 01.11.2022\\Abgabe: 08.11.2022}

%--- Footer ----%
\fancyfoot[L]{{\small 29.10.2022}}
\fancyfoot[C]{{\small Seite \thepage~ von \pageref{LastPage}}}
\fancyfoot[R]{{\small Christian Rene Thelen}}

\begin{document}

\section{Aufgabe 5}

Bestimmen Sie die allgemeine Gleichung der Geraden durch folgende Punkte:

\begin{equation*}
	\begin{array}{ c l l }
		\text{(a)} & P_{1} =( 2|3) & P_{2} =( 1|1)\\
		\text{(b)} & A=( 0|0) & B=( 2|3)\\
		\text{(c)} & R=( 1|2) & S=( 3|2)
	\end{array}
\end{equation*}

Geben Sie sie in einer parameterlosen Darstellung und einer Parameterdarstellung an.

\subsection{Lösung 5}
Für die Parameterform der Geradengleichung wird der erste Punkt $\displaystyle A$ als \textit{Aufpunkt} oder auch \textit{Stützvektor} gewählt. Der Richtungsvektor $\displaystyle \vec{r}$ ergibt sich aus der Differenz der Ortsvektoren beider Punkte. Allgemein gilt für $\displaystyle \lambda \in \mathbb{R}$:
\begin{equation*}
	G_{PF} : \vec{x}=A+\lambda \cdot \vec{r}
\end{equation*}

Für die Umrechnung in die parameterfreie Hauptform der Geradengleichung werden zunächst die Gleichungen für die beiden Koordinaten abgelesen. Danach wird die Gleichung für die x-Koordinate nach $\displaystyle \lambda $ umgestellt und in die Gleichung für die y-Koordinate eingesetzt, sodass sich eine explizite Form der Geradengleichung mit $\displaystyle k$ als der Steigung und $\displaystyle b$ als dem Ordinatenabschnitt ergibt:

\begin{equation*}
	G_{NF} : x_2 = k\cdot x_1 + b
\end{equation*}

\subsection{Lösung 5a}
\begin{gather*}
	G_{PF} :\vec{x}=\begin{pmatrix}[1]
		2\\
		3
	\end{pmatrix} +\lambda \begin{pmatrix}[1]
		1\\
		2
	\end{pmatrix} \ \Bigl| \ \lambda \in \mathbb{R}\\
	\\
	G_{NF} : \begin{pmatrix}[1]
		2\\
		-1
	\end{pmatrix} \cdot \vec{x} = \begin{pmatrix}[1]
		-2\\
		1
	\end{pmatrix}\begin{pmatrix}[1]
		2\\
		3
	\end{pmatrix}\\
	\Leftrightarrow 2 x_1 - x_2 = -1
\end{gather*}

\subsection{Lösung 5b}
\begin{gather*}
	G_{PF} :\vec{x}=\begin{pmatrix}[1]
		0\\
		0
	\end{pmatrix} +\lambda \begin{pmatrix}[1]
		2\\
		3
	\end{pmatrix} \ \Bigl| \ \lambda \in \mathbb{R}\\
	\\
	G_{NF} : 0 = 3 \cdot x_1 - 2 \cdot x_2
\end{gather*}

\subsection{Lösung 5c}
\begin{gather*}
	G_{PF} :\vec{x}=\begin{pmatrix}[1]
		1\\
		2
	\end{pmatrix} +\lambda \begin{pmatrix}[1]
		2\\
		0
	\end{pmatrix} \ \Bigl| \ \lambda \in \mathbb{R}\\
	\\
	G_{NF} : x_2 = 2
\end{gather*}
\documentclass[main.tex]{subfiles}

\begin{document}

\section{Aufgabe 6}
Berechnen Sie die Bestapproximation des Punktes $V=( 2|3|1)$ auf die, von den Vektoren
\begin{equation*}
    v_{1} = \vektor{1 \\ 2 \\ 2}, v_{2} = \vektor{2 \\ -2 \\ 1}
\end{equation*}

aufgespannte Ebene. Klären Sie zunächst, wie die Bestapproximation in der analytischen Geometrie genannt wird. Nutzen Sie bei der Berechnung die orthogonale Projektion auf Unterräume. Bestimmen Sie auch den minimalen Abstand von $V$ zur Ebene.

\subsection{Lösung 6}
Die Bestapproximation wird in der analytischen Geometrie orthogonale Projektion oder auch Orthogonalprojektion genannt.

Nach Satz 3.151 gilt für die Bestapproximation:
\begin{equation*}
	\norm{v-p_{U}(v)} = \underset{u\in U}{\min} \norm{u-v}
\end{equation*}

Nach Satz 3.121 gilt für die orthogonale Projektion $p_{b}(a)$ eines Vektors $a$ auf $b$ mit $b\neq 0$ in jedem unitären Vektorraum:
\begin{equation*}
	p_{b}( a) =\frac{\scalarprod{a,b}}{\scalarprod{b,b}} \cdot b
\end{equation*}

Für die orthogonale Projektion $p_{U}(a) \in U$ eines Vektors $a \in V$ auf einen endlich erzeugten Untervektorraum $U$ muss nach Satz 3.122 gelten:
\begin{equation*}
	a-p_{U}(a) \perp u \ \forall u\in U
\end{equation*}
Für die orthogonale Projektion $p_{E}( a)$ des Vektors $a$ auf eine Ebene $E$, welche durch den Ursprung verläuft und mit $E=\mu \cdot \vec{v} +\lambda \cdot \vec{u}$ beschrieben ist, wobei $\vec{v} \perp \vec{u}$, gilt entsprechend:
\begin{equation*}
	p_{E}( a) =\frac{\scalarprod{a,v}}{\scalarprod{v,v}} \cdot v+\frac{\scalarprod{a,u}}{\scalarprod{u,u}} \cdot u
\end{equation*}
Wir betrachten die durch $v_{1}$ und $v_{2}$ aufgespannte Ebene als Untervektorraum $U$ des $\mathbb{R}^{3}$. Dadurch, dass wir den Nullvektor als Aufpunkt wählen, stellen wir sicher, dass sich dieser in der Hyperebene befindet und wir somit von einem Untervektorraum sprechen können.
\begin{equation*}
	U = \left\{x\in \mathbb{R}^{3}\middle| x=\mu \vektor{1 \\ 2 \\ 2} + \lambda \vektor{2 \\ -2\\ 1},\ \mu ,\lambda \in \mathbb{R}\right\}
\end{equation*}
Für die Orthogonalprojektion von dem Ortsvektor $a$ des Punktes $V$, mit \ $a=(2;3;1)^{T}$, auf den Untervektorraum $U$ gilt also:
\begin{equation*}
	\begin{array}{ c l }
		p_{U}( a) & =\frac{\scalarprod{ a,v_{1}} }{\scalarprod{ v_{1} ,v_{1}} } \cdot v_{1} + \frac{\scalarprod{ a,v_{2}} }{\scalarprod{ v_{2} ,v_{2}} } \cdot v_{2}\\
		 & =\frac{10}{9} \cdot v_{1} -\frac{1}{9} \cdot v_{2}\\
		 & =\frac{1}{9} \cdot ( 10\cdot v_{1} -v_{2})\\
		 & =\frac{1}{9} \cdot \vektor{8 \\ 22 \\ 19}
	\end{array}
\end{equation*}
Für die Bestapproximation gilt nun:
\begin{equation*}
	\norm{ a-p_{U}( a) } = \norm{
		\vektor{2 \\ 3 \\ 1} - \frac{1}{9}
		\vektor{8 \\ 22 \\ 19}
	} = \frac{\sqrt{225}}{9} = \frac{15}{3}
\end{equation*}

Da die Orthogonalprojektion des Punktes $V$ auf den Untervektorraum $U$ auch Lotfußpunkt genannt wird und der Differenzvektor zwischen dem Punkt $V$ und seiner Orthogonalprojektion das Lot ist, dessen Länge sich über seine $\mathcal{l}_{2}$-Norm berechnet, ist der minimale Abstand von $a$ zur Ebene
\begin{equation*}
	\underset{u\in U}{\min} \norm{u-a} = \norm{a-p_{U}(a)} = \frac{5}{3}\text{.}
\end{equation*}

\end{document}

\section{Aufgabe 7}

Gegeben sei die Gerade $g$ mit

\begin{equation*}
  g:x=\begin{pmatrix}
    1\\
    0
  \end{pmatrix} +\alpha \begin{pmatrix}
    -3\\
    2
  \end{pmatrix} .
\end{equation*}Finden Sie eine Gerade, die senkrecht zu der Geraden g ist und zusätzlich durch den Punkt $\displaystyle ( 4|3)$ geht.

\subsection{Lösung 7}

Durch Bemerkung 2.21 wissen wir, dass der Vektor $\displaystyle \begin{pmatrix}
  -3\\
  2
\end{pmatrix} \perp \begin{pmatrix}
  -2\\
  -3
\end{pmatrix}$ liegt. Damit ist der Richtungsvektor der gesuchten Geraden gegeben.\\

Verwendet man nun den Punkt $\displaystyle (4|3)$ als Aufpunkt, so erhält man die Gerade $h$, welche notwendigerweise senkrecht zu $g$ ist und durch den Punkt $(4|3)$ geht:
\begin{equation*}
  h:x=\begin{pmatrix}
    4\\
    3
  \end{pmatrix} +\beta \begin{pmatrix}
    2\\
    3
  \end{pmatrix}
\end{equation*}
\documentclass[main.tex]{subfiles}

\begin{document}

\section{Aufgabe 8}
Gegeben sind die Vektoren

\begin{align*}
    a=\begin{pmatrix}[1]
    1\\
    -1/2\\
    \beta
    \end{pmatrix} & & b=\begin{pmatrix}[1]
    0\\
    2\alpha \\
    -2
    \end{pmatrix} & & c=\begin{pmatrix}[1]
    -1\\
    -\alpha \\
    1
    \end{pmatrix}
\end{align*}

Bestimmen Sie die Variablen $\alpha$ und $\beta$ derart, dass der aus den 3 Vektoren gebildete Spat das Volumen 17 VE hat und das von den Vektoren $a$ und $b$ aufgespannte Parallelogramm den Flächeninhalt 19 FE hat.

\subsection{Lösung 8}

% Ansatz: 2.43 und Abbildung 2.6
% 2.91 und Abbildung 2.8

Das Spatprodukt dreier Vektoren $a,b,c\in \mathbb{R}^{3}$ ist nach Definition 2.90 und 2.91
\begin{equation*}
    \det( a,b,c) =\langle a,b\times c\rangle \in \mathbb{R}
\end{equation*}
Das Volumen $V$ des durch die Spaltenvektoren aufgespannten Spats ist der Betrag der Determinante $V=| \det( a,b,c)| $.

Für ein gegebenes Volumen $V=17VE$ soll also gelten:
\begin{equation*}
    \begin{array}{ c r l }
    & 17VE & =\left| \det\begin{pmatrix}[1]
    1 & 0 & -1\\
    -1/2 & 2\alpha  & -\alpha \\
    \beta  & -2 & 1
    \end{pmatrix}\right| \\
    \Leftrightarrow  & 17VE & =| 2\alpha \beta -1| \\
    \overset{( 2\alpha \beta -1)  >0}{\Leftrightarrow } & 9VE & =\alpha \beta \\
    \Leftrightarrow  & \beta  & =\frac{9VE}{\alpha }
    \end{array}
\end{equation*}

Für den Flächeninhalt des Parallelogramms $A$ muss die Länge des Vektorprodukts bestimmt werden. Nach Folgerung 2.43 lässt sich der Flächeninhalt so berechnen:
\begin{equation*}
    A=\| a\| \cdotp \| b\| \cdotp \sin \theta =\| a\times b\| 
\end{equation*}

Für eine gegebene Fläche $A=19FE$ soll also gelten:
\begin{equation*}
    \begin{array}{ c r l }
    & A & = \left\| \begin{pmatrix}[1]
    1\\
    -1/2\\
    \beta 
    \end{pmatrix} \times \begin{pmatrix}[1]
    0\\
    2\alpha \\
    -2
    \end{pmatrix} \right\| \\
    &  & = \left\| \begin{pmatrix}[1]
    1-2\alpha \beta \\
    2\\
    2\alpha 
    \end{pmatrix} \right\| \\
    &  & =\sqrt{{( 1-2\alpha \beta )}^{2} +4+4\alpha ^{2}}\\
    &  & =\sqrt{5+4\alpha ^{2} \beta ^{2} -4\alpha \beta +4\alpha ^{2}}\\
    \overset{A=19FE}{\Leftrightarrow } & 19FE & =\sqrt{5+4\alpha ^{2} \beta ^{2} -4\alpha \beta +4\alpha ^{2}}\\
    \overset{\beta =9VE/\alpha }{\Leftrightarrow } & 19FE & =\sqrt{5+4\alpha ^{2}{\left(\frac{9VE}{\alpha }\right)}^{2} -4\alpha \left(\frac{9VE}{\alpha }\right) +4\alpha ^{2}}\\
    \Leftrightarrow  & 19FE & =\sqrt{5+4\cdotp {( 9VE)}^{2} -4\cdotp ( 9VE) +4\alpha ^{2}}\\
    \Leftrightarrow  & 19^{2} VE & =5+4\cdotp {( 9VE)}^{2} -4\cdotp ( 9VE) +4\alpha ^{2}\\
    \Leftrightarrow  & 4\alpha ^{2} & =19^{2} VE-5-4\cdotp {( 9VE)}^{2} +4\cdotp ( 9VE)\\
    \Leftrightarrow  & \alpha ^{2} & =\frac{19^{2} VE}{4} -{( 9VE)}^{2} +( 9VE) -\frac{5}{4}\\
    \Leftrightarrow  & \alpha ^{2} & =\frac{19^{2} VE}{4} +( 9VE) -{( 9VE)}^{2} -\frac{5}{4}\\
    \Leftrightarrow  & \alpha ^{2} & =\left(\frac{397}{4} VE\right) -\left( 81VE^{2}\right) -\frac{5}{4}\\
    \Leftrightarrow  & \alpha  & =\sqrt{\left(\frac{397}{4} VE\right) -\left( 81VE^{2}\right) -\frac{5}{4}}\\
    \Leftrightarrow  & \alpha  & =\sqrt{17}
    \end{array}
\end{equation*}


Somit sind die Parameter $\alpha =\sqrt{17}$ und $\beta =\frac{9}{\sqrt{17}}$ eindeutig bestimmt. 

Die Probe ergibt für das Volumen

\begin{equation*}
    \begin{array}{ r l }
    17VE & =\left| \det\begin{pmatrix}[1]
    1 & 0 & -1\\
    -1/2 & 2\sqrt{17} & -\sqrt{17}\\
    \frac{9}{\sqrt{17}} & -2 & 1
    \end{pmatrix}\right| \\
    & =\left| 2\cdotp \sqrt{17} \cdotp \frac{9}{\sqrt{17}} -1\right| \\
    & =| 18-1| \\
    & =17\ \ \checked 
    \end{array}
\end{equation*}
sowie für den Flächeninhalt

\begin{equation*}
    \begin{array}{ c r l }
        19FE & = & \left\| \begin{pmatrix}[1]
        1-2\cdotp \sqrt{17} \cdotp \frac{9}{\sqrt{17}}\\
        2\\
        2\cdotp \sqrt{17}
        \end{pmatrix} \right\| \\
        & = & \left\| \begin{pmatrix}[1]
        -17\\
        2\\
        2\cdotp \sqrt{17}
        \end{pmatrix} \right\| \\
        & = & \sqrt{{( -17)}^{2} +2^{2} +2^{2} \cdotp 17}\\
        & = & \sqrt{{( -17)}^{2} +4+4\cdotp 17}\\
        & = & \sqrt{361}\\
        & = & 19\ \checked 
    \end{array}
\end{equation*}

\textit{Hinweis: Es reicht die Betrachtung des Falls $(2\alpha\beta-1) > 0$, da die Aufgabenstellung nach einer Lösung und nicht nach allen Lösungen fragt.}

\end{document}

  
\end{document}
