\section{Aufgabe 8}

Ein Gebäude in Form einer Pyramide hat die Eckpunkte $O=( 0|0|0) ,\ A=( 6|8|0) ,\ B=( 0|8|0)$ und die Spitze $S=( 2|4|8)$. Von der Ecke $B$ verläuft zum Punkt $P=( 4|6|4)$ ein Stahlträger.

\begin{enumerate}
	\item[(a)] Zeigen Sie, dass $P$ in der Ebene $E_{OAS}$ liegt, die die Pyramidenseite $OAS$ enthält.
	\item[(b)] Überprüfen Sie, ob der Stahlträger senkrecht auf die Ebene $E_{OAS}$ trifft.
\end{enumerate}


\subsection{Lösung 8a}

Die Ebene $E_{OAS}$ kann mit den Richtungsvektoren $\overrightarrow{OA}$ und $\overrightarrow{OS}$ mit der Gleichung
\begin{equation*}
	\begin{array}{ r l }
		E_{OAS} : & \vec{X} =\lambda \begin{pmatrix}[1]
			6\\
			8\\
			0
		\end{pmatrix} +\mu \begin{pmatrix}[1]
			2\\
			4\\
			8
		\end{pmatrix}
	\end{array}
\end{equation*}
beschrieben werden.

Sei $\vec{p}$ der Ortsvektor zum Punkt $P$, so gilt für $\lambda =\frac{1}{2}$ und $\mu =\frac{1}{2}$ 
\begin{equation*}
	\begin{array}{ c c l }
		& \vec{p} & =\frac{1}{2} \cdotp \begin{pmatrix}[1]
			6\\
			8\\
			0
		\end{pmatrix} +\frac{1}{2} \cdotp \begin{pmatrix}[1]
			2\\
			4\\
			8
		\end{pmatrix}\vspace{2mm}\\
		\Leftrightarrow  & \begin{pmatrix}[1]
			4\\
			6\\
			4
		\end{pmatrix} & =\begin{pmatrix}[1]
			3\\
			4\\
			0
		\end{pmatrix} +\begin{pmatrix}[1]
			1\\
			2\\
			4
		\end{pmatrix} \ \ \checkmark
	\end{array}
\end{equation*}

\subsection{Lösung 8b}
Der Normalenvektor $\overrightarrow{n_{E}}$ der Ebene $E_{OAS}$ ist
\begin{equation*}
	\begin{array}{ c l }
		\overrightarrow{n_{E}} & =\begin{pmatrix}[1]
			6\\
			8\\
			0
		\end{pmatrix} \times \begin{pmatrix}[1]
			2\\
			4\\
			8
		\end{pmatrix}\vspace{2mm}\\
		& =\begin{pmatrix}[1]
			64\\
			-48\\
			8
		\end{pmatrix}\text{.}
	\end{array}
\end{equation*}

Der Stahlträger trifft senkrecht auf die Ebene, wenn $\left< \vec{n_{E}} ,\vec{BP} \right> =0$.
\begin{equation*}
	\begin{array}{ r l }
		0 = &
		\left< \begin{pmatrix}[1]
			64\\
			-48\\
			8
		\end{pmatrix} ,\begin{pmatrix}[1]
			4\\
			6\\
			4
		\end{pmatrix}-\begin{pmatrix}[1]
			0\\
			8\\
			0
		\end{pmatrix}\right>\\
		= & 
		\left< \begin{pmatrix}[1]
			64\\
			-48\\
			8
		\end{pmatrix} ,\begin{pmatrix}[1]
			4\\
			-2\\
			4
		\end{pmatrix}\right>\\
		= & 64\cdot 4 + (-2)\cdot (-48) + 8\cdot 4\\
		= & 256 + 96 + 32\ \ \lightning \\
	\end{array}
\end{equation*}