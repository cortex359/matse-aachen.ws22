\section{Aufgabe 5}

Bestimmen Sie die allgemeine Gleichung der Geraden durch folgende Punkte:
\begin{equation*}
	\begin{array}{ c l l }
		\text{a)} & P_{1} =( 2|3) & P_{2} =( 1|1)\\
		\text{b)} & A=( 0|0) & B=( 2|3)\\
		\text{c)} & R=( 1|2) & S=( 3|2)
	\end{array}
\end{equation*}
Geben Sie sie in einer parameterlosen Darstellung und einer Parameterdarstellung an.



\textbf{Lösung 5}

Für die Parameterform der Geradengleichung wird der erste Punkt $\displaystyle A$ als \textit{Aufpunkt} oder auch \textit{Stützvektor} gewählt. Der Richtungsvektor $\displaystyle \vec{r}$ ergibt sich aus der Differenz der Ortsvektoren beider Punkte. Allgemein gilt für $\displaystyle \lambda \in \mathbb{R}$:
\begin{equation*}
	G:X=A+\lambda \cdotp \vec{r}
\end{equation*}


Für die Umrechnung in die parameterfreie Hauptform der Geradengleichung werden zunächst die Gleichungen für die beiden Koordinaten abgelesen. Danach wird die Gleichung für die x-Koordinate nach $\displaystyle \lambda $ umgestellt und in die Gleichung für die y-Koordinate eingesetzt, sodass sich eine explizite Form der Geradengleichung mit $\displaystyle k$ als der Steigung und $\displaystyle b$ als dem Ordinatenabschnitt ergibt:


\begin{equation*}
	G:y=k\cdotp x+b
\end{equation*}
\textbf{Lösung 5a}
\begin{gather*}
	G_{a} :X=\begin{pmatrix}
		2\\
		3
	\end{pmatrix} +\lambda \begin{pmatrix}
		1\\
		2
	\end{pmatrix} \ \Bigl| \ \lambda \in \mathbb{R}\\
	\\
	G_{a} :\ y=2x-1
\end{gather*}
\textbf{Lösung 5b}
\begin{gather*}
	G_{b} :X=\begin{pmatrix}
		0\\
		0
	\end{pmatrix} +\lambda \begin{pmatrix}
		2\\
		3
	\end{pmatrix} \ \Bigl| \ \lambda \in \mathbb{R}\\
	\\
	G_{b} :y=\frac{3}{2} x
\end{gather*}
\textbf{Lösung 5c}
\begin{gather*}
	G_{c} :X=\begin{pmatrix}
		1\\
		2
	\end{pmatrix} +\lambda \begin{pmatrix}
		2\\
		0
	\end{pmatrix} \ \Bigl| \ \lambda \in \mathbb{R}\\
	\\
	G_{c} :y=2
\end{gather*}