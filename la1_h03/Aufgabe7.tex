\section*{Aufgabe 7}

Prüfen Sie nach, ob die folgenden Punkte Eckpunkte eines gleichschenkligen und rechtwinkligen Dreieck sein können, d.h. ob 2 der Verbindungslinien gleich lang sind und einen rechten Winkel bilden.
\begin{equation*}
  P_{1} =\left( 1\middle| 1+\sqrt{3}\right) ,\ P_{2} =\left( 2+\sqrt{3}\middle| 2\right) ,\ P_{3} =\left( 3\middle| 1-\sqrt{3}\right)
\end{equation*}


\subsection*{Lösung 7}

Man betrachtet die Differenz der Ortsvektoren zu den gegebenen Punkten $\displaystyle \overrightarrow{OP_{i}}$ um den Abstand der Punkte zueinander $\displaystyle \left| \overrightarrow{P_{i} P_{j}}\right| $ und somit die Länge der Seiten des Dreiecks zu ermitteln.


\begin{gather*}
  \overrightarrow{OP_{1}} =\begin{pmatrix}
    1\\
    1+\sqrt{3}
  \end{pmatrix} ,\ \overrightarrow{OP_{2}} =\begin{pmatrix}
    2+\sqrt{3}\\
    2
  \end{pmatrix} ,\ \overrightarrow{OP_{3}} =\begin{pmatrix}
    3\\
    1-\sqrt{3}
  \end{pmatrix}\\
  \\
  \begin{array}{ c l }
    \overrightarrow{P_{1} P_{2}} & =\overrightarrow{OP_{1}} -\overrightarrow{OP_{2}}\\
    & =\begin{pmatrix}
      1\\
      1+\sqrt{3}
    \end{pmatrix} -\begin{pmatrix}
      2+\sqrt{3}\\
      2
    \end{pmatrix}\\
    & =\begin{pmatrix}
      1-2-\sqrt{3}\\
      1+\sqrt{3} -2
    \end{pmatrix}\\
    & =\begin{pmatrix}
      -1-\sqrt{3}\\
      -1+\sqrt{3}
    \end{pmatrix}\\
    \left| \overrightarrow{P_{1} P_{2}}\right|  & =\sqrt{\left( -1-\sqrt{3}\right)^{2} +\left( -1+\sqrt{3}\right)^{2}}\\
    & =\sqrt{8}\\
    & =2\sqrt{2}
  \end{array}\\
  \\
  \\
  \begin{array}{ c l }
    \overrightarrow{P_{2} P_{3}} & =\overrightarrow{OP_{2}} -\overrightarrow{OP_{3}}\\
    & =\begin{pmatrix}
      2+\sqrt{3}\\
      2
    \end{pmatrix} -\begin{pmatrix}
      3\\
      1-\sqrt{3}
    \end{pmatrix}\\
    & =\begin{pmatrix}
      2+\sqrt{3} -3\\
      2-1+\sqrt{3}
    \end{pmatrix}\\
    & =\begin{pmatrix}
      \sqrt{3} -1\\
      \sqrt{3} +1
    \end{pmatrix}\\
    \left| \overrightarrow{P_{2} P_{3}}\right|  & =\sqrt{\left(\sqrt{3} -1\right)^{2} +\left(\sqrt{3} +1\right)^{2}}\\
    & =\sqrt{8}\\
    & =2\sqrt{2}
  \end{array}
\end{gather*}
Somit ist gezeigt, dass der Abstand $\displaystyle \left| \overrightarrow{P_{1} P_{2}}\right| =\left| \overrightarrow{P_{2} P_{3}}\right| $ ist und es sich um die Eckpunkte eines gleichschenkligen Dreiecks handelt. Dabei sind die Vektoren $\displaystyle \overrightarrow{P_{1} P_{2}}$ und $\displaystyle \overrightarrow{P_{2} P_{3}}$ die Schenkel und die Verbindungslinie $\displaystyle \overrightarrow{P_{1} P_{3}}$ die Basis des Dreiecks.


Da die Winkel an den Punkten $\displaystyle P_{1}$ und $\displaystyle P_{3}$, welche gegenüber der Schenkel liegen, gleich groß sein müssen, ist der Winkel $\displaystyle \beta $ am Punkt $\displaystyle P_{2}$ gegenüber der Basis des Dreiecks zu untersuchen.
\begin{equation*}
  \begin{array}{ r r l }
    & \cos( \beta ) = & \frac{\langle \overrightarrow{P_{1} P_{2}} ,\overrightarrow{P_{2} P_{3}} \rangle }{|\overrightarrow{P_{1} P_{2}} |\cdot |\overrightarrow{P_{2} P_{3}} |}\vspace{2mm}\\
    & = & \frac{\left( -1-\sqrt{3}\right) \cdot \left(\sqrt{3} -1\right) +\left( -1+\sqrt{3}\right) \cdot \left(\sqrt{3} +1\right)}{\sqrt{8} \cdot \sqrt{8}}\vspace{2mm}\\
    & = & \frac{\left( -\sqrt{3} +1-3+\sqrt{3}\right) +\left( -\sqrt{3} -1+3+\sqrt{3}\right)}{8}\vspace{2mm}\\
    & = & \frac{( -2) +2}{8}\vspace{2mm}\\
    & = & 0\vspace{2mm}\\
    \Leftrightarrow  & \beta = & \arccos( 0)\vspace{2mm}\\
    \Leftrightarrow  & \beta = & \frac{\pi }{2}\vspace{2mm}\\
    & \Rightarrow  & \overrightarrow{P_{1} P_{2}} \perp \overrightarrow{P_{2} P_{3}}
  \end{array}
\end{equation*}
Dies hätte sich auch einfacher mit der Definition 2.22 zeigen lassen. Es gilt nämlich $\displaystyle a\perp b$ genau dann wenn $\displaystyle \langle a,b\rangle =0$ ist.
\begin{equation*}
  \langle \overrightarrow{P_{1} P_{2}} ,\overrightarrow{P_{2} P_{3}} \rangle =\left( -1-\sqrt{3}\right) \cdot \left(\sqrt{3} -1\right) +\left( -1+\sqrt{3}\right) \cdot \left(\sqrt{3} +1\right) =0\ \checkmark 
\end{equation*}
Verbindet man die gegebenen Punkte so erhält man ein rechtwinkliges, gleichschenkliges Dreieck.