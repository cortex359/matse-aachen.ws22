\documentclass[main.tex]{subfiles}

\begin{document}

\section{Aufgabe 4}

Berechnen Sie jeweils die Ableitung $F'( x)$ der Funktionen

\begin{enumerate}
    \item $F( x) =\int\limits _{t=1}^{x}\sqrt{1+t^{2} \ } dt$
    \item $F( x) =\int\limits _{t=x^{2}}^{1+x^{4}}\frac{\sin( t\cdotp x)}{t} \ dt$
\end{enumerate}

\subsection{Lösung 4a}
Wir betrachten das Integral $\int\nolimits _{1}^{x}\sqrt{1+t^{2} \ } dt$ und substituieren $u( t) =1+t^{2}$. Für die untere Grenze gilt nun $u( 1) =2$ und für die obere Grenze $u( x) =1+x^{2}$. Somit lässt sich die Funktion wie folgt umschreiben.
\begin{equation*}
    \begin{array}{ c l }
        F( x) & =\int\nolimits _{t=1}^{x}\sqrt{1+t^{2} \ } dt\\
        & =\int\nolimits _{2}^{1+x^{2}}\sqrt{u} \ du\\
        & =\left[\frac{2}{3} u^{3/2}\right]_{2}^{1+x^{2}}\\
        & =\frac{2}{3}{\left( 1+x^{2}\right)}^{3/2} -\frac{2}{3}{(2)}^{3/2}\\
        & =\frac{2}{3}{\left( 1+x^{2}\right)}^{3/2} -\frac{4\sqrt{2}}{3}
    \end{array}
\end{equation*}

Die Ableitung der Funktion lässt sich nun mit der Kettenregel bestimmen.
\begin{equation*}
    \begin{array}{ r c l }
        f( x) & = & g( v( x))\\
        f'( x) & = & v'( x) \cdotp g'( v( x))
    \end{array}
\end{equation*}

Mit $v( x) =1+x^{2}$ und $g( v) =\frac{2}{3} v^{3/2}$ und entsprechend $v'( x) =2x$ und $g'( v) =\sqrt{v}$:
\begin{equation*}
    F'( x) =2x\cdotp \sqrt{1+x^{2}}
\end{equation*}

\end{document}
