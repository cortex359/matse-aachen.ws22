\documentclass[main.tex]{subfiles}

\begin{document}

\section{Aufgabe 2}
Berechnen Sie die Mantelfläche des Rotationskörpers, der durch Drehung des Graphen der Wurzelfunktion um die $x$-Achse im Intervall $[0;5]$ entsteht.

\subsection{Lösung 2}
Die Manteloberfläche lässt sich nach der folgenden Formel berechnen:
\begin{equation*}
    M_{a}^{b}( f) \ =\ 2\pi \cdotp \int _{a}^{b} f( x) \cdotp \sqrt{1+( f'( x))^{2}} \ dx
\end{equation*}

Für $f( x) =\sqrt{x}$ und entsprechend $f'( x) =\frac{1}{2} x^{-1/2} =\frac{1}{2\sqrt{x}}$ bedeutet dies im Intervall $x\in [ 0;5]$:
\begin{equation*}
    \begin{array}{ c l }
        M_{a}^{b}( f) \  & =\ 2\pi \cdotp \int _{0}^{5}\sqrt{x} \cdotp \sqrt{1+\left(\frac{1}{2\sqrt{x}}\right)^{2}} \ dx\\
        & =\ 2\pi \cdotp \int _{0}^{5}\sqrt{x} \cdotp \sqrt{1+\frac{1}{4x}} \ dx\\
        & =\ 2\pi \cdotp \int _{0}^{5}\sqrt{x} \cdotp \sqrt{\frac{4x+1}{4x}} \ dx\\
        & =\ 2\pi \cdotp \int\nolimits _{0}^{5}\sqrt{x} \cdotp \sqrt{\frac{x+\frac{1}{4}}{x}} \ dx\\
        & =\ 2\pi \cdotp \int\nolimits _{0}^{5}\sqrt{x} \cdotp \frac{\sqrt{x+\frac{1}{4}}}{\sqrt{x}} \ dx\\
        & =\ 2\pi \cdotp \int\nolimits _{0}^{5}\sqrt{x+\frac{1}{4}} \ dx
    \end{array}
\end{equation*}

Die Substitution mit $u=x+\frac{1}{4}$ und $du=dx$ gibt die neue untere Grenze $u( 0) =0+\frac{1}{4} =\frac{1}{4}$ und die obere Grenze $u( 5) =5+\frac{1}{4} =\frac{21}{4}$. 
\begin{equation*}
    \begin{array}{ c l }
        M_{a}^{b}( f) \  & =\ 2\pi \cdotp \int\nolimits _{1/4}^{21/4}\sqrt{u} \ du\\
        & =\ 2\pi \cdotp \left[\frac{2}{3} x^{3/2}\right]_{1/4}^{21/4}\\
        & =\ 2\pi \cdotp \left( \ \frac{2}{3}\left(\frac{21}{4}\right)^{3/2} -\frac{2}{3}\left(\frac{1}{4}\right)^{3/2}\right)\\
        & =\ \frac{4\pi }{3} \cdotp \left( \ \left(\frac{21}{4}\right)^{3/2} -\left(\frac{1}{4}\right)^{3/2}\right)\\
        & =\ \frac{\pi }{6} \cdotp \left( 21\sqrt{21} -1\right)\\
        & \approx \ 49,864\ FE
    \end{array}
\end{equation*}
\end{document}
