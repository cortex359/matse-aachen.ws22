\documentclass[main.tex]{subfiles}

\begin{document}

\section{Aufgabe 8}
Zeigen Sie, dass die Vektoren
\begin{equation*}
x=\begin{pmatrix}[1]
-3/5\\
4/5\\
0
\end{pmatrix} ,\ y=\begin{pmatrix}[1]
4/5\\
3/5\\
0
\end{pmatrix} ,\ z=\begin{pmatrix}[1]
0\\
0\\
1
\end{pmatrix}
\end{equation*}
eine Orthonormalbasis des euklidischen Raums $R^{3}$ (mit dem Standardskalarprodukt) bilden.

\subsection{Lösung 8}
Nach Definition 3.128, Seite 111 muss für eine Orthonormalbasis $\mathcal{B}_{ON} =( x,y,z)$ folgende Bedingungen erfüllt sein:

1. Orthogonalsystem: $x\perp y\land y\perp z\land x\perp z$

2. Orthonormalsystem: $\| x\| =\| y\| =\| z\| =1$

3. Orthogonalbasis: $x,y,z$ ist ein minimales Erzeugendensystem vom Vektorraum $V$\\


1. $\mathcal{B}_{ON}$ ist ein Orthogonalsystem, wenn alle Vektoren paarweise orthogonal sind:
\begin{equation*}
    \begin{array}{ c l }
        <x,y>  & =\frac{-3\cdotp 4}{5} +\frac{4\cdotp 3}{5} +0\\
        & =0\\
        & \Rightarrow x\perp y\\
        & \\
        <y,z>  & =\frac{4\cdotp 0}{5} +\frac{3\cdotp 0}{5} +0\\
        & =0\\
        & \Rightarrow y\perp z\\
        & \\
        <x,z>  & =\frac{-3\cdotp 0}{5} +\frac{4\cdotp 0}{5} +0\\
        & =0\\
        & \Rightarrow x\perp z
    \end{array}
\end{equation*}
$\Rightarrow x\perp y\land y\perp z\land x\perp z\ \checked $\\


2. Ein Orthogonalsystem $\mathcal{B}_{ON}$ ist ein Orthonormalsystem, wenn die Norm aller Vektoren $1$ beträgt:
\begin{equation*}
    \begin{array}{ c l }
        \| x\|  & =\sqrt{\left(\frac{-3}{5}\right)^{2} +\left(\frac{4}{5}\right)^{2} +0^{2}}\\
        & =\sqrt{\frac{9}{25} +\frac{16}{25}}\\
        & =\sqrt{\frac{25}{25}}\\
        & =1\\
        & \\
        \| y\|  & =\sqrt{\left(\frac{4}{5}\right)^{2} +\left(\frac{3}{5}\right)^{2} +0^{2}}\\
        & =\sqrt{\frac{16}{25} +\frac{9}{25}}\\
        & =1\\
        & \\
        \| z\|  & =\sqrt{0^{2} +0^{2} +1^{2}}\\
        & =1
    \end{array}
\end{equation*}
$\Rightarrow \| x\| =\| y\| =\| z\| =1$\\


3. Ein Orthogonalsystem $\mathcal{B}_{ON}$ ist eine Orthogonalbasis von $V$, wenn es eine Basis von $V$ ist. 
Man sieht, dass die Vektoren $x,y,z$ eine Lineare Hülle aufspannen, sodass $L( x,y,z) =\mathbb{R}^{3}$. Es bleibt die lineare Unabhängigkeit der Vektoren zu zeigen:
\begin{equation*}
    \begin{array}{ c l }
        \det( x,y,z) & =\frac{-9}{5} +0+0-0-0-\left(\frac{-9}{5}\right)\\
        & =0
    \end{array}
\end{equation*}
$\Rightarrow $$x,y,z$ ist ein minimales Erzeugendes System vom Vektorraum $V$. $\checked $\\

Aus 1, 2 und 3 folgt, dass es sich bei der $\mathcal{B}_{ON}$ um eine Orthonormalbasis des Vektorraums $\mathbb{R}^{3}$ handelt.

\end{document}
