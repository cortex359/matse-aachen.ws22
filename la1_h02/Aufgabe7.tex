% !TeX root = la1_thelen_christianrene_h02.tex
\section{Aufgabe 7}

Zeigen Sie, dass durch $\displaystyle \langle u,v\rangle =u_{1} v_{1} +u_{2} v_{3} -u_{3} v_{2} +u_{4} v_{4}$ für 

\begin{equation*}
  u=\begin{pmatrix}
    u_{1}\\
    u_{2}\\
    u_{3}\\
    u_{4}
  \end{pmatrix} \ \text{ und } \ v=\begin{pmatrix}
    v_{1}\\
    v_{2}\\
    v_{3}\\
    v_{4}
  \end{pmatrix}
\end{equation*}
kein Skalarprodukt definiert wird. 


\subsection{Lösung 7}

Bedingung SP1 (Symmetrie): Wenn die beschriebene Abbildung $\displaystyle \langle u,v\rangle $ ein Skalarprodukt definieren würde, dann müsste gelten $\displaystyle \langle u,v\rangle =\langle v,u\rangle $ für alle $\displaystyle u,v\in \mathbb{R}^{n}$ und somit
\begin{equation*}
  \begin{array}{ c r c l l }
    & u_{1} v_{1} +u_{2} v_{3} -u_{3} v_{2} +u_{4} v_{4} & = & v_{1} u_{1} +v_{2} u_{3} -v_{3} u_{2} +v_{4} u_{4} & |\ -( u_{1} v_{1}) \ |\ -( u_{4} v_{4})\\
    \Leftrightarrow  & u_{2} v_{3} -u_{3} v_{2} & = & v_{2} u_{3} -v_{3} u_{2} & \\
    \Leftrightarrow  & u_{2} v_{3} -u_{3} v_{2} & = & u_{3} v_{2} -u_{2} v_{3} & 
  \end{array}
\end{equation*}
Dies gilt aber nicht.
