% !TeX root = la1_thelen_christianrene_h02.tex
\section{Aufgabe 8}

Bestimmen Sie die Lösungsmenge des LGS. 
\begin{equation*}
  \begin{array}{ r l }
    1x_{1} -4x_{2} +9x_{3} & =1\\
    2x_{1} +4x_{2} -12x_{3} & =2\\
    -3x_{1} +3x_{2} -3x_{3} & =3
  \end{array}
\end{equation*}

\subsection{Lösung 8}

Das LGS lässt sich als erweiterte Matrix $\displaystyle [ X|\vec{y}]$ schreiben und mit dem Gaußschen Eliminationsverfahren in die Zeilenstufenform bringen:
\begin{align*}
  \begin{pmatrix}
    1 & -4 & 9 & 1\\
    2 & 4 & -12 & 2\\
    -3 & 3 & -3 & 3
  \end{pmatrix} & \begin{array}{ l c }
    & \\
    \cdot ( -\frac{1}{2}) +I & \\
    +( 3\cdot I) & 
  \end{array}\\
  \begin{pmatrix}
    1 & -4 & 9 & 1\\
    0 & -6 & 15 & 0\\
    0 & -9 & 24 & 6
  \end{pmatrix} & \begin{array}{ l c }
    & \\
    :( -6) & \\
    & 
  \end{array}\\
  \begin{pmatrix}
    1 & -4 & 9 & 1\\
    0 & 1 & \sfrac{-5}{2} & 0\\
    0 & -9 & 24 & 6
  \end{pmatrix} & \begin{array}{ l c }
    & \\
    & \\
    +( II\cdot 9) & 
  \end{array}\\
  \begin{pmatrix}
    1 & -4 & 9 & 1\\
    0 & 1 & \sfrac{-5}{2} & 0\\
    0 & 0 & \sfrac{3}{2} & 6
  \end{pmatrix} & \begin{array}{ l c }
    & \\
    & \\
    \cdot \frac{2}{3} & 
  \end{array}\\
  \begin{pmatrix}
    1 & -4 & 9 & 1\\
    0 & 1 & \sfrac{-5}{2} & 0\\
    0 & 0 & 1 & 4
  \end{pmatrix} & 
\end{align*}

Daraus folgt für die nun übrigen Gleichungen:
\begin{equation*}
  \begin{array}{ c r l }
    & \boldsymbol{x_{3}} & = \boldsymbol{4}\\
    &  & \\
    & x_{2} -\frac{5} x_{3} & =0\ \\
    \Leftrightarrow \  & \boldsymbol{x_{2}} & = \boldsymbol{10}\\
    &  & \\
    & x_{1} -4x_{2} +9x_{3} & =1\\
    \Leftrightarrow  & x_{1} -40+36 & =1\\
    \Leftrightarrow  & \boldsymbol{x_{1}} & = \boldsymbol{5}
  \end{array}
\end{equation*}

Es existiert eine eindeutige Lösung.

$\displaystyle \Rightarrow \mathbb{L} =\left\{\begin{pmatrix}
  5\\
  10\\
  4
\end{pmatrix}\right\}$