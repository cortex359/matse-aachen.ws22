\documentclass[main.tex]{subfiles}

\begin{document}

\section{Aufgabe 4}
Differenzieren Sie:

\begin{enumerate}
    \item $f( x) =\frac{e^{x} -e^{-x}}{e^{x} +e^{-x}}$
    \item $f( x) =\arcsin\frac{1-x^{2}}{1+x^{2}}$
    \item $f( x) =x^{\cos( x)}$
    \item $f( x) =\sqrt{x\cdotp \sqrt{x\cdotp \sqrt{x}}}$
    \item $f( x) =x{^{x}}^{a} \ \text{für} \ a >0$
    \item $f( x) =x{^{a}}^{x} \ \text{für} \ a >0$
    \item $f( x) =\cos\left(\ln\left(\tan\left(\sqrt{1+x^{2}}\right)\right)\right)$
    \item $f( x) =x^{2} \cdotp e^{\frac{x}{x+1}}$
\end{enumerate}

\subsection{Lösung 4}
Produktregel (Schelthoff\footnotemark{}, Satz 159)
\begin{equation*}
    \begin{array}{ r c l }
    f( x) & = & u( x) \cdotp v( x)\\
    f'( x) & = & u( x) \cdotp v'( x) +u'( x) \cdotp v( x)
    \end{array}
\end{equation*}

Quotientenregel (Schelthoff\footnotemark[\value{footnote}], Satz 160)
\begin{equation*}
    \begin{array}{ r c l }
    f( x) & = & \frac{u( x)}{v( x)}\\
    f'( x) & = & \frac{u'( x) v( x) -v'( x) u( x)}{v( x)^{2}}
    \end{array}
\end{equation*}

Kettenregel (Schelthoff\footnotemark[\value{footnote}], Satz 161)
\begin{equation*}
    \begin{array}{ r c l }
    f( x) & = & g( v( x))\\
    f'( x) & = & v'( x) \cdotp g'( v( x))
    \end{array}
\end{equation*}

\footnotetext{Schelthoff, Christof (2018): MATSE-MATIK. Analysis 1, 6. Auflage, Aachen, Shaker Verlag.}

\subsubsection{Lösung 4a}
Nach Quotientenregel:
\begin{equation*}
    \begin{array}{ c l }
    u( x) & =e^{x} -e^{-x}\\
    v( x) & =e^{x} +e^{-x}\\
    u'( x) & =e^{-x} +e^{x}\\
    v'( x) & =e^{x} -e^{-x}\\
    f'( x) & =\frac{\left( e^{-x} +e^{x}\right)\left( e^{x} +e^{-x}\right) -\left( e^{x} -e^{-x}\right)\left( e^{x} -e^{-x}\right)}{\left( e^{x} +e^{-x}\right)\left( e^{x} +e^{-x}\right)}\\
    & =\frac{\left( e^{x} +e^{-x}\right)\left(\left( e^{-x} +e^{x}\right) -\left( e^{x} -e^{-x}\right)\right)}{\left( e^{x} +e^{-x}\right)\left( e^{x} +e^{-x}\right)}\\
    & =\frac{\left( e^{-x} +e^{x}\right) -\left( e^{x} -e^{-x}\right)}{\left( e^{x} +e^{-x}\right)}\\
    & =\frac{2\cdotp e^{-x}}{e^{x} +e^{-x}}
    \end{array}
\end{equation*}

\subsubsection{Lösung 4b}
Kettenregel
\begin{equation*}
    \begin{array}{ r c l }
    f( x) & = & \arcsin\frac{1-x^{2}}{1+x^{2}}\\
    f( x) & = & g( v( x))\\
    g( x) & = & \arcsin( x)\\
    g'( x) & = & \frac{1}{\sqrt{1-x^{2}}}\\
    v( x) & = & \frac{1-x^{2}}{1+x^{2}}\\
    v'( x) & = & \frac{-4x}{\left( 1+x^{2}\right)^{2}}\\
    f'( x) & = & v'( x) \cdotp g'( v( x))\\
    & = & \frac{-4x}{\left( 1+x^{2}\right)^{2}} \cdotp \frac{1}{\sqrt{1-\left(\frac{1-x^{2}}{1+x^{2}}\right)^{2}}}\\
    & = & \frac{-4x}{\left( 1+x^{2}\right)^{2} \cdotp \sqrt{1-\left(\frac{1-x^{2}}{1+x^{2}}\right)^{2}}}
    \end{array}
\end{equation*}


\subsubsection{Lösung 4c}
Kettenregel
\begin{equation*}
    \begin{array}{ r c l }
    f( x) & = & x^{\cos (x)}\\
    f( x) & = & g( v( x))\\
    u & = & v( x)\\
    v( x) & = & \cos( x)\\
    v'( x) & = & -\sin( x)\\
    g( u) & = & x^{u}\\
    g'( u) & = & u\cdotp x^{( u-1)}\\
    g'( v( x)) & = & \cos( x) \cdotp x^{(\cos( x) -1)}\\
    &  & \\
    f'( x) & = & v'( x) \cdotp g'( v( x))\\
    & = & -\sin( x) \cdotp \cos( x) \cdotp x^{(\cos( x) -1)}
    \end{array}
\end{equation*}

\subsubsection{Lösung 4d}
Potenzregel
\begin{equation*}
    \begin{array}{ r c l }
    f( x) & = & \sqrt{x\cdotp \sqrt{x\cdotp \sqrt{x}}}\\
    f'( x) & = & \frac{7}{8x^{1/8}}
    \end{array}
\end{equation*}

\subsubsection{Lösung 4e-h}
\textcolor[rgb]{0.55,0.89,0.2}{Aus Zeitmangel ausgelassen. Es wäre schon ziemlich dämlich für die 8 Unterpunkte von Aufgabe 4 jeweils 3 Punkte zu vergeben, oder? xD}

\end{document}
