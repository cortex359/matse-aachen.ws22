\documentclass[main.tex]{subfiles}

\begin{document}

\section{Aufgabe 1}
Bestimmen Sie mit Hilfe des Differentialquotienten die Tangentensteigung an der Stelle $x_0 = 3$ von folgenden Funktionen:

\begin{align*}
    \text{a) }\ f(x) = 2x^3+x-1
    & &
    \text{b) }\ f(x) = \frac{2}{3x^3}
    & &
    \text{c) }\ f(x) = \sqrt{2x+3}
\end{align*}

\subsection{Lösung 1}

Der Differentialquotient oder auch die Ableitung von $f$ an der Stelle $x_{0}$ ist:
\begin{equation*}
    \lim _{x\rightarrow x_{0}} \upDelta ( x) =f'( x_{0}) =\lim _{h\rightarrow 0}\frac{f( x_{0} +h) -f( x_{0})}{h}
\end{equation*}

Entsprechend gilt im Folgenden:

\subsubsection*{Lösung 1a}
\begin{equation*}
    \begin{array}{ c c l }
    \lim\limits _{x\rightarrow x_{0}} \upDelta ( x) & \eqdef  & \lim\limits _{h\rightarrow 0}\frac{\left( 2( x_{0} +h)^{3} +x_{0} +h-1\right) -\left( 2x_{0}^{3} +x_{0} -1\right)}{h}\\
    & = & \lim\limits _{h\rightarrow 0}\frac{2( x_{0} +h)^{3} +h-2x_{0}^{3}}{h}\\
    & = & \lim\limits _{h\rightarrow 0}\frac{2x_{0}^{3} +6x_{0}^{2} h+6x_{0} h^{2} +2h^{3} +h-2x_{0}^{3}}{h}\\
    & = & \lim\limits _{h\rightarrow 0}\frac{h\cdotp \left( 6x_{0}^{2} +6x_{0} h +2h^{2} +1\right)}{h}\\
    & = & \lim\limits _{h\rightarrow 0} 6x_{0}^{2} +6x_{0} h +2h^{2} +1\\
    & = & 6x_{0}^{2} +1\\
    & \overset{x_{0} =\ 3}{=} & 6\cdotp 9+1\\
    & = & 55
    \end{array}
\end{equation*}


\subsubsection*{Lösung 1b}
\begin{equation*}
    \begin{array}{ c c l }
    \lim\limits _{x\rightarrow x_{0}} \upDelta ( x) & \eqdef  & \lim\limits _{h\rightarrow 0}\frac{\frac{2}{3( x_{0} +h)^{3}} -\frac{2}{3( x_{0})^{3}}}{h}\\
    & = & \lim\limits _{h\rightarrow 0}\frac{\frac{6\cdotp x{_{0}}^{3} -\ 6\cdotp ( x_{0} +h)^{3}}{9\cdotp ( x_{0} +h)^{3} \cdotp ( x_{0})^{3}}}{h}\\
    & = & \lim\limits _{h\rightarrow 0}\frac{6\cdotp x{_{0}}^{3} -\ 6\cdotp \left( x_{0}^{3} +3x_{0}^{2} h+3x_{0} h^{2} +h^{3}\right)}{9\cdotp \left( x_{0}^{3} +3x_{0}^{2} h+3x_{0} h^{2} +h^{3}\right) \cdotp x{_{0}}^{3}} \cdotp \frac{1}{h}\\
    & = & \lim\limits _{h\rightarrow 0}\frac{h\cdotp \left( -18x_{0}^{2} -18x_{0} h^{1} -6\cdotp h^{2}\right)}{\left( 9x_{0}^{3} +27x_{0}^{2} h+27x_{0} h^{2} +9h^{3}\right) \cdotp x{_{0}}^{3} \cdotp h}\\
    & = & \lim\limits _{h\rightarrow 0}\frac{-18x_{0}^{2} -18x_{0} h -6\cdotp h^{2}}{\left( 9x_{0}^{3} +27x_{0}^{2} h+27x_{0} h^{2} +9h^{3}\right) \cdotp x{_{0}}^{3}}\\
    & = & \frac{-18x_{0}^{2} -\lim\limits _{h\rightarrow 0}\left( 18x_{0} h -6\cdotp h^{2}\right)}{9x_{0}^{6} +\lim\limits _{h\rightarrow 0}\left( 27x_{0}^{5} h+27x_{0}^{4} h^{2} +9h^{3} x{_{0}}^{3}\right)}\\
    & = & \frac{-18x_{0}^{2}}{9x_{0}^{6}}\\
    & = & -\frac{2}{x_{0}^{4}}\\
    & \overset{x_{0} =\ 3}{=} & -\frac{2}{3^{4}}\\
    & = & -\frac{2}{81}
    \end{array}
\end{equation*}

\subsubsection*{Lösung 1c}
\begin{equation*}
    \begin{array}{ c c l }
    \lim\limits _{x\rightarrow x_{0}} \upDelta ( x) & \eqdef  & \lim\limits _{h\rightarrow 0}\frac{\sqrt{2( x_{0} +h) +3} -\sqrt{2x_{0} +3}}{h}\\
    & = & \lim\limits _{h\rightarrow 0}\frac{\sqrt{2x_{0} +2h+3} -\sqrt{2x_{0} +3}}{h}\\
    & \overset{x_{0} =\ 3}{=} & \lim\limits _{h\rightarrow 0}\frac{\sqrt{6+2h+3} -\sqrt{6+3}}{h}\\
    & = & \lim\limits _{h\rightarrow 0}\frac{\sqrt{9+2h} -\sqrt{9}}{h}\\
    & = & \lim\limits _{h\rightarrow 0}\frac{\left(\sqrt{9+2h} -3\right) \cdotp \left(\sqrt{9+2h} +3\right)}{h\cdotp \left(\sqrt{9+2h} +3\right)}\\
    & = & \lim\limits _{h\rightarrow 0}\frac{9+2h-9}{h\cdotp \left(\sqrt{9+2h} +3\right)}\\
    & = & \lim\limits _{h\rightarrow 0}\frac{2}{\sqrt{9+2h} +3}\\
    & = & \frac{2}{\sqrt{9} +3}\\
    & = & \frac{1}{3}
    \end{array}
\end{equation*}


\end{document}
