\documentclass[main.tex]{subfiles}

\begin{document}

\section{Aufgabe 2}
Berechnen Sie mit Hilfe des Differenzenquotienten die Ableitung der folgenden Funktionen an einem Punkt $x_0$.
\begin{align*}
    \text{a) }\ j(x) = 3x & &
    \text{b) }\ k(x) = x^2+5 & &
    \text{c) }\ l(x) = x^3 +1
\end{align*}

\subsection*{Lösung 2}

Der Differenzenquotient von $f$ in $x_{0}$ ist definiert als
\begin{equation*}
    \upDelta ( x) :=\frac{f( x) -f( x_{0})}{x-x_{0}}
\end{equation*}


\subsubsection*{Lösung 2a}
\begin{equation*}
    \begin{array}{ c c l }
    \Delta ( x) & \eqdef  & \frac{3x-3x_{0}}{x-x_{0}}\\
    & = & \frac{3\cdotp ( x-x_{0})}{x-x_{0}}\\
    & = & 3
    \end{array}
\end{equation*}


\subsubsection*{Lösung 2b}
\begin{equation*}
    \begin{array}{ c c l }
    \Delta ( x) & \eqdef  & \frac{x^{2} +5-\left( x_{0}^{2} +5\right)}{x-x_{0}}\\
    & = & \frac{x^{2} -x_{0}^{2}}{x-x_{0}}\\
    & = & \frac{( x+x_{0}) \cdotp ( x-x_{0})}{x-x_{0}}\\
    & = & x+x_{0}
    \end{array}
\end{equation*}

\subsubsection*{Lösung 2c}
\begin{equation*}
    \begin{array}{ c c l }
    \Delta ( x) & \eqdef  & \frac{x^{3} +1-\left( x_{0}^{3} +1\right)}{x-x_{0}}\\
     & = & \frac{x^{3} -x_{0}^{3}}{x-x_{0}}\\
     & = & \frac{\left( x^{2} +x\cdotp x_{0} +x_{0}^{2}\right) \cdotp ( x-x_{0})}{x-x_{0}}\\
     & = & x^{2} +x\cdotp x_{0} +x_{0}^{2}
    \end{array}
\end{equation*}


\end{document}
