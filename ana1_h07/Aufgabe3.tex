\documentclass[main.tex]{subfiles}

\begin{document}

\section{Aufgabe 3}
\begin{enumerate}
    \item Bestimmen Sie die Tangentengleichung der Kurve $f(x) = x^3$ an der Stelle $x=2$.
    \item Bestimmen Sie die Gerade, welche eine Tangente an der folgenden Funktion ist:
\end{enumerate}

\begin{equation*}
    f(x) = x^2 \ \text{ und }\ g(x) = x^2-2x
\end{equation*}

\subsection{Lösung 3}
Der Ansatz für die Tangente einer Funktion $f( x)$ in einem Punkt $x_{0}$ lautet
\begin{equation*}
    T( x) =f_{1}( x) =m\cdotp ( x-x_{0}) +b
\end{equation*}
Dabei gilt für die Steigung $m=f'( x_{0})$ und für $b=f( x_{0})$.

Die Tangentengleichung lautet somit:
\begin{equation*}
    T( x) =f'( x_{0}) \cdotp ( x-x_{0}) +f( x_{0})
\end{equation*}

\subsubsection*{Lösung 3a}
Für die Tangente $T( x)$ von der Funktion $f( x) =x^{3}$ an der Stelle $x_{0} =2$ gilt somit:
\begin{gather*}
    \begin{array}{ r c l }
    f'( x) & = & 3x^{2}\\
    f'( 2) & = & 12\\
    f( 2) & = & 8
    \end{array}\\
    \\
    \begin{array}{ c l }
    T( x) & =12\cdotp ( x-2) +8\\
    & =12x-24+8\\
    & =12x-16
    \end{array}
\end{gather*}

\subsubsection*{Lösung 3b}

\textbf{Ansatz 1:}\\

Die allgemeinen Tangentengleichungen der Funktionen ergeben sich aus der genannten Formel und den Ableitungen der Funktionen.
\begin{equation*}
    \begin{array}{ r c l }
        f'( x) & = & 2x\\
        g'( x) & = & 2x-2
    \end{array}
\end{equation*}

Eine gemeinsame Tangente muss zwei Bedingungen erfüllen.
\begin{enumerate}
    \item $\exists \ x_{0} :f'( x_{0}) =g'( x_{0})$ und
    \item $f( x_{0}) =g( x_{0})$
\end{enumerate}



Bedingung 1:
\begin{gather*}
    2x=2y-2\\
    \\
    x=y-1
\end{gather*}


Bedingung 2:
\begin{gather*}
    x^{2} =y^{2} -2y\\
    x^{2} =y\cdotp ( y-2)
\end{gather*}

Aus 1 und 2 erhält man:
\begin{gather*}
    ( y-1)( y-1) =y\cdotp ( y-2)\\
    \\
    y^{2} -y-y+1=y^{2} -2y\\
    y^{2} -2y+1=y^{2} -2y\\
    1=0\ \lightning 
\end{gather*}

Beide Bedingungen können nicht zusammen erfüllt sein. 

\textbf{Ansatz 2:}\\

\begin{gather*}
    T_{f}( x) =2x_{0} \cdotp ( x-x_{0}) +x^{2}\\
    T_{g}( x) =( 2x_{0} -2) \cdotp ( x-x_{0}) +\left( x^{2} -2x\right)
\end{gather*}

Gesucht ist eine gemeinsame Tangente, also soll gelten $T_{f}( x) =T_{g}( x)$:
\begin{equation*}
    \begin{array}{ c r c l l }
     & 2x_{0} \cdotp ( x-x_{0}) +x^{2} & = & ( 2x_{0} -2) \cdotp ( x-x_{0}) +\left( x^{2} -2x\right) & \\
    \Leftrightarrow  & 2x_{0} x-2x_{0}^{2} +x^{2} & = & 2x_{0} x-2x_{0}^{2} -2x+2x_{0} +x^{2} -2x & \Bigl| -x^{2}\\
    \Leftrightarrow  & 2x_{0} x-2x_{0}^{2} & = & 2x_{0} x-2x_{0}^{2} -2x+2x_{0} -2x & \Bigl| -2x_{0} x\\
    \Leftrightarrow  & -2x_{0}^{2} & = & -2x_{0}^{2} -2x+2x_{0} -2x & \Bigl| +2x_{0}^{2}\\
    \Leftrightarrow  & 0 & = & 2x_{0} -4x & \Bigl| +4x\\
    \Leftrightarrow  & 2x & = & x_{0} & 
    \end{array}
\end{equation*}

Setzt man nun die gefundene Information für $x_{0}$ in die Tangentengleichungen ein, so erhält man:
\begin{equation*}
    \begin{array}{ c c l }
    T_{f}( x) & = & 4x\cdotp ( x-2x) +x^{2}\\
     & = & -4x^{2} +x^{2}\\
     & = & -3x^{2}
    \end{array}
\end{equation*}

Zur Probe auch noch in die zweite Tangentengleichung:
\begin{equation*}
    \begin{array}{ c c l }
    T_{g}( x) & = & ( 2x_{0} -2) \cdotp ( x-x_{0}) +\left( x^{2} -2x\right)\\
     & = & ( 4x-2) \cdotp ( x-2x) +\left( x^{2} -2x\right)\\
     & = & -4x^{2} +2x+x^{2} -2x\\
     & = & -4x^{2} +x^{2}\\
     & = & -3x^{2}
    \end{array}
\end{equation*}
Somit ist gezeigt, dass $T_{f}( x) =T_{g}( x) \ =-3x^{2}$.

Problematisch ist nur, dass es sich dabei um eine Parabel und keine Tangente handelt, außer vielleicht eine tangierende Parabel, aber es war ja nach einer Geraden gesucht.


\end{document}
