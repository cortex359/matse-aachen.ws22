\documentclass[main.tex]{subfiles}

\begin{document}

\section{Aufgabe 7}

Bildet $\mathbb{N}_0$ mit der Verknüpfung
\begin{equation*}
    a \circ b = \left| a-b \right|
\end{equation*}
eine abelsche Gruppe?

\subsection{Lösung 7}

Es ist zu untersuchen, ob es sich bei $G=(\mathbb{N}_{0},\circ )$ um eine abelsche Gruppe handelt. 

Für die Verknüpfung $\circ $ gilt nach Definition der Betragsfunktion:

\begin{equation*}
    a\circ b:=| a-b| = 
    \begin{cases}
        ( a-b)              & \text{für} \ a>b \\
            0               & \text{für} \ a=b \\
        ( a-b) \cdotp ( -1) & \text{für} \ b >a
    \end{cases}
\end{equation*}



\textbf{[G0]} Abgeschlossenheit: 

Beweis der Abgeschlossenheit durch vollständige Fallunterscheidung:
\begin{equation*}
    \begin{array}{ c r l c }
        \text{Fall 1: } & ( a-b) \in \mathbb{N}_{0} & \text{für} \ a >b & \checked \\
        \text{Fall 2: } & 0\ \in \mathbb{N}_{0} & \text{für} \ a=b & \checked \\
        \text{Fall 3: } & ( -1) \cdotp ( a-b) \ \in \mathbb{N}_{0} & \text{für} \ b >a & \checked 
    \end{array}
\end{equation*}
Daraus folgt $\forall a,b\in \mathbb{N}_{0} :| a-b| \in \mathbb{N}_{0}$ und die Abgeschlossenheit ist gezeigt.\\

\textbf{[G1]} Assoziativität:

Es ist zu untersuchen, ob
\begin{gather*}
    \forall a,b,c\in \mathbb{N}_{0} :( a\circ b) \circ c\questeq a\circ ( b\circ c)\\
    \\
    \forall a,b,c\in \mathbb{N}_{0} :| | a-b| -c| \questeq | a-| b-c| | 
\end{gather*}

Beweis durch Gegenbeispiel:

Es sei $a=3$, $b=2$ und $c=1$, dann muss nach obiger Annahme gelten
\begin{equation*}
    \begin{array}{ c r c l c }
        & | | 3-2| -1|  & = & | 3-| 2-1| |  & \\
        \Leftrightarrow  & | 1-1|  & = & | 3-1|  & \\
        \Leftrightarrow  & 0 & = & 2 & \lightning 
    \end{array}
\end{equation*}

Daraus folgt, dass es sich bei dem Tuppel $(\mathbb{N}_{0} ,\circ )$ nicht um eine Gruppe und damit auch nicht um eine abelsche Gruppe handelt.

\end{document}
