\documentclass[main.tex]{subfiles}

\begin{document}

\section{Aufgabe 6}
\textit{Hinweis: Aufgabentext zur besseren Verständlichkeit abgeändert.}\\

Eine Luke ist mit einer Platte verschlossen, welche mit zwei Scharniere an den Punkten $S = (0|0|0)$ und $T = (0|4|0)$ montiert ist.
Die Platte hat eine Aufhängung, welche sich im geschlossenen Zustand am Punkt $A = (-3|2|0)$ befindet und im geöffneten Zustand im Punkt $B = (\frac{-3}{\sqrt{2}}|2|\frac{3}{\sqrt{2}})$.

\begin{enumerate}
    \item[(a)] Zeigen Sie, dass die Platte beim Öffnen um $45^{\circ}$ gedreht wird.
    \item[(b)] Wie ist der Abstand zwischen dem Aufhängungspunkt im geschlossenen Zustand $A$ und einem weiteren Punkt $F = (3|-1|6)$, welcher als Befestigung dienen soll?
    \item[(c)] Welcher Punkt $H$ auf der Strecke von $F$ nach $G = (3|8|3)$, hat den geringsten Abstand zum Aufhängungspunkt?
\end{enumerate}

\subsection{Lösung 6}

\subsubsection{Lösung 6a}
Winkel zwischen den Ebenen $E_{STA}\measuredangle E_{STB}$.

\subsubsection{Lösung 6b}
\begin{equation*}
    \left| \vec{AF} \right| = 9
\end{equation*}

\subsubsection{Lösung 6c}
Abstand vom Punkt $A$ nach $f = \vec{FG}$

\begin{equation*}
    \left| \vec{Af} \right| = 5,8
\end{equation*}

\end{document}
