\documentclass[main.tex]{subfiles}

\begin{document}

\section{Aufgabe 6}
\textit{Hinweis: Aufgabentext zur besseren Verständlichkeit abgeändert.}\\

Eine Luke ist mit einer Platte verschlossen, welche mit zwei Scharniere an den Punkten $S = (0|0|0)$ und $T = (0|4|0)$ montiert ist.
Die Platte hat eine Aufhängung, welche sich im geschlossenen Zustand am Punkt $A = (-3|2|0)$ befindet und im geöffneten Zustand im Punkt $B = (\frac{-3}{\sqrt{2}}|2|\frac{3}{\sqrt{2}})$.

\begin{enumerate}
    \item[(a)] Zeigen Sie, dass die Platte beim Öffnen um $45^{\circ}$ gedreht wird.
    \item[(b)] Wie ist der Abstand zwischen dem Aufhängungspunkt im geschlossenen Zustand $A$ und einem weiteren Punkt $F = (3|-1|6)$, welcher als Befestigung dienen soll?
    \item[(c)] Welcher Punkt $H$ auf der Strecke von $F$ nach $G = (3|8|3)$, hat den geringsten Abstand zum Aufhängungspunkt?
\end{enumerate}

\subsection{Lösung 6}
% https://www.geogebra.org/3d/wafqm5em

Zur Veranschaulichung auf GeoGebra:\\
\href{https://www.geogebra.org/3d/wafqm5em}{https://www.geogebra.org/3d/wafqm5em}

\subsubsection{Lösung 6a}
Für den Winkel $\varphi $ zwischen den Ebenen $E_{STA} \measuredangle E_{STB}$ gilt
\begin{equation*}
\cos \varphi =\frac{\langle n_{1} ,n_{2}\rangle }{| n_{1}| \cdotp | n_{2}| }
\end{equation*}
mit $n_{1} ,n_{2}$ als den Normalen der Ebenen.


\begin{gather*}
    \begin{array}{ c l }
        E_{STA} & =\begin{pmatrix}[1]
        0\\
        0\\
        0
        \end{pmatrix} +\lambda \begin{pmatrix}[1]
        0\\
        4\\
        0
        \end{pmatrix} +\mu \begin{pmatrix}[1]
        -3\\
        2\\
        0
        \end{pmatrix}\\
        E_{STB} & =\begin{pmatrix}[1]
        0\\
        0\\
        0
        \end{pmatrix} +\lambda \begin{pmatrix}[1]
        0\\
        4\\
        0
        \end{pmatrix} +\mu \begin{pmatrix}[1]
        -3/\sqrt{2}\\
        2\\
        3/\sqrt{2}
        \end{pmatrix}\\
        n_{1} & =\begin{pmatrix}[1]
        0\\
        4\\
        0
        \end{pmatrix} \times \begin{pmatrix}[1]
        -3\\
        2\\
        0
        \end{pmatrix}\\
        & =\begin{pmatrix}[1]
        0\\
        0\\
        12
        \end{pmatrix}\\
        n_{2} & =\begin{pmatrix}[1]
        0\\
        4\\
        0
        \end{pmatrix} \times \begin{pmatrix}[1]
        -3/\sqrt{2}\\
        2\\
        3/\sqrt{2}
        \end{pmatrix}\\
        & =\begin{pmatrix}[1]
        12/\sqrt{2}\\
        0\\
        12/\sqrt{2}
        \end{pmatrix}
    \end{array}\\
    \\
    \begin{array}{ c r l }
    & \cos \varphi  & =\frac{\langle n_{1} ,n_{2}\rangle }{| n_{1}| \cdotp | n_{2}| }\\
    &  & =\frac{12\cdotp \frac{12}{\sqrt{2}}}{\sqrt{12^{2}} \cdotp \sqrt{\frac{12^{2}}{2} +\frac{12^{2}}{2}}}\\
    &  & =\frac{12\cdotp \frac{12}{\sqrt{2}}}{12^{2}}\\
    &  & =\frac{1}{\sqrt{2}}\\
    \Leftrightarrow  & \varphi  & =\arccos\left(\frac{1}{\sqrt{2}}\right)\\
    &  & =\frac{\pi }{4}
    \end{array}
\end{gather*}
Der Winkel zwischen den Ebenen beträgt im Bogenmaß $\frac{1}{4} \pi $ oder im Gradmaß $45^{\circ}$.

\subsubsection{Lösung 6b}
Die Länge der Strecke zwsichen den Punkten $A$ und $F$ ist gleich dem Betrag des Abstandsvektors $\overrightarrow{AF}$ ihrer Ortsvektoren. Daher gilt:
\begin{equation*}
    \begin{array}{ c l }
    \overrightarrow{AF} & =\begin{pmatrix}[1]
    -3\\
    2\\
    0
    \end{pmatrix} -\begin{pmatrix}[1]
    3\\
    -1\\
    6
    \end{pmatrix}\\
    & =\begin{pmatrix}[1]
    -6\\
    3\\
    -6
    \end{pmatrix}\\
    & \\
    | \overrightarrow{AF}|  & =\sqrt{( -6)^{2} +3^{2} +( -6)^{2}}\\
    & =\sqrt{9+2\cdotp 36}\\
    & =\sqrt{81}\\
    & =9
    \end{array}
\end{equation*}
Das an dem Punkt $F$ befestigte Seil muss mindestens eine Länge von $9\ LE$ haben.

\subsubsection{Lösung 6c}
% Korrektur 2022-11-28 berücksichtigt

Für den Abstand $d$ zwischen dem Punkt $A$ und der Geraden $g_{FG}$ gilt allgemein:
\begin{equation*}
    d=\frac{\left|(\vec{a} - \vec{f}) \times \overrightarrow{FG}\right|}{\left| \overrightarrow{FG} \right|}
\end{equation*}

Die orthogonale Projektion von $\vec{FA}$ auf den Richtungsvektor der Graden $g: X = \vec{OF} + \lambda \cdotp \vec{FG}$ ist die Strecke zwischen dem Aufpunkt $F$ der Geraden und dem gesuchten Punkt $H$.

\begin{gather*}
    \overrightarrow{FG} =\begin{pmatrix}[1]
    3-3\\
    8-(-1)\\
    3-6
    \end{pmatrix} =\begin{pmatrix}[1]
    0\\
    9\\
    -3
    \end{pmatrix}\\
    \overrightarrow{FA} =\begin{pmatrix}[1]
    3-( -3)\\
    ( -1) -2\\
    6-0
    \end{pmatrix} =\begin{pmatrix}[1]
    6\\
    -3\\
    6
    \end{pmatrix}\\
\end{gather*}

Die Formel der orthogonalen Projektion lautet:
\begin{gather*}
    p_{b}(a) = \frac{ \left\langle a,b \right\rangle }{ \left\langle b,b \right\rangle } \cdotp b
\end{gather*}

Das bedeutet in diesem Fall:
\begin{equation*}
	\begin{array}{ r l }
	p_{\vec{FG}}(\vec{FA}) & =\frac{\left\langle \vec{FA},\vec{FG} \right\rangle}{\left\langle \vec{FG},\vec{FG} \right\rangle} \cdotp \vec{FG} \\
		& = \frac{-27-18}{90} \cdotp \begin{pmatrix}[1]
			0 \\ 9 \\ -3 \end{pmatrix}\\
		& = -\frac{1}{2} \begin{pmatrix}[1]
			0 \\ 9 \\ -3 \end{pmatrix}
	\end{array}
\end{equation*}

Der Punkt $H$ befindet sich nun an der Stelle $H = \vec{FA} - p_{\vec{FG}}(\vec{FA}) + A$:

	$$
		H = \vec{FA} - p_{\vec{FG}}(\vec{FA}) + A \\
		= \begin{pmatrix}[1] 6 \\ -3 \\ 6 \end{pmatrix}
		+ \begin{pmatrix}[1] 0 \\ 9/2 \\ -3/2 \end{pmatrix}
		+ \begin{pmatrix}[1] -3 \\ 2 \\ 0 \end{pmatrix}\\
		= \begin{pmatrix}[1] 3 \\ 7/2 \\ 9/2 \end{pmatrix}
	$$

\end{document}
