\documentclass[main.tex]{subfiles}

\begin{document}

\section{Aufgabe 6}
\textit{Hinweis: Aufgabentext zur besseren Verständlichkeit abgeändert.}\\

Eine Luke ist mit einer Platte verschlossen, welche mit zwei Scharniere an den Punkten $S = (0|0|0)$ und $T = (0|4|0)$ montiert ist.
Die Platte hat eine Aufhängung, welche sich im geschlossenen Zustand am Punkt $A = (-3|2|0)$ befindet und im geöffneten Zustand im Punkt $B = (\frac{-3}{\sqrt{2}}|2|\frac{3}{\sqrt{2}})$.

\begin{enumerate}
    \item[(a)] Zeigen Sie, dass die Platte beim Öffnen um $45^{\circ}$ gedreht wird.
    \item[(b)] Wie ist der Abstand zwischen dem Aufhängungspunkt im geschlossenen Zustand $A$ und einem weiteren Punkt $F = (3|-1|6)$, welcher als Befestigung dienen soll?
    \item[(c)] Welcher Punkt $H$ auf der Strecke von $F$ nach $G = (3|8|3)$, hat den geringsten Abstand zum Aufhängungspunkt?
\end{enumerate}

\subsection{Lösung 6}
% https://www.geogebra.org/3d/wafqm5em

\subsubsection{Lösung 6a}
Für den Winkel $\varphi $ zwischen den Ebenen $E_{STA} \measuredangle E_{STB}$ gilt
\begin{equation*}
\cos \varphi =\frac{\langle n_{1} ,n_{2}\rangle }{| n_{1}| \cdotp | n_{2}| }
\end{equation*}
mit $n_{1} ,n_{2}$ als den Normalen der Ebenen.


\begin{gather*}
    \begin{array}{ c l }
        E_{STA} & =\begin{pmatrix}[1]
        0\\
        0\\
        0
        \end{pmatrix} +\lambda \begin{pmatrix}[1]
        0\\
        4\\
        0
        \end{pmatrix} +\mu \begin{pmatrix}[1]
        -3\\
        2\\
        0
        \end{pmatrix}\\
        E_{STB} & =\begin{pmatrix}[1]
        0\\
        0\\
        0
        \end{pmatrix} +\lambda \begin{pmatrix}[1]
        0\\
        4\\
        0
        \end{pmatrix} +\mu \begin{pmatrix}[1]
        -3/\sqrt{2}\\
        2\\
        3/\sqrt{2}
        \end{pmatrix}\\
        n_{1} & =\begin{pmatrix}[1]
        0\\
        4\\
        0
        \end{pmatrix} \times \begin{pmatrix}[1]
        -3\\
        2\\
        0
        \end{pmatrix}\\
        & =\begin{pmatrix}[1]
        0\\
        0\\
        12
        \end{pmatrix}\\
        n_{2} & =\begin{pmatrix}[1]
        0\\
        4\\
        0
        \end{pmatrix} \times \begin{pmatrix}[1]
        -3/\sqrt{2}\\
        2\\
        3/\sqrt{2}
        \end{pmatrix}\\
        & =\begin{pmatrix}[1]
        12/\sqrt{2}\\
        0\\
        12/\sqrt{2}
        \end{pmatrix}
    \end{array}\\
    \\
    \begin{array}{ c r l }
    & \cos \varphi  & =\frac{\langle n_{1} ,n_{2}\rangle }{| n_{1}| \cdotp | n_{2}| }\\
    &  & =\frac{12\cdotp \frac{12}{\sqrt{2}}}{\sqrt{12^{2}} \cdotp \sqrt{\frac{12^{2}}{2} +\frac{12^{2}}{2}}}\\
    &  & =\frac{12\cdotp \frac{12}{\sqrt{2}}}{12^{2}}\\
    &  & =\frac{1}{\sqrt{2}}\\
    \Leftrightarrow  & \varphi  & =\arccos\left(\frac{1}{\sqrt{2}}\right)\\
    &  & =\frac{\pi }{4}
    \end{array}
\end{gather*}
Der Winkel zwischen den Ebenen beträgt im Bogenmaß $\frac{1}{4} \pi $ oder im Gradmaß $45^{\circ}$.

\subsubsection{Lösung 6b}
Die Länge der Strecke zwsichen den Punkten $A$ und $F$ ist gleich dem Betrag des Abstandsvektors $\overrightarrow{AF}$ ihrer Ortsvektoren. Daher gilt:
\begin{equation*}
    \begin{array}{ c l }
    \overrightarrow{AF} & =\begin{pmatrix}[1]
    -3\\
    2\\
    0
    \end{pmatrix} -\begin{pmatrix}[1]
    3\\
    -1\\
    6
    \end{pmatrix}\\
    & =\begin{pmatrix}[1]
    -6\\
    3\\
    -6
    \end{pmatrix}\\
    & \\
    | \overrightarrow{AF}|  & =\sqrt{( -6)^{2} +3^{2} +( -6)^{2}}\\
    & =\sqrt{9+2\cdotp 36}\\
    & =\sqrt{81}\\
    & =9
    \end{array}
\end{equation*}
Das an dem Punkt $F$ befestigte Seil muss mindestens eine Länge von $9\ LE$ haben.

\subsubsection{Lösung 6c}
% \textbf{Ansatz 1:}
% Wir modelieren die gerade Verbindung zwischen den Punkten $F$ und $G$ als Gerade $g$, mit der folgenden Geradengleichung: 
% \begin{equation*}
%     \begin{array}{ c c l }
%     g: & \vec{x} & =\vec{f} +\lambda \cdotp \overrightarrow{FG}\\
%     g_{FG} : & \vec{x} & =\begin{pmatrix}[1]
%     3\\
%     -1\\
%     6
%     \end{pmatrix} +\lambda \begin{pmatrix}[1]
%     3-3\\
%     8-( -1)\\
%     3-6
%     \end{pmatrix}\\
%     &  & =\begin{pmatrix}[1]
%     3\\
%     -1\\
%     6
%     \end{pmatrix} +\lambda \begin{pmatrix}[1]
%     0\\
%     9\\
%     -3
%     \end{pmatrix}
%     \end{array}
% \end{equation*}
% Für den Abstand $d( \lambda )$ zwischen dem Punkt $A$ und einem beliebigen Punkt $H\in g$ ist 
% \begin{equation*}
% \begin{array}{ c c l }
% d( \lambda ) & = & \sqrt{(( 3+\lambda \cdotp 0) -( -3))^{2} +(( -1+\lambda \cdotp 9) -2)^{2} +(( 6+\lambda \cdotp ( -3)) -0)^{2}}\\
%  & = & \sqrt{6^{2} +( \lambda \cdotp 9-3)^{2} +( 6-3\lambda )^{2}}\\
%  & = & \sqrt{36+9\lambda ^{2} -54\lambda +9+36-36\lambda +9\lambda ^{2}}\\
%  & = & \sqrt{81+18\lambda ^{2} -90\lambda }\\
%  & = & \sqrt{81+18\lambda ^{2} -90\lambda }
% \end{array}
% \end{equation*}
% Es gilt das Minimum der Abstandsfunktion $d( \lambda )$ zu bestimmen. Da ein Abstand immer positiv ist, kann auch das Minimum von $d^{2}( \lambda )$ bestimmt werden.
% \begin{equation*}
% \begin{array}{ c c }
% d^{2}( \lambda ) & =18\lambda ^{2} -90\lambda +81
% \end{array}
% \end{equation*}
% Das Minimum der Parabel liegt am Scheitelpunkt $S$
% \begin{equation*}
% \begin{array}{ c l }
% S & =\left( -\frac{90}{2\cdotp 18}\middle| 81-\frac{90^{2}}{4\cdotp 18}\right)\\
%  & =\left( -\frac{5}{2}\middle| -\frac{63}{2}\right)
% \end{array}
% \end{equation*}
% Somit ist der minimale Abstand an der Stelle $\lambda =-\frac{5}{2}$. Daraus folgt für den Punkt $H\in g$:
% \begin{equation*}
% \begin{array}{ c l }
% \vec{h} & =\begin{pmatrix}[1]
% 3\\
% -1\\
% 6
% \end{pmatrix} -\frac{5}{2}\begin{pmatrix}[1]
% 0\\
% 9\\
% -3
% \end{pmatrix}\\
%  & =\begin{pmatrix}[1]
% 3\\
% \frac{-43}{2}\\
% \frac{27}{2}
% \end{pmatrix}
% \end{array}
% \end{equation*}
% Mit dem Abstand $| \vec{h}| =\sqrt{3^{2} +\left(\frac{-43}{2}\right)^{2} +\left(\frac{27}{2}\right)^{2}} \approx 25,5636\dotsc $



% \textbf{Ansatz 2}

Für den Abstand $d$ zwischen dem Punkt $A$ und der Geraden $g_{FG}$ gilt allgemein:
\begin{equation*}
    d=\frac{| (\vec{a} -\vec{f}) \times \overrightarrow{FG}| }{| \overrightarrow{FG}| }
\end{equation*}
In diesem Fall also:
\begin{equation*}
    \begin{array}{ c l }
    d & =\frac{\left| \left(\begin{pmatrix}[1]
    -3\\
    2\\
    0
    \end{pmatrix} -\begin{pmatrix}[1]
    3\\
    -1\\
    6
    \end{pmatrix}\right) \times \begin{pmatrix}[1]
    0\\
    9\\
    -3
    \end{pmatrix}\right| }{\left| \begin{pmatrix}[1]
    0\\
    9\\
    -3
    \end{pmatrix}\right| }\\
    & =\frac{\left| \begin{pmatrix}[1]
    -6\\
    3\\
    -6
    \end{pmatrix} \times \begin{pmatrix}[1]
    0\\
    9\\
    -3
    \end{pmatrix}\right| }{\sqrt{81+9}}\\
    & =\frac{\left| \begin{pmatrix}[1]
    45\\
    -18\\
    -54
    \end{pmatrix}\right| }{\sqrt{90}}\\
    & =\frac{\sqrt{45^{2} +( -18)^{2} +( -54)^{2}}}{\sqrt{90}}\\
    & =\sqrt{\frac{45^{2} +18^{2} +54^{2}}{90}}\\
    & =\sqrt{\frac{117}{2}}\\
    & \approx 7,6485\dotsc 
    \end{array}
\end{equation*}

Da aber nach der Position von dem Punkt $H$ und nicht nach der Distanz gefragt war, muss so vorgegangen werden:

Die orthogonale Projektion von $\vec{FG}$ auf $\vec{FA}$

\begin{gather*}
    \overrightarrow{FG} :=a=\begin{pmatrix}[1]
    3-3\\
    ( -1) -8\\
    6-3
    \end{pmatrix} =\begin{pmatrix}[1]
    0\\
    -9\\
    3
    \end{pmatrix}\\
    \overrightarrow{FA} :=b=\begin{pmatrix}[1]
    3-( -3)\\
    ( -1) -2\\
    6-0
    \end{pmatrix} =\begin{pmatrix}[1]
    6\\
    -3\\
    6
    \end{pmatrix}\\
    \\
    a_{b} =\frac{\langle a,b\rangle }{|| b| |} \cdotp b\\
    \\
    \Rightarrow a_{b} =\frac{0+27+18}{\sqrt{6^{2} +( -3)^{2} +6^{2}}} \cdotp \begin{pmatrix}[1]
    6\\
    -3\\
    6
    \end{pmatrix} =\frac{45}{\sqrt{81}} \cdotp \begin{pmatrix}[1]
    6\\
    -3\\
    6
    \end{pmatrix} =5\cdotp \begin{pmatrix}[1]
    6\\
    -3\\
    6
    \end{pmatrix} =\begin{pmatrix}[1]
    30\\
    -15\\
    30
    \end{pmatrix}
\end{gather*}    



\end{document}
