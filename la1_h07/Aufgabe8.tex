\documentclass[main.tex]{subfiles}

\begin{document}

\section{Aufgabe 8}
Gegeben sind die Vektoren

\begin{align*}
    a=\begin{pmatrix}[1]
    1\\
    -1/2\\
    \beta
    \end{pmatrix} & & b=\begin{pmatrix}[1]
    0\\
    2\alpha \\
    -2
    \end{pmatrix} & & c=\begin{pmatrix}[1]
    -1\\
    -\alpha \\
    1
    \end{pmatrix}
\end{align*}

Bestimmen Sie die Variablen $\alpha$ und $\beta$ derart, dass der aus den 3 Vektoren gebildete Spat das Volumen 17 VE hat und das von den Vektoren $a$ und $b$ aufgespannte Parallelogramm den Flächeninhalt 19 FE hat.

\subsection{Lösung 8}

% Ansatz: 2.43 und Abbildung 2.6
% 2.91 und Abbildung 2.8

Das Spatprodukt dreier Vektoren $a,b,c\in \mathbb{R}^{3}$ ist nach Definition 2.90 und 2.91
\begin{equation*}
    \det( a,b,c) =\langle a,b\times c\rangle \in \mathbb{R}
\end{equation*}
Das Volumen $V$ des durch die Spaltenvektoren aufgespannten Spats ist der Betrag der Determinante $V=| \det( a,b,c)| $.

Für ein gegebenes Volumen $V=17VE$ soll also gelten:
\begin{equation*}
    \begin{array}{ c r l }
    & 17VE & =\left| \det\begin{pmatrix}[1]
    1 & 0 & -1\\
    -1/2 & 2\alpha  & -\alpha \\
    \beta  & -2 & 1
    \end{pmatrix}\right| \\
    \Leftrightarrow  & 17VE & =| 2\alpha \beta -1| \\
    \overset{( 2\alpha \beta -1)  >0}{\Leftrightarrow } & 9VE & =\alpha \beta \\
    \Leftrightarrow  & \beta  & =\frac{9VE}{\alpha }
    \end{array}
\end{equation*}

Für den Flächeninhalt des Parallelogramms $A$ muss die Länge des Vektorprodukts bestimmt werden. Nach Folgerung 2.43 lässt sich der Flächeninhalt so berechnen:
\begin{equation*}
    A=\| a\| \cdotp \| b\| \cdotp \sin \theta =\| a\times b\| 
\end{equation*}

Für eine gegebene Fläche $A=19FE$ soll also gelten:
\begin{equation*}
    \begin{array}{ c r l }
    & A & = \left\| \begin{pmatrix}[1]
    1\\
    -1/2\\
    \beta 
    \end{pmatrix} \times \begin{pmatrix}[1]
    0\\
    2\alpha \\
    -2
    \end{pmatrix} \right\| \\
    &  & = \left\| \begin{pmatrix}[1]
    1-2\alpha \beta \\
    2\\
    2\alpha 
    \end{pmatrix} \right\| \\
    &  & =\sqrt{{( 1-2\alpha \beta )}^{2} +4+4\alpha ^{2}}\\
    &  & =\sqrt{5+4\alpha ^{2} \beta ^{2} -4\alpha \beta +4\alpha ^{2}}\\
    \overset{A=19FE}{\Leftrightarrow } & 19FE & =\sqrt{5+4\alpha ^{2} \beta ^{2} -4\alpha \beta +4\alpha ^{2}}\\
    \overset{\beta =9VE/\alpha }{\Leftrightarrow } & 19FE & =\sqrt{5+4\alpha ^{2}{\left(\frac{9VE}{\alpha }\right)}^{2} -4\alpha \left(\frac{9VE}{\alpha }\right) +4\alpha ^{2}}\\
    \Leftrightarrow  & 19FE & =\sqrt{5+4\cdotp {( 9VE)}^{2} -4\cdotp ( 9VE) +4\alpha ^{2}}\\
    \Leftrightarrow  & 19^{2} VE & =5+4\cdotp {( 9VE)}^{2} -4\cdotp ( 9VE) +4\alpha ^{2}\\
    \Leftrightarrow  & 4\alpha ^{2} & =19^{2} VE-5-4\cdotp {( 9VE)}^{2} +4\cdotp ( 9VE)\\
    \Leftrightarrow  & \alpha ^{2} & =\frac{19^{2} VE}{4} -{( 9VE)}^{2} +( 9VE) -\frac{5}{4}\\
    \Leftrightarrow  & \alpha ^{2} & =\frac{19^{2} VE}{4} +( 9VE) -{( 9VE)}^{2} -\frac{5}{4}\\
    \Leftrightarrow  & \alpha ^{2} & =\left(\frac{397}{4} VE\right) -\left( 81VE^{2}\right) -\frac{5}{4}\\
    \Leftrightarrow  & \alpha  & =\sqrt{\left(\frac{397}{4} VE\right) -\left( 81VE^{2}\right) -\frac{5}{4}}\\
    \Leftrightarrow  & \alpha  & =\sqrt{17}
    \end{array}
\end{equation*}


Somit sind die Parameter $\alpha =\sqrt{17}$ und $\beta =\frac{9}{\sqrt{17}}$ eindeutig bestimmt. 

Die Probe ergibt für das Volumen

\begin{equation*}
    \begin{array}{ r l }
    17VE & =\left| \det\begin{pmatrix}[1]
    1 & 0 & -1\\
    -1/2 & 2\sqrt{17} & -\sqrt{17}\\
    \frac{9}{\sqrt{17}} & -2 & 1
    \end{pmatrix}\right| \\
    & =\left| 2\cdotp \sqrt{17} \cdotp \frac{9}{\sqrt{17}} -1\right| \\
    & =| 18-1| \\
    & =17\ \ \checked 
    \end{array}
\end{equation*}
sowie für den Flächeninhalt

\begin{equation*}
    \begin{array}{ c r l }
        19FE & = & \left\| \begin{pmatrix}[1]
        1-2\cdotp \sqrt{17} \cdotp \frac{9}{\sqrt{17}}\\
        2\\
        2\cdotp \sqrt{17}
        \end{pmatrix} \right\| \\
        & = & \left\| \begin{pmatrix}[1]
        -17\\
        2\\
        2\cdotp \sqrt{17}
        \end{pmatrix} \right\| \\
        & = & \sqrt{{( -17)}^{2} +2^{2} +2^{2} \cdotp 17}\\
        & = & \sqrt{{( -17)}^{2} +4+4\cdotp 17}\\
        & = & \sqrt{361}\\
        & = & 19\ \checked 
    \end{array}
\end{equation*}

\textit{Hinweis: Es reicht die Betrachtung des Falls $(2\alpha\beta-1) > 0$, da die Aufgabenstellung nach einer Lösung und nicht nach allen Lösungen fragt.}

\end{document}
