\section{Aufgabe 2}

Untersuchen Sie mit Hilfe des Minoranten-/Majorantenkriteriums auf Konvergenz:
\begin{align*}
  \begin{array}{ c c }
    \text{a)} & \sum\limits _{k=1}^{\infty }\frac{3k+2}{k^{2} +5k-1}
  \end{array} & & \begin{array}{ c c }
    \text{b)} & \sum\limits _{k=1}^{\infty }\frac{2\cos( k) +\sin^{2}( k)}{3k^{2}}
  \end{array}
\end{align*}

\subsection{Lösung 2}

\underline{Majorantenkriterium:} 

Sei $\displaystyle \sum _{n=0}^{\infty } b_{n} :b_{n} \geq 0\ \forall n\in \mathbb{N}$ konvergent, dann gilt

$\displaystyle \forall n\geq n_{0} :|a_{n} |\leq b_{n} \Rightarrow \sum _{n=0}^{\infty } a_{n}$ konvergent absolut



\underline{Minorantenkriterium:}

Sei $\displaystyle \sum _{n=0}^{\infty } b_{n} :b_{n} \geq 0\ \forall n\in \mathbb{N}$ divergent, dann gilt

$\displaystyle \forall n\geq n_{0} :a_{n} \geq b_{n} \Rightarrow \sum _{n=0}^{\infty } a_{n}$ divergent.

\subsubsection{2a)} 

Betrachen wir die gegebene Reihe mit dem Minorantenkriterium
\begin{gather*}
  \sum\limits _{k=1}^{\infty }\frac{3k+2}{k^{2} +5k-1} \geq \sum\limits _{k=1}^{\infty }\frac{3k}{k^{2} +5k} \geq \sum\limits _{k=1}^{\infty }\frac{3k}{k\cdotp ( k +5)} =\sum\limits _{k=1}^{\infty }\frac{3}{( k +5)}\\
  \\
  \\
  \sum\limits _{k=1}^{\infty }\frac{3}{( k +5)} =\sum\limits _{k=6}^{\infty }\frac{3}{k} =\underbrace{\sum\limits _{k=1}^{\infty }\frac{3}{k}}_{\text{divergiert}\rightarrow \infty } -\underbrace{\sum\limits _{k=1}^{5}\frac{3}{k}}_{\text{konvergiert}}
\end{gather*}
Somit ist gezeigt, dass die Folge $\displaystyle \sum\nolimits _{k=1}^{\infty }\frac{1}{k}$ immer kleiner oder gleich der untersuchten Folge ist.\\

Da wir wissen, dass diese Folge gegen Unendlich divergiert und gezeigt haben, dass die untersuchte Folge immer größer ist, muss die untersuchte Folge auch gegen Unendlich divergieren. 


\subsubsection{2b)}

Wir untersuchen die gegebene Reihe mit dem Majorantenkriterium auf Konvergenz:
\begin{equation*}
  \sum\limits _{k=1}^{\infty }\frac{2\cos( k) +\sin^{2}( k)}{3k^{2}} \leq \sum\limits _{k=1}^{\infty }\frac{2+1}{3k^{2}} =\sum\limits _{k=1}^{\infty }\frac{3}{3k^{2}} =\underbrace{\sum\limits _{k=1}^{\infty }\frac{1}{k^{2}}}_{\text{konvergiert }}
\end{equation*}
Somit ist gezeigt, dass die Folge $\displaystyle \sum\nolimits _{k=1}^{\infty }\frac{1}{k^{2}}$ immer größer oder gleich der untersuchten Folge ist.\\

Da bekannt ist, dass jene Folge gegen $\displaystyle \frac{\pi ^{2}}{6}$ konvergiert und wir gezeigt haben, dass die untersuchte Folge immer kleiner ist, muss die untersuchte Folge auch konvergent sein.