\section{Aufgabe 1}

Konvergieren oder divergieren die folgenden rekursiven Folgen $\displaystyle ( n\in \mathbb{N}_{o})$? Berechnen Sie gegebenenfalls den Grenzwert.
\begin{align*}
  \begin{array}{ c c }
    \text{a)} & a_{n+1} =\sqrt{4a_{n} -3} \ \text{mit} \ a_{o} =3
  \end{array} & & \begin{array}{ c c }
    \text{b)} & a_{n+1} =\frac{n+2}{n+1} \cdot a_{n} \ \text{mit} \ a_{0} =1
  \end{array}
\end{align*}

\subsection{Lösung 1a}

Die Berechnung der ersten Folgenglieder lässt vermuten, dass die explizite Form der Folge als $\displaystyle a_{n} =3$ geschrieben werden kann. Dies soll im Folgenden durch vollständige Induktion bewiesen werden:\\

\textbf{Induktionsanfang: }

Für $\displaystyle n_{o} =0$ gilt $\displaystyle a_{0} =3\ \checkmark $

\textbf{Induktionsannahme:}

Es gelte für ein festes, aber beliebiges $\displaystyle n\geq n_{0} ,\ n\in \mathbb{N}_{0}$
\begin{equation*}
a_{n} =3\text{.}
\end{equation*}
\textbf{Induktionsbehauptung:}

Dann gilt für $\displaystyle n+1$ auch
\begin{equation*}
a_{n+1} =3\text{.}
\end{equation*}
\textbf{Beweis des Induktionsschritts:}
\begin{equation*}
a_{n+1} =\sqrt{4a_{n} -3} \ \xlongequal{IV}\sqrt{4\cdotp 3-3} =\sqrt{9} =3\ \checkmark 
\end{equation*}
Damit ist gezeigt, dass die Aussage $\displaystyle a_{n} =3$ für alle $\displaystyle n\in \mathbb{N}$ erfüllt ist, sie also allgemeingültig ist.\\

Der Grenzwert der Folge ist
\begin{equation*}
\lim _{n\rightarrow \infty } a_{n} =\lim _{n\rightarrow \infty } 3=3\text{.}
\end{equation*}
Die Folge $\displaystyle a_{n}$ ist konstant und eine konstante Folge ist konvergent.



\subsection{Lösung 1b}

Für die ersten Folgenglieder ergibt sich
\begin{gather*}
a_{1} =\frac{2}{1} \cdot 1=2\\
a_{2} =\frac{3}{2} \cdot 2=3\\
a_{3} =\frac{4}{3} \cdot 3=4
\end{gather*}
sodass sich vermuten lässt, dass die explizite Form der Folge als $\displaystyle a_{n} =n+1$ behauptet werden kann. Dies soll im Folgenden durch vollständige Induktion bewiesen werden: \\

\textbf{Induktionsanfang: }

Für $\displaystyle n_{o} =0$ gilt $\displaystyle a_{0} =0+1=1\ \checkmark $

\textbf{Induktionsannahme:}

Es gelte für ein festes, aber beliebiges $\displaystyle n\geq n_{0} ,\ n\in \mathbb{N}_{0}$
\begin{equation*}
a_{n} =n+1\text{.}
\end{equation*}
\textbf{Induktionsbehauptung:}

Dann gilt für $\displaystyle n+1$ auch
\begin{equation*}
a_{n+1} =( n+1) +1\text{.}
\end{equation*}
\textbf{Beweis des Induktionsschritts:}
\begin{equation*}
a_{n+1} =\frac{n+2}{n+1} \cdot a_{n}\xLeftrightarrow{IV}( n+1) +1=\frac{n+2}{n+1} \cdot ( n+1) \Leftrightarrow \ ( n+1) +1=n+2\ \checkmark 
\end{equation*}
Somit ist gezeigt, dass aus der Induktionsannahme stets die Induktionsbehauptung folgt und zwar für alle $\displaystyle n\in \mathbb{N}$.\\

Versucht man nun den Grenzwert der Folge zu mit der zuvor bewiesenen, expliziten Form zu bestimmen, so zeigt sich, dass die Folge keinen Grenzwert hat.
\begin{equation*}
\lim _{n\rightarrow \infty } a_{n} =\lim _{n\rightarrow \infty } n+1=\infty 
\end{equation*}
Die Folge $\displaystyle a_{n}$ ist somit divergent.